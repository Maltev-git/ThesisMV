\chapter{Solar wind impact on the magnetosphere}
%\label{chap:}

motivation (see also introduction): solar wind impact nowcast, CME impact forecast\\
%COFI -- chapter outline and flow integration\\


\section{Solar wind interaction processes with the magnetosphere}
%\label{sec:}

%theory
As is known for some time the terrestrial magnetic field shows disturbances that are caused by solar wind (see \autoref{sec:solar_influence_on_earth}).\\

there are several underlying physical mechanisms, whose contribution is not yet quantified'?'\\
% physical mechanisms:\\
% - reconnection\\
% - compression\\
% - turbulence\\
% - induction?\\

three ways for solar wind momentum and energy transfer into magnetosphere:\\
- sw entering sphere\\
- waves/eddies\\
- reconnection\\

%[the relation of these events was confirmed two decades ago by (who and \citet{Bothmer1993})]\\

-> example events CIR/HSS and CME\\


\section{Solar wind--magnetosphere coupling functions}
\label{sec:solar_wind_magnetosphere_coupling_functions}

%theory from literature
coupling mechanisms and therefrom derived functions\\
VBzgsm - E-field...\\

Studies finding best coupling function, Newell, etc.\\

\citet{Newell2007}: universal sw-magnetosphere coupling function (opening flux rate)\\
\citet{Newell2008}: coupling function merging and viscous term\\
merging and viscous terms (reconnection and turbulence)\\
merging term: rate magnetic flux is opened at the magnetopause ($d\Phi_\text{MP}/dt$)\\
viscous term: reconnection due to Kelvin-Helmholtz instabilies at the boundary ($n^{1/2} v^2$)\\
equation for the least variance linear prediction of Kp: $Kp = 0.05 + \num{2.244e-4} d\Phi_\text{MP}/dt + \num{2.844e-6} n^{1/2} v^2$\\
combination of both terms works best (r = 0.866)\\

see also \autoref{sec:solar_wind_magnetosphere_coupling}\\


\section{Parameter selection}

%fix usage of Kp index\\
In our analyses we use the planetary geomagnetic disturbance indicator Kp (see Section~XX...), because it is designed to measure solar particle radiation by its magnetic effects (cite? Bartels...?).\\
and close relation to aurorae, NOAA scale\\

choose solar wind parameters based on coupling functions\\
V, B, Bzgsm, N, T\\


\section{Data selection}

choose data sets, data resolution and time period\\

geomagnetic disturbance index Kp: magnetic field maximal variation range within 3~hours; 3-hourly index\\
need high resolution solar wind data (e.g. 1~min) to be able to determine maximal values/variations within 3~hours\\

choosing data time range\\
the Kp time series started in XXXX, when there were no spacecraft to measure in situ solar wind --> time range defined by available solar wind data\\
OMNI data set -> longest continuous solar wind data set\\


data processing\\
choosing data averaging duration and method (min/avg/max)\\
OMNI 1min data to 3hmin/max\\
now: 3~hour solar wind extreme value, to back up argument refer to cc table (same data period, different resolution/measure (measure=mean, max or min))\\
		(tested: 3"~hour solar wind variation range gives 8~\% lower cc)\\


\section{General Kp correlations}

general frequency distributions\\

general correlation coefficients + plots\\

general cc table\\

\section{Solar cycle influence}

solar cycle dependence\\
- parameter time plots\\
- parameter vs SSN matrix-plots\\
- cc time plots\\
- cc vs SSN plots\\

\section{Isolating the CME influence}

the causes of the strongest geomagnetic storms are draping and magnetic clouds of CMEs \citep{Bothmer1993}\\

-> example event CME\\

\subsection{Solar wind structure list}

solar wind structures (SWS) OMNI list of Ian~Richardson \citep{Richardson2012}\\
open source 1995--2015 \url{http://www.srl.caltech.edu/ACE/ASC/DATA/level3/icmetable2.htm}\\
restricted source 1963--2015 \url{http://cedarweb.vsp.ucar.edu/wiki/index.php/Tools_and_Models:Solar_Wind_Structures} \citet{Richardson2012}\\	%Rules of the Road: Please contact Ian Richardson about your use of this data.

overall CME fraction of solar wind - compute (for used period) from file $sws_swstruc_yearly_63001_14035.txt$\\
e.g. 2013: 0.220 CMEs, 0.236 CIRs/HSSs and 0.544 slows (low speed streams - LSSs)\\

\subsection{CME correlations}

same analysis for CMEs\\
- parameter time plots\\
- parameter vs SSN matrix-plots\\
- cc time plots\\
- cc vs SSN plots\\

