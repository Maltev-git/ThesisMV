
\chapter*{Acknowledgements}
\addcontentsline{toc}{chapter}{\texorpdfstring{\color{gray}Acknowledgements}{Acknowledgements}}
%\addchap{\texorpdfstring{\color{gray}Acknowledgments}{Acknowledgments2}}
%\addchap{\hyperref[sec:toc]{\color{gray}Acknowledgments}}

I thank\\
my supervisor Volker~Bothmer\\
the comittee (Ansgar~Reiners, second?)\\
the proofreaders\\
The author like to thank the proofreaders for very helpful comments and suggestions.\\
the 'Büro' for cozily accomodating me all the long hours\\

for creating and making scientific data available:\\
- SWS list (see Acknowledgements in Elliott2013; directly Ian~Richardson)\\
- OMNI data (see paper)\\
- ACE data\\
- Helios data (see paper)\\
- Kp data\\
- SSN data\\

I acknowledge the financial support from the projects\\
AFFECTS: Solar wind correlation with magnetosphere\\
CGAUSS: Solar wind model and extrapolation\\
HELCATS: Minimum variance analyses of magnetic clouds\\
OPTIMAP: Solar wind ACE time series\\


This work was done within the scope of the CGAUSS project, funded by the German Aerospace Center (grant number...).\\
The authors thank the Helios and OMNI teams for creating and making availabe the solar wind in situ data. The Helios and the OMNI data sets were supplied by the Space Physics Data Facility at NASA's Goddard Space Flight Center.\\
We thank the 'referees' for /important comments and suggestions. /careful review of this paper.\\

facilities/persons which made available the Helios and OMNI data\\
``Ian Richardson and Hilary Cane for making their Interplanetary Coronal Mass Ejection list readily available''\\
``Joseph King and Natalia Papitashvili for creating the combined OMNI data set''\\
The OMNI data were downloaded from NASA's Space Physics Data Facility ...\\
``We thank the reviewers for the careful review of this paper''\\
The Authors extend their thanks to the referee for important comments and suggestions.\\
This work was supported by the DLR CGAUSS project\\
We would like to thank XX for reading drafts of this manuscript...\\
This work has been done in the frame of the CGAUSS project (url), funded by the DLR\\
We would like to thank the Helios team and NASA's SPDF for supplying the plasma and magnetic field data (url)\\


The results presented in this thesis rely on the \Kp{}~index, calculated and made available by the German Research Centre for Geosciences in Potsdam from data collected at magnetic observatories. We thank the involved national institutes, the INTERMAGNET network and ISGI (isgi.unistra.fr).\\



acknowledgements to MAG/SWEPAM Team for ACE data in basics plots...\\

Helios data acknowledgments:\\
Solar wind data courtesy of R.~Schwenn, Max-Planck-Institut für Aeronomie, Lindau, magnetic field data courtesy of F.~Neubauer, Universität zu Köln. (see paper...)\\

Data acknowledgments for \autoref{chap:chapter2}:\\
%\section{Acknowledgments}
The research leading to these results has received funding from the European Union's Seventh Framework Programme (FP7/2007-2013) under the grant agreement number 263506 (AFFECTS). The results presented in this paper rely on the \Kp{}~index, calculated and made available by the German Research Centre for Geosciences in Potsdam from data collected at magnetic observatories. We thank the involved national institutes, the INTERMAGNET network and ISGI (isgi.unistra.fr). The authors thank the OMNI PIs/teams for creating and making available the solar-wind in-situ data. The OMNI data are supplied by the NASA Space Science Data Coordinated Archive and the Space Physics Data Facility at NASA's Goddard Space Flight Center. Additional thanks for maintaining and providing the international sunspot number series goes to the World Data Center -- Sunspot Index and Long-term Solar Observations at the Solar Influences Data Analysis Center (SIDC), Royal Observatory of Belgium. The hourly solar wind structure list was kindly provided by Ian~Richardson of the NASA Goddard Space Flight Center and CRESST/University of Maryland via the CEDAR Database at the National Center for Atmospheric Research, which is supported by the National Science Foundation.


