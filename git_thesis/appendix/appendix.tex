
\chapter{Appendix}
\label{chap:appendix}

\section{Solar surface differential rotation}
\label{sec:solar_surface_differential_rotation}

%solar rotation
The solar differential rotation is visible on the surface and was first discovered from sunspot observations by \citet{Scheiner1630}. (double)\\

%rotation period
\citet{Bartels1934} set the synodic solar rotation period to 27~days for the definition of his solar rotation number. The Bartels' Rotation Number counts the solar rotations starting with 8 February 1832.\\
Carrington solar rotation period of 27.2753~days (Where Carrington Rotation Number is based upon, starting with November 9, 1853; Wikipedia...)\\
%synodic rotation period of 27.2753 days (or a sidereal period of 25.38 days; This chosen period roughly corresponds to rotation at a latitude of 16~deg (sic), which is consistent with the typical latitude of sunspots and corresponding periodic solar activity; cite from https://en.wikipedia.org/wiki/Solar_rotation

Solar surface rotation period at 16\degree{} latitude:\\
sidereal: 25.38~d (of 609.12~h Sun Fact Sheet...), synodic: 27.2753~d (derived)\\

rotation axis tilt (see next section)\\

%surface differential rotation
The Sun's inner thermal convective circulation results in a differential rotation caused by transport of angular momentum away from the rotation axis.

%best-fitting function
The Sun's sidereal differential angular velocity best-fitting function with values as stated in (Sun~Fact~Sheet...)\footnote{NASA's \textit{Sun~Fact~Sheet} (\url{http://nssdc.gsfc.nasa.gov/planetary/factsheet/sunfact.html}, accessed 2016-08-19).} is
\begin{align}
	\omega_\odot(\theta) = \omega_\text{eq} + B \sin^2(\theta) + C \sin^4(\theta)	\label{eq:omega_differential}
\end{align}
with the heliolatitude $\theta$, the equatorial angular velocity $\omega_\text{eq} = \SI{14.37}{\degree\per\day}$, the coefficients $B = \SI{-2.33}{\degree\per\day}$, and $C = \SI{-1.56}{\degree\per\day}$ (see \autoref{fig:differential_rotation_pdfcairo_plot}).

see \autoref{fig:differential_rotation_pdfcairo_plot}
\begin{figure}[htb]
	%\centering
	\fcapside[\FBwidth]{
		\includegraphics[width=0.5\textwidth]{figures_of_mine/gnuplots/differential_rotation_pdfcairo_plot.pdf}
	}{
		\caption{Diagram of the sidereal solar surface differential rotation. It shows the angular velocity for different latitudes. remove sides...}
		\label{fig:differential_rotation_pdfcairo_plot}
	}
\end{figure}

\noindent Thus, the solar equatorial rotation period (sidereal) is
\begin{align}
	T_\odot^\text{eq} &= 360\degree / A\\
	&= 25.05~\text{d}	\nonumber\\
\shortintertext{and the synodic period is}
	T_\odot^\text{eq,syn} &= 1/(1/T_\odot^\text{eq} - 1/T_\text{Earth})\\
	&= 26.90~\text{d}	\nonumber
\end{align}
with the Earth's orbital rotation period $T_\text{Earth} = 365.25$~d (1/100 Julian century).\\

Solar surface rotation period at equator\\
sidereal: 25.05~d (Sun Fact Sheet...), synodic: 26.90~d (derived)\\
Solar surface rotation period at poles:\\
sidereal: 34.35~d (diff. rot. formula), synodic: 37.92~d (derived)\\
are listed in \autoref{tab:solar_surface_rotation_periods}.\\
\begin{table}[htb]\small
	\caption{Solar surface rotation periods for the equator, \SI{+-16}{\degree}~latitudes and the poles in sidereal and synodic rotation.}
	\label{tab:solar_surface_rotation_periods}
	\centering
	\begin{tabular}{llll}
		\hline\hline
				&Equator	&\SI{+-16}{\degree}~latitudes	&Poles\\
				&\multicolumn{1}{c}{[d]}	&\multicolumn{1}{c}{[d]}	&\multicolumn{1}{c}{[d]}\\
		\hline
		Sidereal	&25.05	&25.38	&34.35\\
		Synodic	&26.90	&27.2753$^\text{a}$	&37.92\\
		\hline
		\multicolumn{4}{l}{\footnotesize{$^\text{a}$Carrington solar rotation period}}
	\end{tabular}
\end{table}


%meridional flow

The meridional circulation is the proposed equatorial updrift and polar downdrift - a result of Reynolds stress and convective transport (cite?).\\


\section{Electric field at the magnetopause}
\label{sec:electric_field_at_the_magnetopause}
The Lorentz force law defines the electric and magnetic field vectors $\vect{E}$ and $\vect{B}$:
\begin{align}
	\vect{F} = q\,(\vect{E} + \vect{v} \times \vect{B})	\,.
\end{align}
It describes the force $\vect{F}$ that acts on a charge $q$ with velocity $v$. The full Ohm's law accounts for the Lorentz force:
\begin{align}
	\vect{j} &= \sigma(\vect{E} + \vect{v} \times \vect{B})	& \Longleftrightarrow	&	&	\vect{E} &= - \vect{v} \times \vect{B} + \frac{\vect{j}}{\sigma}	\,.
\end{align}
Solar wind is often approximated as an ideal MHD plasma having an infinite conductivity ($\sigma = \infty$). Thus electric fields do not generate electric currents $\vect{j}$:
\begin{align}
	\vect{E} = - \vect{v} \times \vect{B}	\,.
\end{align}
Assuming the solar wind flows into the x-direction ($v_\text{y} = v_\text{z} = 0$), the resulting electric field components become
\begin{align}
	\vect{E} =& \begin{bmatrix}
		E_\text{x}\\
		E_\text{y}\\
		E_\text{z}\\
	\end{bmatrix} = \begin{bmatrix}
		0\\
		- v_\text{x} \cdot B_\text{z}\\
		v_\text{x} \cdot B_\text{y}
	\end{bmatrix}	\,,\\
		=& \begin{bmatrix}
		0\\
		- v_\text{x} \cdot |\vect{B_\text{yz}}| \cdot \cos(\theta_c)\\
		v_\text{x} \cdot |\vect{B_\text{yz}}| \cdot \sin(\theta_c)\\
	\end{bmatrix}	\,.
\end{align}
If in addition the magnetic field is strictly orientated along the z-axis, the only non-zero electric field component remaining is $E_\text{y}$:
\begin{align}
	\vect{E} = - v_\text{x} \cdot B_\text{z}	\,.
\end{align}
Reconnection occurs only where the magnetic fields of the solar wind and magnetopause are antiparallel. Due to the magnetosphere's dipole topology, this is the case at the equator of the sunward magnetopause (when using GSM coordinates).

However, reconnection can occur on the whole magnetopause surface facing the solar wind, that is, even at higher latitudes where $B_\text{z}$ is not the only field component. Thus, it makes sense to use the magnetic field's absolute value $|B|$ and its clock angle $\theta_c = \tan^{-1}\left(\frac{B_\text{y}}{B_\text{z}}\right)$ instead of just the field's $B_\text{z}$ component.\\


Lockwood2013:\\
on timescales of T = 1 yr, the IMF orientation factor is averaged out and the average southward IMF component and, hence, the level of geomagnetic activity, is proportional to B , as first noted by Stamper et al. (1999). This is the basic reason that we are able to make deductions about B from geomagnetic activity when averaging is done on annual timescales.\\

the distribution of annual values of the ratio B/ [ B sin 4 ( theta / 2)]. The mean value of this distribution is 3.251 and the standard deviation is 0.369.\\


% Electromagnetism is one of the four fundamental forces at the common level of energy. In all situations examined in this thesis (solar wind plasma, magnetosphere) it is by far the strongest force and the others can be neglected.
% The Maxwell equations in differential notation:
% \begin{align}
% 	\text{div}\,\vect{B} &= 0\\
% 	\text{div}\,\vect{D} &= \rho\\
% 	\text{rot}\,\vect{H} &= \vect{j} + \dot{\vect{D}}\\
% 	\text{rot}\,\vect{E} &= - \dot{\vect{B}}
% \end{align}
% With the magnetic flux density $\vect{B}$ (aka magnetic field), the electric displacement field $\vect{D}$, the charge density $\rho$, the magnetic field $\vect{H}$, the electric field $\vect{E}$ and the current density $\vect{j}$.\\
% %note about B naming convention within this work as magnetic field strength


\section{Plasma beta}
\label{sec:plasma_beta}
The thermal pressure of a plasma is defined as $p = n k_\text{B} T$, with the number density $n$ and the Boltzmann constant $k_\text{B}$. However, according to magnetohydrodynamics (MHD), the magnetic energy density $w_\text{mag} = \frac{B^2}{2 \mu_0}$, with the magnetic field strength $B$ and the permeability constant $\mu_0$, behaves like an additional pressure that adds to the thermal pressure of a plasma \citep[p.~50]{Kivelson1995}. The ratio of the thermal pressure to the magnetic pressure determines the behavior of the plasma. If the thermal pressure dominates the magnetic pressure (warm plasma), the plasma movements transport the magnetic field, else the plasma movements are guided by the magnetic field lines (cold plasma). This ratio is called plasma beta:
\begin{align}
	\beta &= \frac{p}{p_\text{mag}}	\,,\\
	&= \frac{2 \mu_0 n k_\text{B} T}{B^2}	\,.
\end{align}
The plasma at the photosphere has typical beta values around~14 and that of the low corona around~0.2 \citep{Gary2001}. Further up, beta raises again and the region where it equals~1 is defined as the source surface for the solar wind \citep{Schatten1969}.	% source surface where beta equals 1 (Schatten1969), at 0.6~Rs\\
This surface is typically located at about \SIrange{1.2}{2.5}{\Rs} \citep{Gary2001}, more cites...\\
% low source surface at 1.2~Rs (Gary2001)\\
% typical value at about 2.5~Rs\\
% source surface in the distance range of \SIrange{1.2}{4}{\Rs}\\

The solar wind usually has plasma beta values higher than~1 -- it carries the solar magnetic field away into the heliosphere. Together with the solar rotation, this effect creates the spiral form of the interplanetary magnetic field (Parker spiral). Yet in some solar wind structures, such as magnetic clouds, $\beta \ll 1$ and thus the magnetic field can still contain the plasma.\\

Savani2017:\\
``\textit{By using statistical analysis, Riley and Richardson [2013] suggest that a distinct magnetic flux rope object may be contained within an overall propagating CME structure and that the plasma beta is a good predictor variable. Savani et al. [2013b] used simulations to confirm that such a scenario of a distinct coherent core obstacle can occur within the CME structure and that a transition in plasma beta can be observed}.''\\


% https://omniweb.gsfc.nasa.gov/ftpbrowser/bow_derivation.html
% Plasma beta = thermal energy density (= thermal pressure) /magnetic energy density
%     = D*Np*k*Tp*8pi/B**2
% Beta = [(4.16*10**-5 * Tp) + 5.34] * Np/B**2 (B in nT)


%dynamic pressure $p_\text{dyn} = \rho v^2$\\
%ram pressure??\\
%VB1998p104


\section{Alfvén velocity}
\label{sec:alfvén_velocity}
The incompressible wave mode within MHD plasmas, the shear Alfvén wave, consists of periodic disturbances in the magnetic field orthogonal to its direction \citep{Alfven1942}. Alfvén waves are prevalent in open coronal regions and therefore occur in fast solar wind \citep{Cranmer2005}. Their propagation velocity is an important parameter to characterize a plasma. In an ideal incompressible MHD plasma (viscosity $\mu = 0$ and electrical conductivity $\sigma = \infty$) the kinetic and magnetic energy density are of equal value \citep[p.~51]{Kivelson1995}: 
\begin{align}
	w_\text{kin} &= w_\text{mag}\\
	\frac{\rho v^2}{2} &= \frac{B^2}{2 \mu_0}	\nonumber
\end{align}
with the permeability constant $\mu_0$ and the total mass density $\rho$ of the charged plasma particles. Thus, the Alfvén velocity can be calculated from
\begin{align}
	v_\text{A} = \frac{|B|}{\sqrt{\mu_0 \rho}}	\text{\,.}
\end{align}
The wave's phase velocity is
\begin{align}
	v_\text{ph} = v_\text{A} \cos(\theta)
\end{align}
with $\theta$ as the angle between wave propagation direction and magnetic field line, that is, Alfvén waves travel along magnetic field lines. They consist of periodic disturbances in the magnetic field, the electric field, the plasma velocity, and the current density. Plasma density, pressure and magnetic field magnitude are not affected by them. Additionally, there exist two types of compressional wave modes within MHD plasmas, the fast-mode wave and the slow-mode wave. The phase speeds of the three MHD waves meet $v_\text{fast} \geq v_\text{A} \geq v_\text{slow}$ \citep[p.~52]{Kivelson1995}. Within solar wind at \SI{1}{\au}, the typical frequency of Alfvén waves is 1--4 per hour and their average velocity is $v_\text{A} = \SI{56.8}{\km\per\s}$ \citep{Veselovsky2010}.\\	% at 1~au in 1963--2007\\

Alfvén critical surface...\\
sonic critical surfaces...\\

shocks \& MHD waves (Alfvén waves)\\
Alfvén velocity ca. 40~km/s (Kivelson1995)\\
Alfvén velocity 56.8~km/s (Veselovsky2009)\\

sonic and Alfvénic critical point positions (see Sittler \& Guhathakurta (1999))\\
sonic point and slow solar wind origin (Sheeley et al. 1997)\\


%wolfram alpha:
%solve v = 0.001*B/sqrt(mu*rho), B=5.6*10^-9, rho=5.3*10^6*1.66*10^-27, mu=4*pi*10^-7

%v_A=53.3km/s for r=1au; B=5.6nT; N=5.3cm-3
%v_A=91.5km/s for r=0.3 au; B=37.9nT; N=82.2cm-3
%v_A=280.6km/s for r=0.04 au; B=930.6nT; N=5274cm-3

%slow wind: v_A=28.5km/s for r=1au; B=4.7nT; N=13.0cm-3
%fast wind: v_A=68.7km/s for r=1au; B=5.7nT; N=3.3cm-3

%$v_\text{A} = 53$~km/s for $B = 5.6$~nT and $\rho = 5.3$~cm$^{-3}$


\section{Sun-Earth orbit geometry}
\label{sec:sun_earth_orbit_geometry}

%all from https://en.wikipedia.org/wiki/Earth's_orbit
%see also http://nssdc.gsfc.nasa.gov/planetary/factsheet/earthfact.html
orbit defines ecliptic\\

Earth orbit parameters (cite?):\\
semimajor axis: $a = 1.000001018$~au\\
eccentricity: $e = 0.0167086$~au\\
%perihelion, point on solar orbit with minimum distance to Sun\\
%aphelion, point on solar orbit with maximum distance to Sun\\
distance at perihelion: (formula cite?, accuracy?)\\
\begin{align}
	r_\text{p} &= a (1 - e)\\
		&= 0.98329~\text{au}	\nonumber
\end{align}
distance at aphelion:\\
\begin{align}
	r_\text{p} &= a (1 + e)\\
		&= 1.0167~\text{au}	\nonumber
\end{align}

for calculation of heliospheric distance see HORIZONS Web-Interface at \url{http://ssd.jpl.nasa.gov/horizons.cgi}

perihelion/aphelion times...
%http://aa.usno.navy.mil/data/docs/EarthSeasons.php

\subsection*{Solar distance}	%remove subsections?

Sun-Earth distance over the course of the year.\\
In the year 2017 Earth's perihelion was on 5~January with a distance of \SI{-1.67}{\percent} from \SI{1}{\au} (Horizons On-Line Ephemeris System\footnote{\url{http://ssd.jpl.nasa.gov/horizons.cgi}}, Solar System Dynamics Group, Jet~Propulsion Laboratory).\\
The cosine approximation
\begin{align}
	r_\text{E}(t) = 1 - 0.0167 \cdot \cos\left(2 \pi \left(t - 2017 - \frac{5}{365}\right)\right)\,,
\end{align}
with $t$ in years, suffices for our accuracy requirements.\\
% HORIZONS Web-Interface
% http://ssd.jpl.nasa.gov/horizons.cgi
% Ephemeris Type:	VECTORS
% Target Body:		Earth [Geocenter] [399]
% Coordinate Origin:	Sun (body center) [500@10]
% Time Span:		Start=2017-01-01, Stop=2017-12-31, Step=1 d
% Table Settings:	CSV format=YES
% Display/Output:	download/save (plain text file)
%-> distance:	9.833098226363186E-01
%-> diff. to 1 au:	0.166901774
%-> change to day before:	4.4e-6	-> 0.1669018(44)

seasonal variation function:\\
$X_\text{avg}(t) = a\,r_\text{E}(t)^b$\\
% B_avg = 5.969(29)~nT\\
% B_med = 5.435(24)~nT\\
% N_avg = 6.329(85)~cm-3\\
% N_med = 4.858(60)~cm-3\\
% T_avg = 1.198(16)e5~K\\
% T_med = 7.17(12)e4~K\\
% V1_avg = 4.132(22)e2~km/s\\
% V1_med = 4.034(21)e2~km/s\\
% V2_avg = 4.132(22)e2~km/s\\
% V2_med = 4.034(21)e2~km/s\\

\subsection*{Solar rotation axis tilt to the ecliptic}

The inclination of the solar equator to the ecliptic (tilt/obliquity) is $i_\odot = 7.25$\textdegree{} \citep{USNO2015}.\\	%USNO2015 C3

the rotation axis is tilted from the ecliptic normal\\


Viewed from Earth the projected solar rotation axis tilt angle varies as the Earth is moving on its orbit.\\

At the time XX the angle is zero.\\

The projected tilt angle to Earth over the year is\\

% Stonyhurst disk
% 
% Calculation of solar tilt angle at given time
% solar tilt/obliquity to ecliptic: i_sun = 7.25\degrees (Sun fact sheet: \url{http://nssdc.gsfc.nasa.gov/planetary/factsheet/sunfact.html)}
% modulate this angle with a sine over the year:
% projected tilt from Earth: i_proj = i_sun * sin(beta)
% this years vernal equinox: eq = 2015.0 + 2.0/12.0 + 20.0/365.0 = 2015.2215
% actual separation angle from vernal equinox position: phi = (today - eq) * 360
% 
% ecliptic longitude: \url{http://cohoweb.gsfc.nasa.gov/helios/plan_des.html}
% Ecliptic longitude of ascending node of the Sun's equator
% http://sspg1.bnsc.rl.ac.uk/SEG/Coordinates/angles.htm

%Hapgood (1992), Space Physics Coordinate Transformations: A User Guide
\citet{Hapgood1992}:
\begin{align}
	\omega = 73.67 + 0.013\,958 * (today - 1850.0)	%#why? yearly change
\end{align}
% beta = phi - omega
% 
% Meridians (longitude lines)
% 
% Parallels (latitude lines)


% vernal_equinox = 2016.0 + 2.0/12.0 + 20.0/366.0
% tilt(x) = 7.25 * sin(-(73.67 + 0.013958 * (x - 1850.0)) + (-vernal_equinox + x) * 360.0)
% 
% 	tilt(2016.0+(x-1.0)/12.0) with lines title "sine approx." ls 2 lw 2,\

solar tilt over the year, see \autoref{fig:Solar_tilt_seasons_plot}
\begin{figure}[htb]
	\fcapside[\FBwidth]{
		\includegraphics[width=0.5\textwidth]{figures_of_mine/gnuplots/Solar_tilt_seasons_plot.pdf}
	}{
		\caption{Projected solar tilt angle over the year as viewed from Earth. remove sides...}
		\label{fig:Solar_tilt_seasons_plot}
	}
\end{figure}

\subsection*{Earth rotation axis tilt to the ecliptic}


\section{GSE, GSM, and HGI coordinate systems}
\label{sec:coordinate_systems}

A wise choice of coordinate system often eases the understanding and the calculation of a given problem, especially in geo- and astrophysics there exist an abundance of different coordinate systems. The coordinate systems used in this thesis are Earth- and Sun-centered and are described in the following. Three different Cartesian systems are used: GSE, GSM, and HGI coordinates.

% GSE - Geocentric Solar Ecliptic
\subsection*{Geocentric Solar Ecliptic coordinates}
The Geocentric Solar Ecliptic (GSE) coordinates are centered at Earth and oriented with its orbit around the Sun. The system's x"~axis points in direction of the Sun and its z"~axis in direction of the ecliptic north pole \citep{Russell1971,Hapgood1992}. The y"~axis completes the right-handed orthogonal system.
Thus compared to a fixed coordinate system, the GSE system completes one rotation around its z"~axis per year.
% Russell1971
% http://jsoc.stanford.edu/doc/keywords/Chris_Russel/Geophysical%20Coordinate%20Transformations.htm
% SPENVIS coordinate systems
% https://www.spenvis.oma.be/help/background/background.html

GSE coordinates are preferably used when looking at near-Earth solar wind measurements. In this work GSE coordinates are applied for the near-Earth solar wind measurements, e.g., with ACE data in Figures~\ref{fig:ACE_64s_v7_thesis_CIRs_2013-5-1_65_plot} and \ref{fig:ACE_64s_v7_thesis_CME_2013-6-26_6}.

% GSM - Geocentric Solar Magnetospheric
\subsection*{Geocentric Solar Magnetospheric coordinates}
The Geocentric Solar Magnetospheric (GSM) coordinates are also centered at Earth but oriented with its magnetic dipole axis. The x"~axis points in direction of the Sun and the z"~axis is parallel to the projection of the magnetic dipole axis on the plane normal to the x"~axis \citep{Russell1971,Hapgood1992}. Its direction is towards the negative magnetic pole located near the geographic North Pole. The dipole axis is defined by the centered geomagnetic dipole which is the best approximation to the International Geomagnetic Reference Field \citep{Thebault2015}. Again, the y"~axis completes the right-handed orthogonal system.
Relative to the GSE system, the GSM system shows a wobbling rotation around its x"~axis. The amplitude varies in time and is dependent on the tilt of the Earth's rotation axis to the ecliptic normal (\SI{23.44}{\degree}) and the tilt of the dipole axis, which itself slightly changes over the years. Most of the last century, the tilt had an angle near \SI{11.5}{\degree} from the geographic poles, yet the current (2015) angular distance is \SI{9.53}{\degree} \citep{Thebault2015}.

GSM coordinates are useful in problems regarding the magnetosphere. These coordinates are applied throughout \autoref{chap:chapter2}.

% HGI - Heliographic Inertial
\subsection*{Heliographic Inertial coordinates}
The Heliographic Inertial (HGI) coordinates are centered at the Sun and oriented with its rotation axis. The z"~axis points northward parallel to the Sun's rotation axis. The x"~axis is parallel to the intersection line between the solar equatorial plane and the ecliptic plane \citep{Burlaga1984a}, which are inclined by \SI{7.25}{\degree}. This x"~axis points to the longitude of the ascending node which was at \SI{74.367}{\degree} on 1~January 1900 at 12:00~UT, but increases slowly with time (about \SI{1.4}{\degree} per century) due to the Earth's precession\footnote{Numbers according to NASA's COHOWeb documentation: \urlfoot{https://omniweb.gsfc.nasa.gov/coho/html/cw_data.html}}. The y"~axis completes the right-handed orthogonal system.
Over one year, the position of Earth oscillates within the HGI latitude range of \SI{+-7.25}{\degree}.

HGI coordinates are useful when looking at processes concerning the solar wind in the heliosphere. These coordinates are applied for the processing of the Helios data, e.g., in Figures~\ref{fig:Helios_r_b_ssn}, \ref{fig:Helios12_orbits_ecliptic_polar}, and \ref{fig:helios_data_frequency}.


\section{True skill statistic}
\label{sec:true_skill_statistic}
The true skill statistic (TSS) or true skill score is a means for measuring the predictive value of a variable. It was developed by \citet{Hanssen1965} for evaluating different predictors for rain and dry weather conditions. Therefore it is also often called the Hanssen-Kuipers skill score or the Hanssen-Kuipers discriminant. As with other skill scores, the TSS is based on a 2x2 contingency table that categorizes forecasted and observed events, see the \autoref{tab:contingency_table}.
\begin{table}[htb]
	\caption{Generic contingency table for categorizing forecasted and observed events. The designations for the event counts are given in parentheses.}
	\label{tab:contingency_table}
	\centering
	\begin{tabular}{ll|cc}
		\hline\hline
				&\multicolumn{3}{c}{\hspace*{1em}Observed}\\
				&	&Yes	&No\\
		\cline{2-4}
		\multirow{4}{*}{Forecasted}	&\multirow{2}{*}{Yes}	&Hit	&False alarm\\
				&	&($n_\text{H}$)	&($n_\text{FA}$)\\
				&\multirow{2}{*}{No}	&Miss	&Correct null\\
				&	&($n_\text{M}$)	&($n_\text{CN}$)\\
		\hline
	\end{tabular}
\end{table}
The following ratios, derived from this kind of table, reveal relevant information about the forecast quality \citep{Doswell1990}: The proportion of correct predictions
\begin{align}
	\mathit{PC} = \frac{n_\text{H} + n_\text{CN}}{n_\text{H} + n_\text{FA} + n_\text{M} + n_\text{CN}}	\,,
\end{align}
the probabilities of detection and of false detection (i.e., the hit rate and the false alarm rate)
\begin{align}
	\mathit{POD} = \frac{n_\text{H}}{n_\text{H} + n_\text{M}}	&	&\text{and}	&	&\mathit{POFD} = \frac{n_\text{FA}}{n_\text{FA} + n_\text{CN}}	\,,
\end{align}
as well as the false alarm and detection failure ratios
\begin{align}
	\mathit{FAR} = \frac{n_\text{FA}}{n_\text{FA} + n_\text{H}}	&	&\text{and}	&	&\mathit{DFR} = \frac{n_\text{M}}{n_\text{M} + n_\text{CN}}	\,.
\end{align}
The TSS is the difference between the hit rate and the false alarm rate\footnote{False alarm rate -- not to be confused with false alarm ratio.} and thus it is calculated as follows \citep[Eq.~15]{Hanssen1965}:
\begin{align}
	\mathit{TSS} &= \mathit{POD} - \mathit{POFD}\\
		&= \frac{n_\text{H} \cdot n_\text{CN} - n_\text{FA} \cdot n_\text{M}}{\left(n_\text{H}+n_\text{M}\right) (n_\text{FA}+n_\text{CN})}	\,.	\label{eq:true_skill_score}
\end{align}
Its value is in the range from -1 to 1, with 1 representing an ideal prediction, 0 a random prediction, and -1 an ideal negative prediction. The TSS is positive if the hit rate is higher than the false alarm rate.

In the case of rare event forecasting, the contingency table is dominated by correct nulls and the TSS approaches the hit rate. \citet{Doswell1990} show that the Heidke skill score is superior in these situations, in that it considers correct null forecasts in a controlled way. They recommend to use the Heidke skill score in preference to TSS for rare event forecasting. Other often applied skill scores are the threat score, also called critical success index or Gilbert score, and the equitable threat score. However, the TSS is commonly applied to evaluate and compare space weather prediction methods, especially in forecasting the \Kp~index \citep{Detman1999,Wing2005,Savani2017}.

% Fraction Correct is given by [0,1]\\
% chance hits
% ch=((x+y)*(x+z))/(w+x+y+z)


\section{Lognormal distribution}
\label{sec:lognormal_distribution}

This is a small summary about the lognormal probability distribution \citep[p.~780]{Bronstein2000}. The lognormal distribution is the distribution of a random variable $X$ if the logarithm of $X$ conforms to a normal distribution. Its shape is highly asymmetric, however in a semi-log plot the Gaussian bell curve is recognizable (see the second panel of \autoref{fig:lognormal_3panel_pdfcairo_plot}).
\begin{figure}[htb]
	\begin{floatrow}
		\ffigbox{
			\includegraphics[width=0.5\Xhsize]{figures_of_mine/gnuplots/lognormal_3panel_pdfcairo_plot.pdf}
		}{
			\caption{The lognormal probability density function ($\sigma = 1, \mu = 0$) plotted in a linear, semi-log and log-log way. remove borders...}
			\label{fig:lognormal_3panel_pdfcairo_plot}
		}
		\ffigbox{
			\includegraphics[width=\Xhsize]{figures_of_mine/gnuplots/lognormal_ms_pdfcairo_plot.pdf}
		}{
			\caption{Five lognormal distributions plotted with fixed $\sigma$ (top) and fixed $\mu$ (bottom). remove borders...}
			\label{fig:lognormal_ms_pdfcairo_plot}
		}
	\end{floatrow}
\end{figure}
Its probability density function is
\begin{align}
	f(x) &= \frac{1}{\sigma \sqrt{2 \pi} x} \, \text{e}^{- \frac{(\ln x - \mu)^2}{2 \sigma^2}}
\end{align}
with the location ($\mu$) and the shape parameter ($\sigma$). Changes in $\mu$ affect both the horizontal and vertical scaling of the function, whereas $\sigma$ has an influence on its shape (see \autoref{fig:lognormal_ms_pdfcairo_plot}).\\

Because it is a probability distribution, its area is normalized
\begin{align}
	\int_0^\infty f(x) \text{d} x = 1\,.
\end{align}

For a lognormally distributed random variable the geometric moments mean, standard deviation and variance are:
\begin{align*}
	&\mu_\text{g} = \text{e}^\mu ,\\
	&\sigma_\text{g} = \text{e}^\sigma ,\\
	&var_\text{g} = \text{e}^{\sigma^2}~~(!) .
\end{align*}

Its arithmetic moments are:
\begin{align*}
	&\mu_\text{a} = \text{e}^{\mu + \frac{\sigma^2}{2}} ,\\
	&\sigma_\text{a} = \text{e}^{\mu + \frac{\sigma^2}{2}} \, \left(\text{e}^{\sigma^2} - 1\right) ,\\
	&var_\text{a} = \sigma_\text{a}^2 .
\end{align*}

Other useful characteristics are the median and the mode
\begin{align*}
	&x_\text{median} = \text{e}^{\mu},\\
	&x_\text{mode} = \text{e}^{\mu - \sigma^2}\,.
\end{align*}
Note that for the lognormal distribution its median is equal to its geometric mean.\\

Applications of lognormal distributions...\\
%Limpert2001: Lognormal Distributions across the Sciences: Keys and Clues
%http://bioscience.oxfordjournals.org/content/51/5/341.full

Most natural quantities which can only be positive are lognormally distributed. e.g. animal body sizes?, animal life expectancies, financial stock prices...; income distributions.\\

%life expectancy analyses of economic, technical and biological processes. Bronstein2000



\section{Astronomical constants}
\label{sec:astronomical_constants}

Astronomical unit: 1~au = 149\,597\,870\,700~m \citep{USNO2015}\\ %Astronomical constants
Solar mass: $M_\odot = 1.9884(2)\times10^{30}$~kg \citep{USNO2015}\\ %see also Mamajek2015
Nominal solar radius (photosphere): $R_\odot = \SI{695700}{\km}$ \citep{Mamajek2015}\\ %Astronomical constants
Solar rotation axis tilt: $i_\odot = 7.25$\textdegree{} \citep{USNO2015}\\ %C3; Obliquity to ecliptic
Solar surface rotation period at equator, sidereal: 25.05~d (Sun Fact Sheet...)\\
Nominal solar effective temperature (photosphere): $T_{\text{eff}\odot} = \SI{5772}{\kelvin}$ \citep{Mamajek2015}\\
Sun escape velocity: $v_\text{esc} = 617.6$~km/s (Sun Fact Sheet...)\\

%Astronomical constants
%http://asa.usno.navy.mil/static/files/2016/Astronomical_Constants_2016.pdf

NASA maintains the Planetary~Fact~Sheets\footnote{NASA Planetary Fact Sheets website: \urlfoot{https://nssdc.gsfc.nasa.gov/planetary/planetfact.html}} online.\\



% \section{Abbreviations}
% \label{sec:abbreviations}
% 
% Projects:
% \begin{description*}
% 	\item[AFFECTS] Advanced Forecast For Ensuring Communications Through Space
% 	\item[HELCATS] Heliographic Cataloging, Analysis and Techniques Service
% 	\item[FP7] Framework Programme 7
% 	\item[CGAUSS] Coronagraphic German And US SolarProbePlus Survey
% 	\item[OPTIMAP] OPerational Tool for Ionospheric Mapping And Prediction
% \end{description*}
% 
% Spacecraft:\\
% SPP -- Solar Probe Plus\\
% WISPR -- Wide-field Imager for Solar Probe\\
% ACE -- Advanced Composition Explorer\\
% 	MAG -- Magnetometer\\
% 	SWEPAM -- Solar Wind Electron Proton Alpha Monitor\\
% 	RTSW -- Real Time Solar Wind\\
% SDO -- Solar Dynamics Observatory\\
% SOHO -- Solar and Heliospheric Observatory\\
% STEREO -- Solar TErrestrial RElations Observatory\\
% 
% Organizations:\\
% NASA -- National Aeronautics and Space Administration\\
% 	SPDF -- Space Physics Data Facility\\
% NOAA -- National Oceanic and Atmospheric Administration\\
% 	SWPC -- Space Weather Prediction Center\\
% UGOE -- University of Göttingen\\
% IAG -- Institute for Astrophysics Göttingen\\
% GFZ -- GeoForschungsZentrum\\
% WDC-SILSO -- World Data Center-Sunspot Index and Long-term Solar Observations\\
% 
% Sun:\\
% DB -- disparition brusques (disappearing filaments?; quiescent filaments?)\\
% SSN -- sunspot number\\
% 
% Solar wind:\\
% IMF -- interplanetary magnetic field\\
% CME -- coronal mass ejection\\
% ICME -- interplanetary coronal mass ejection\\
% MC -- magnetic cloud\\
% HSS -- high speed stream\\
% CIR -- corotating interaction region\\
% SIR -- stream interaction region\\
% SB -- sector boundary\\
% BDE -- bidirectional electrons\\
% HCS -- heliospheric current sheet\\
% HPS -- heliospheric plasma sheet\\
% 
% Earth:\\
% Kp -- planetare Kennziffer\\
% Dst -- Disturbance storm time\\
% 
% Coordinate systems:\\
% GSE -- geocentric solar ecliptic\\
% GSM -- geocentric solar magnetospheric\\
% 
% Theories and techniques:\\
% MVA -- minimum variance analysis\\
% MHD -- magnetohydrodynamic\\
% GCS -- Graduated Cylindrical Shell\\
% CAT -- CME Analysis Tool\\

