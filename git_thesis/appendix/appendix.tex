
\chapter{Appendix}
\label{chap:appendix}

%in the end: sort sections in order of reference in work

\section{Electromagnetism}	%maybe exclude this section?
Electromagnetism is one of the four fundamental forces at the common level of energy. In all situations examined in this thesis (solar wind plasma, magnetosphere) it is by far the strongest force and the others can be neglected.\\

%Lorentz force law as the definition of E and B\\
%The electromagnetic force F on a test charge at a given point and time is a certain function of its charge q and velocity v, which can be parameterized by exactly two vectors E and B, in the functional form: (wiki)\\
The Lorentz force law defines the electric and magnetic field vectors $\vect{E}$ and $\vect{B}$:
\begin{align}
	\vect{F} = q\,(\vect{E} + \vect{v} \times \vect{B})	\,,
\end{align}
with the charge $q$ the force is acting on and its velocity $v$.\\

The Maxwell equations in differential notation:
\begin{align}
	\text{div}\,\vect{B} &= 0\\
	\text{div}\,\vect{D} &= \rho\\
	\text{rot}\,\vect{H} &= \vect{j} + \dot{\vect{D}}\\
	\text{rot}\,\vect{E} &= - \dot{\vect{B}}
\end{align}
With the magnetic flux density $\vect{B}$ (aka magnetic field), the electric displacement field $\vect{D}$, the charge density $\rho$, the magnetic field $\vect{H}$, the electric field $\vect{E}$ and the current density $\vect{j}$.\\
%note about B naming convention within this work as magnetic field strength% maybe put into beginning...


\section{Plasma beta}
\label{sec:plasma_beta}
The thermal pressure of a plasma is defined as $p = n k_\text{B} T$, with the number density $n$ and the Boltzmann constant $k_\text{B}$. However, according to magnetohydrodynamics (MHD), the magnetic energy density $w_\text{mag} = \frac{B^2}{2 \mu_0}$, with the magnetic field strength $B$ and the permeability constant $\mu_0$, behaves like an additional pressure that adds to the thermal pressure of a plasma \citep[p.~50]{Kivelson1995}. The ratio of the thermal pressure to the magnetic pressure determines the behavior of the plasma. If the thermal pressure dominates the magnetic pressure (warm plasma), the plasma movements transport the magnetic field else the plasma movements are guided by the magnetic field lines (cold plasma). This ratio is called plasma beta:
\begin{align}
	\beta &= \frac{p}{p_\text{mag}}	\,,\\
	&= \frac{2 \mu_0 n k_\text{B} T}{B^2}	\,.
\end{align}
The plasma at the photosphere has typical beta values around~14 and that of the low corona around~0.2 \citep{Gary2001}. Further up, beta raises again and the region where it equals~1 is defined as the source surface for the solar wind \citep{Schatten1969}.	% source surface where beta equals 1 (Schatten1969), at 0.6~Rs\\
This surface is typically located at about \SIrange{1.2}{2.5}{\Rs} \citep{Gary2001}, more cites...\\
% low source surface at 1.2~Rs (Gary2001)\\
% typical value at about 2.5~Rs\\
% source surface in the distance range of \SIrange{1.2}{4}{\Rs}\\

The solar wind usually has plasma beta values higher than~1 -- it carries the solar magnetic field away into the heliosphere. Together with the solar rotation, this effect creates the spiral form of the interplanetary magnetic field (Parker spiral). Yet in some solar wind structures, such as magnetic clouds, $\beta \ll 1$ and thus the magnetic field can still contain the plasma.\\


% https://omniweb.gsfc.nasa.gov/ftpbrowser/bow_derivation.html
% Plasma beta = thermal energy density (= thermal pressure) /magnetic energy density
%     = D*Np*k*Tp*8pi/B**2
% Beta = [(4.16*10**-5 * Tp) + 5.34] * Np/B**2 (B in nT)


%dynamic pressure $p_\text{dyn} = \rho v^2$\\
%ram pressure??\\
%VB1998p104


\section{Alfvén velocity}
\label{sec:alfvén_velocity}
The incompressible wave mode within MHD plasmas, the shear Alfvén wave, consists of periodic disturbances in the magnetic field orthogonal to its direction \citep{Alfven1942}. Alfvén waves are prevalent in open coronal regions and therefore occur in fast solar wind \citep{Cranmer2005}. Their propagation velocity is an important parameter to characterize a plasma. In an ideal incompressible MHD plasma (viscosity $\mu = 0$ and electrical conductivity $\sigma = \infty$) the kinetic and magnetic energy density are of equal value \citep[p.~51]{Kivelson1995}: 
\begin{align}
	w_\text{kin} &= w_\text{mag}\\
	\frac{\rho v^2}{2} &= \frac{B^2}{2 \mu_0}	\nonumber
\end{align}
with the permeability constant $\mu_0$ and the total mass density $\rho$ of the charged plasma particles. Thus, the Alfvén velocity can be calculated from
\begin{align}
	v_\text{A} = \frac{|B|}{\sqrt{\mu_0 \rho}}	\text{\,.}
\end{align}
The wave's phase velocity is
\begin{align}
	v_\text{ph} = v_\text{A} \cos(\theta)
\end{align}
with $\theta$ as the angle between wave propagation direction and magnetic field line, that is, Alfvén waves travel along magnetic field lines. They consist of periodic disturbances in the magnetic field, the electric field, the plasma velocity, and the current density. Plasma density, pressure and magnetic field magnitude are not affected by them. Additionally, there exist two types of compressional wave modes within MHD plasmas, the fast-mode wave and the slow-mode wave. The phase speeds of the three MHD waves meet $v_\text{fast} \geq v_\text{A} \geq v_\text{slow}$ \citep[p.~52]{Kivelson1995}. Within solar wind at \SI{1}{\au}, the typical frequency of Alfvén waves is 1--4 per hour and their average velocity is $v_\text{A} = \SI{56.8}{\km\per\s}$ \citep{Veselovsky2010}.\\	% at 1~au in 1963--2007\\

Alfvén critical surface...\\
sonic critical surfaces...\\


%wolfram alpha:
%solve v = 0.001*B/sqrt(mu*rho), B=5.6*10^-9, rho=5.3*10^6*1.66*10^-27, mu=4*pi*10^-7

%v_A=53.3km/s for r=1au; B=5.6nT; N=5.3cm-3
%v_A=91.5km/s for r=0.3 au; B=37.9nT; N=82.2cm-3
%v_A=280.6km/s for r=0.04 au; B=930.6nT; N=5274cm-3

%slow wind: v_A=28.5km/s for r=1au; B=4.7nT; N=13.0cm-3
%fast wind: v_A=68.7km/s for r=1au; B=5.7nT; N=3.3cm-3

%$v_\text{A} = 53$~km/s for $B = 5.6$~nT and $\rho = 5.3$~cm$^{-3}$


\section{Solar surface differential rotation}
\label{sec:solar_surface_differential_rotation}

%solar rotation
The solar differential rotation is visible on the surface and was first discovered from sunspot observations by \citet{Scheiner1630}. (double)\\

%rotation period
\citet{Bartels1934} set the synodic solar rotation period to 27~days for the definition of his solar rotation number. The Bartels' Rotation Number counts the solar rotations starting with 8 February 1832.\\
Carrington solar rotation period of 27.2753~days (Where Carrington Rotation Number is based upon, starting with November 9, 1853; Wikipedia...)\\
%synodic rotation period of 27.2753 days (or a sidereal period of 25.38 days; This chosen period roughly corresponds to rotation at a latitude of 16~deg (sic), which is consistent with the typical latitude of sunspots and corresponding periodic solar activity; cite from https://en.wikipedia.org/wiki/Solar_rotation

Solar surface rotation period at 16\degree{} latitude:\\
sidereal: 25.38~d (of 609.12~h Sun Fact Sheet...), synodic: 27.2753~d (derived)\\

rotation axis tilt (see next section)\\

%surface differential rotation
The Sun's inner thermal convective circulation results in a differential rotation caused by transport of angular momentum away from the rotation axis.

%best-fitting function
The Sun's sidereal differential angular velocity best-fitting function with values as stated in (Sun~Fact~Sheet...)\footnote{NASA's \textit{Sun~Fact~Sheet} (\url{http://nssdc.gsfc.nasa.gov/planetary/factsheet/sunfact.html}, accessed 2016-08-19).} is
\begin{align}
	\omega_\odot(\theta) = \omega_\text{eq} + B \sin^2(\theta) + C \sin^4(\theta)	\label{eq:omega_differential}
\end{align}
with the heliolatitude $\theta$, the equatorial angular velocity $\omega_\text{eq} = \SI{14.37}{\degree\per\day}$, the coefficients $B = \SI{-2.33}{\degree\per\day}$, and $C = \SI{-1.56}{\degree\per\day}$ (see \autoref{fig:differential_rotation_pdfcairo_plot}).

see \autoref{fig:differential_rotation_pdfcairo_plot}
\begin{figure}[htb]
	%\centering
	\fcapside[\FBwidth]{
		\includegraphics[width=0.5\textwidth]{figures_of_mine/gnuplots/differential_rotation_pdfcairo_plot.pdf}
	}{
		\caption{Diagram of the sidereal solar surface differential rotation. It shows the angular velocity for different latitudes. remove sides...}
		\label{fig:differential_rotation_pdfcairo_plot}
	}
\end{figure}

\noindent Thus, the solar equatorial rotation period (sidereal) is
\begin{align}
	T_\odot^\text{eq} &= 360\degree / A\\
	&= 25.05~\text{d}	\nonumber\\
\shortintertext{and the synodic period is}
	T_\odot^\text{eq,syn} &= 1/(1/T_\odot^\text{eq} - 1/T_\text{Earth})\\
	&= 26.90~\text{d}	\nonumber
\end{align}
with the Earth's orbital rotation period $T_\text{Earth} = 365.25$~d (1/100 Julian century).\\

Solar surface rotation period at equator\\
sidereal: 25.05~d (Sun Fact Sheet...), synodic: 26.90~d (derived)\\
Solar surface rotation period at poles:\\
sidereal: 34.35~d (diff. rot. formula), synodic: 37.92~d (derived)\\
are listed in \autoref{tab:solar_surface_rotation_periods}.\\
\begin{table}[htb]\small
	\centering
	\captionsetup{belowskip=4pt}
	\caption{Solar surface rotation periods for equator, \SI{+-16}{\degree}~latitude and poles (sidereal and synodic).}
	\begin{tabular}{llll}
		\toprule
				&Equator	&\SI{+-16}{\degree}~latitude	&Poles\\
				&\multicolumn{1}{c}{[d]}	&\multicolumn{1}{c}{[d]}	&\multicolumn{1}{c}{[d]}\\
		\midrule
		Sidereal	&25.05	&25.38	&34.35\\
		Synodic	&26.90	&27.2753$^\text{a}$	&37.92\\
		\bottomrule
		\multicolumn{4}{l}{\footnotesize{$^\text{a}$Carrington solar rotation period}}
	\end{tabular}
	\label{tab:solar_surface_rotation_periods}
\end{table}


%meridional flow

The meridional circulation is the proposed equatorial updrift and polar downdrift - a result of Reynolds stress and convective transport (cite?).\\


\section{Earth orbit geometry}
\label{sec:earth_orbit_geometry}

%all from https://en.wikipedia.org/wiki/Earth's_orbit
%see also http://nssdc.gsfc.nasa.gov/planetary/factsheet/earthfact.html
orbit defines ecliptic\\

Earth orbit parameters (cite?):\\
semimajor axis: $a = 1.000001018$~au\\
eccentricity: $e = 0.0167086$~au\\
%perihelion, point on solar orbit with minimum distance to Sun\\
%aphelion, point on solar orbit with maximum distance to Sun\\
distance at perihelion: (formula cite?, accuracy?)\\
\begin{align}
	r_\text{p} &= a (1 - e)\\
		&= 0.98329~\text{au}	\nonumber
\end{align}
distance at aphelion:\\
\begin{align}
	r_\text{p} &= a (1 + e)\\
		&= 1.0167~\text{au}	\nonumber
\end{align}

for calculation of heliospheric distance see HORIZONS Web-Interface at \url{http://ssd.jpl.nasa.gov/horizons.cgi}

perihelion/aphelion times...
%http://aa.usno.navy.mil/data/docs/EarthSeasons.php

\subsection{Solar distance}

Sun-Earth distance over the course of the year.\\
In the year 2017 Earth's perihelion was on 5~January with a distance of \SI{-1.67}{\percent} from \SI{1}{\au} (Horizons On-Line Ephemeris System\footnote{\url{http://ssd.jpl.nasa.gov/horizons.cgi}}, Solar System Dynamics Group, Jet~Propulsion Laboratory).\\
The cosine approximation
\begin{align}
	r_\text{E}(t) = 1 - 0.0167 \cdot \cos\left(2 \pi \left(t - 2017 - \frac{5}{365}\right)\right)\,,
\end{align}
with $t$ in years, suffices for our accuracy requirements.\\
% HORIZONS Web-Interface
% http://ssd.jpl.nasa.gov/horizons.cgi
% Ephemeris Type:	VECTORS
% Target Body:		Earth [Geocenter] [399]
% Coordinate Origin:	Sun (body center) [500@10]
% Time Span:		Start=2017-01-01, Stop=2017-12-31, Step=1 d
% Table Settings:	CSV format=YES
% Display/Output:	download/save (plain text file)
%-> distance:	9.833098226363186E-01
%-> diff. to 1 au:	0.166901774
%-> change to day before:	4.4e-6	-> 0.1669018(44)

seasonal variation function:\\
$X_\text{avg}(t) = a\,r_\text{E}(t)^b$\\
% B_avg = 5.969(29)~nT\\
% B_med = 5.435(24)~nT\\
% N_avg = 6.329(85)~cm-3\\
% N_med = 4.858(60)~cm-3\\
% T_avg = 1.198(16)e5~K\\
% T_med = 7.17(12)e4~K\\
% V1_avg = 4.132(22)e2~km/s\\
% V1_med = 4.034(21)e2~km/s\\
% V2_avg = 4.132(22)e2~km/s\\
% V2_med = 4.034(21)e2~km/s\\

\subsection{Solar rotation axis tilt}

The inclination of the solar equator to the ecliptic (tilt/obliquity) is $i_\odot = 7.25$\textdegree{} \citep{USNO2015}.\\	%USNO2015 C3

the rotation axis is tilted from the ecliptic normal\\


Viewed from Earth the projected solar rotation axis tilt angle varies as the Earth is moving on its orbit.\\

At the time XX the angle is zero.\\

The projected tilt angle to Earth over the year is\\

% Stonyhurst disk
% 
% Calculation of solar tilt angle at given time
% solar tilt/obliquity to ecliptic: i_sun = 7.25\degrees (Sun fact sheet: \url{http://nssdc.gsfc.nasa.gov/planetary/factsheet/sunfact.html)}
% modulate this angle with a sine over the year:
% projected tilt from Earth: i_proj = i_sun * sin(beta)
% this years vernal equinox: eq = 2015.0 + 2.0/12.0 + 20.0/365.0 = 2015.2215
% actual separation angle from vernal equinox position: phi = (today - eq) * 360
% 
% ecliptic longitude: \url{http://cohoweb.gsfc.nasa.gov/helios/plan_des.html}
% Ecliptic longitude of ascending node of the Sun's equator
% http://sspg1.bnsc.rl.ac.uk/SEG/Coordinates/angles.htm

%Hapgood (1992), Space Physics Coordinate Transformations: A User Guide
\citet{Hapgood1992}:
\begin{align}
	\omega = 73.67 + 0.013\,958 * (today - 1850.0)	%#why? yearly change
\end{align}
% beta = phi - omega
% 
% Meridians (longitude lines)
% 
% Parallels (latitude lines)


% vernal_equinox = 2016.0 + 2.0/12.0 + 20.0/366.0
% tilt(x) = 7.25 * sin(-(73.67 + 0.013958 * (x - 1850.0)) + (-vernal_equinox + x) * 360.0)
% 
% 	tilt(2016.0+(x-1.0)/12.0) with lines title "sine approx." ls 2 lw 2,\

solar tilt over the year, see \autoref{fig:Solar_tilt_seasons_plot}
\begin{figure}[htb]
	%\centering
	\fcapside[\FBwidth]{
		\includegraphics[width=0.5\textwidth]{figures_of_mine/gnuplots/Solar_tilt_seasons_plot.pdf}
	}{
		\caption{Projected solar tilt angle over the year as viewed from Earth. remove sides...}
		\label{fig:Solar_tilt_seasons_plot}
	}
\end{figure}

\subsection{Earth tilt}


\section{Coordinate systems}
\label{sec:coordinate_systems}

Coordinate systems used in this thesis:\\
GSE - Geocentric Solar Ecliptic\\
GSM - Geocentric Solar Magnetospheric\\
%RTN - Radial Tangential Normal\\
%SE - Solar Ecliptic\\
%SSE - Sun-centered Solar Ecliptic\\
HGI - Heliographic Inertial\\

refer to \citet{Hapgood1992} for GSE and GSM\\

figures for GSE and GSM

% RTN - Radial Tangential Normal
%     Spacecraft centered coordinate system. 
%     R = Sun to Spacecraft unit vector 
%     T = (Omega x R) / | (Omega x R) |
%         where Omega is Sun's spin axis (in J2000 GCI)
%     N completes the right-handed triad

%R is a unit vector pointing from the Sun to the spacecraft. T is Omega x R, where Omega is a unit vector along the Sun's rotation axis. N = R x T.

% The RTN system is fixed at a spacecraft (or the planet). The R axis is directed
% radially away from the Sun, the T axis is the cross product of the solar
% rotation axis and the R axis, and the N axis is the cross product of R and T. 
% At zero Heliographic Latitude when the spacecraft is in the solar equatorial
% plane the N and solar rotation axes are parallel.
%source: http://cdaweb.gsfc.nasa.gov/misc/NotesU.html#UY_COHO1HR_MERGED_MAG_PLASMA
  
%http://omniweb.sci.gsfc.nasa.gov/coho/helios/plan_des.html
%Solar Ecliptic Coordinate System (SE)
%The SE is a heliocentric coordinate system with the Z-axis normal to and northward from the ecliptic plane. The X-axis extends toward the first point of Aries (Vernal Equinox, i.e. to the Sun from Earth in the first day of Spring). The Y-axis completes the right handed set. The Vernal Equinox direction changes slowly; commonly invoked equinox epochs are (1) B-1950, (2) Mean-of-(current) Date, and (3) J-2000. The ecliptic longitude SE_LONG increases from zero in the x-direction towards Y-direction; the latitude, SE_LAT increases to +90 deg towards north ecliptic pole and to -90 deg towards south pole.These Lat/Long are designated as ELAT and ELON in output.

%SSE - Sun-centered Solar Ecliptic coordinates

%HGI - Heliographic Inertial coordinates
%The HGI coordinates are Sun-centered and inertially fixed with respect to an X-axis directed along the intersection line of the ecliptic and solar equatorial planes, and defines zero of the longitude, HGI_LONG. The solar equator plane is inclined at 7.25 degrees from the ecliptic. This direction was towards ecliptic longitude of 74.367 deg on 1 January 1900 at 12:00 UT; because of the precession of the Earth's equator, this longitude increases by 1.4 deg/century. The Z-axis is directed perpendicular to and northward of the solar equator, and the Y-axis completes the right-handed set. The longitude, HGI_LONG increase from zero in the X-direction towards Y-direction.The latitude HG_LAT increases to +90 deg towards the north pole, and to -90 deg towards south pole.These Lat/Long are designated as HLAT & HILON in output

% The Heliographic Inertial (HGI) coordinates are Sun-centered and inertially
% fixed with respect to an X-axis directed along the intersection line of the
% ecliptic and solar equatorial  planes. The solar equator plane is inclined at
% 7.25 degrees from the ecliptic. This direction was towards ecliptic longitude of
% 74.36 degrees  on  1  January  1900  at  1200  UT; because of precession of the
% celestial equator, this longitude increases by 1.4 degrees/century. The Z axis
% is directed perpendicular and northward from the solar equator, and the Y-axis
% completes the right-handed set. This system differs from the usual heliographic
% coordinates (e.g. Carrington longitudes) which are fixed in the frame of the
% rotating Sun.
%source: http://cdaweb.gsfc.nasa.gov/misc/NotesU.html#UY_COHO1HR_MERGED_MAG_PLASMA


\subsection{Geocentric Solar Ecliptic}
\label{sec:geocentric_solar_ecliptic}

The Geocentric Solar Ecliptic (GSE) coordinates are\\
GSE - Geocentric Solar Ecliptic\\
    X = Earth-Sun Line\\
    Z = Ecliptic North Pole

GSE coordinates are used in ACE solar wind data, etc.

%http://www.srl.caltech.edu/ACE/ASC/coordinate_systems.html
%https://www.spenvis.oma.be/help/background/coortran/coortran.html
%http://adsabs.harvard.edu/abs/1992P%26SS...40..711H

see Jursa1985, p.~4-3\\
``the polar axis is the axis inclined 11.5° to the axis of rotation, intersecting the earth surface at the point 78.5°N, 291.0°E which defines the geomagnetic north pole. This was once at one time the axis of the best centered-dipole approximation to the field''\\


\subsection{Geocentric Solar Magnetospheric}

GSM - Geocentric Solar Magnetospheric\\
X = Earth-Sun Line\\
Z = Projection of dipole axis on GSE YZ plane\\
%(Z-axis (positive) is perpendicular to the X-axis and parallel to the projection of the negative dipole moment on a plane perpendicular to the X-axis (the northern magnetic pole is in the same hemisphere as the tail of the magnetic moment vector).)\\

GSM is defined with a time dependent dipole axis.\\
the dipole axis orientation changes over time; at 1995 the northern pole was at $l = 288.59\degree$ and $b = 79.30\degree$; more recent year (2015)?... cite?\\
% https://www.spenvis.oma.be/help/background/coortran/coortran.html#MAG

\subsection{Heliographic Inertial}

Heliographic Inertial (HGI) coordinates\\
HGI coordinates are Sun-centered with the z-axis directed along the solar rotation axis and directed northward of the solar equator. The solar equator plane is inclined \SI{7.25}{\degree} from the ecliptic.\\

HGI coordinates; latitude range \SI{-7.25}{\degree}~to~\SI{-7.25}{\degree}\\
latitude variation (see Schwenn1990, p.~127)\\



\section{Astronomical constants}
\label{sec:astronomical_constants}

Astronomical unit: 1~au = 149\,597\,870\,700~m \citep{USNO2015}\\ %Astronomical constants
Solar mass: $M_\odot = 1.9884(2)\times10^{30}$~kg \citep{USNO2015}\\ %see also Mamajek2015
Nominal solar radius (photosphere): $R_\odot = \SI{695700}{\km}$ \citep{Mamajek2015}\\ %Astronomical constants
Sun escape velocity: $v_\text{esc} = 617.6$~km/s (Sun Fact Sheet...)\\
Solar rotation axis tilt: $i_\odot = 7.25$\textdegree{} \citep{USNO2015}\\ %C3; Obliquity to ecliptic
Solar surface rotation period at equator, sidereal: 25.05~d (Sun Fact Sheet...)\\
Nominal solar effective temperature (photosphere): $T_{\text{eff}\odot} = \SI{5772}{\kelvin}$ \citep{Mamajek2015}\\

%Astronomical constants
%http://asa.usno.navy.mil/static/files/2016/Astronomical_Constants_2016.pdf


% \section{Abbreviations}
% \label{sec:abbreviations}
% 
% Projects:
% \begin{description*}
% 	\item[AFFECTS] Advanced Forecast For Ensuring Communications Through Space
% 	\item[HELCATS] Heliographic Cataloging, Analysis and Techniques Service
% 	\item[FP7] Framework Programme 7
% 	\item[CGAUSS] Coronagraphic German And US SolarProbePlus Survey
% 	\item[OPTIMAP] OPerational Tool for Ionospheric Mapping And Prediction
% \end{description*}
% 
% Spacecraft:\\
% SPP -- Solar Probe Plus\\
% WISPR -- Wide-field Imager for Solar Probe\\
% ACE -- Advanced Composition Explorer\\
% 	MAG -- Magnetometer\\
% 	SWEPAM -- Solar Wind Electron Proton Alpha Monitor\\
% 	RTSW -- Real Time Solar Wind\\
% SDO -- Solar Dynamics Observatory\\
% SOHO -- Solar and Heliospheric Observatory\\
% STEREO -- Solar TErrestrial RElations Observatory\\
% 
% Organizations:\\
% NASA -- National Aeronautics and Space Administration\\
% 	SPDF -- Space Physics Data Facility\\
% NOAA -- National Oceanic and Atmospheric Administration\\
% 	SWPC -- Space Weather Prediction Center\\
% UGOE -- University of Göttingen\\
% IAG -- Institute for Astrophysics Göttingen\\
% GFZ -- GeoForschungsZentrum\\
% WDC-SILSO -- World Data Center-Sunspot Index and Long-term Solar Observations\\
% 
% Sun:\\
% DB -- disparition brusques (disappearing filaments?; quiescent filaments?)\\
% SSN -- sunspot number\\
% 
% Solar wind:\\
% IMF -- interplanetary magnetic field\\
% CME -- coronal mass ejection\\
% ICME -- interplanetary coronal mass ejection\\
% MC -- magnetic cloud\\
% HSS -- high speed stream\\
% CIR -- corotating interaction region\\
% SIR -- stream interaction region\\
% SB -- sector boundary\\
% BDE -- bidirectional electrons\\
% HCS -- heliospheric current sheet\\
% HPS -- heliospheric plasma sheet\\
% 
% Earth:\\
% Kp -- planetare Kennziffer\\
% Dst -- Disturbance storm time\\
% 
% Coordinate systems:\\
% GSE -- geocentric solar ecliptic\\
% GSM -- geocentric solar magnetospheric\\
% 
% Theories and techniques:\\
% MVA -- minimum variance analysis\\
% MHD -- magnetohydrodynamic\\
% GCS -- Graduated Cylindrical Shell\\
% CAT -- CME Analysis Tool\\


\section{Lognormal distribution}
\label{sec:lognormal_distribution}

This is a small summary about the lognormal probability distribution \citep[p.~780]{Bronstein2000}. The lognormal distribution is the distribution of a random variable $X$ if the logarithm of $X$ conforms to a normal distribution. Its shape is highly asymmetric, however in a semi-log plot the Gaussian bell curve is recognizable (see the second panel of \autoref{fig:lognormal_3panel_pdfcairo_plot}).
\begin{figure}[htb]
	\begin{floatrow}
		\ffigbox{
			\includegraphics[width=0.5\Xhsize]{figures_of_mine/gnuplots/lognormal_3panel_pdfcairo_plot.pdf}
		}{
			\caption{The lognormal probability density function ($\sigma = 1, \mu = 0$) plotted in a linear, semi-log and log-log way. remove borders...}
			\label{fig:lognormal_3panel_pdfcairo_plot}
		}
		\ffigbox{
			\includegraphics[width=\Xhsize]{figures_of_mine/gnuplots/lognormal_ms_pdfcairo_plot.pdf}
		}{
			\caption{Five lognormal distributions plotted with fixed $\sigma$ (top) and fixed $\mu$ (bottom). remove borders...}
			\label{fig:lognormal_ms_pdfcairo_plot}
		}
	\end{floatrow}
\end{figure}
Its probability density function is
\begin{align}
	f(x) &= \frac{1}{\sigma \sqrt{2 \pi} x} \, \text{e}^{- \frac{(\ln x - \mu)^2}{2 \sigma^2}}
\end{align}
with the location ($\mu$) and the shape parameter ($\sigma$). Changes in $\mu$ affect both the horizontal and vertical scaling of the function, whereas $\sigma$ has an influence on its shape (see \autoref{fig:lognormal_ms_pdfcairo_plot}).\\

Because it is a probability distribution, its area is normalized
\begin{align}
	\int_0^\infty f(x) \text{d} x = 1\,.
\end{align}

For a lognormally distributed random variable the geometric moments mean, standard deviation and variance are:
\begin{align*}
	&\mu_\text{g} = \text{e}^\mu ,\\
	&\sigma_\text{g} = \text{e}^\sigma ,\\
	&var_\text{g} = \text{e}^{\sigma^2}~~(!) .
\end{align*}

Its arithmetic moments are:
\begin{align*}
	&\mu_\text{a} = \text{e}^{\mu + \frac{\sigma^2}{2}} ,\\
	&\sigma_\text{a} = \text{e}^{\mu + \frac{\sigma^2}{2}} \, \left(\text{e}^{\sigma^2} - 1\right) ,\\
	&var_\text{a} = \sigma_\text{a}^2 .
\end{align*}

Other useful characteristics are the median and the mode
\begin{align*}
	&x_\text{median} = \text{e}^{\mu},\\
	&x_\text{mode} = \text{e}^{\mu - \sigma^2}\,.
\end{align*}
Note that for the lognormal distribution its median is equal to its geometric mean.\\

Applications of lognormal distributions...\\
%Limpert2001: Lognormal Distributions across the Sciences: Keys and Clues
%http://bioscience.oxfordjournals.org/content/51/5/341.full

Most natural quantities which can only be positive are lognormally distributed. e.g. animal body sizes?, animal life expectancies, financial stock prices...; income distributions.\\

%life expectancy analyses of economic, technical and biological processes. Bronstein2000


\section{Goodness of fit}
\label{sec:goodness_of_fit}

%figure with fit residuals here
$SSR$ -- sum of squared residuals\\
\begin{align}
	SSR = \sum_i (y_i - f_i)^2
\end{align}
data values $y_i$, fit function values $f_i$\\
$SSR_\text{red}$ -- reduced SSR, divided by number of degrees of freedom $\nu$\\
\begin{align}
	SSR_\text{red} = \frac{SSR}{\nu}
\end{align}
$TSS$ -- total sum of squares (in relation to the data mean)\\
\begin{align}
	TSS = \sum_i (y_i - \bar{y})^2
\end{align}
with data mean $\bar{y}$\\
%http://math.stackexchange.com/questions/1225103/whats-in-a-name-sum-of-squares
%https://en.wikipedia.org/wiki/Goodness_of_fit

$\chi^2$ -- chi-square\\
$\chi^2_\text{red}$ -- reduced chi-square, divided by number of degrees of freedom $\nu$\\

$R^2$ -- coefficient of determination\\
$0 \leq R^2 \leq 1$, ``values can be less than zero''!. if $R^2 = 1$ -> ideal fit; if $R^2 = 0$ -> bad fit
\begin{align}
	R^2 &= 1 - \frac{SSR}{TSS}
\end{align}
``In case of a single regressor, fitted by least squares, R2 is the square of the Pearson product-moment correlation coefficient relating the regressor and the response variable.'' cite from wikipedia \\%https://en.wikipedia.org/wiki/Coefficient_of_determination
%https://de.wikipedia.org/wiki/Bestimmtheitsma%C3%9F

Kolmogorov-Smirnov K-S-Test
%http://www.physics.csbsju.edu/stats/KS-test.html
%http://www.wessa.net/rwasp_Reddy-Moores%20K-S%20Test.wasp


\section{other}

%minimum variance analysis
minimum variance analysis (MVA)\\
determining magnetic cloud configuration \citep{Bothmer1998}\\

hodogramm?
%https://en.wikipedia.org/wiki/Hodograph



least-squares fit approximates mean
linear regression

robust statistics

Non-parametric inferential statistical methods are mathematical procedures for statistical hypothesis testing which make no assumptions about the probability distributions of the variables being assessed. The most frequently used tests include
- median
- percentiles (quartiles)
- Spearman's rank correlation coefficient

histogram

generalized mean
\url{http://en.wikipedia.org/wiki/Generalized_mean}


normal distribution

% inverse erf
% http://mathworld.wolfram.com/InverseErf.html
% 
% \begin{align}
% 	erfc^{-1}(x) = erf^{-1}(1 - x)
% \end{align}
% 
% quantile function: inverse cumulative distribution function (CDF) of normal distribution
% \begin{align}
% 	\mu - \sqrt{2}*\sigma*erfc^{-1}(2*x)
% \end{align}
% http://www.wolframalpha.com/input/?i=Quantile+function+normal+distribution


auto correlation

cross correlation\\
Pearson linear correlation\\
Spearman rank correlation

Correlation in Linear Regression: \url{http://www.stat.yale.edu/Courses/1997-98/101/correl.htm}\\
