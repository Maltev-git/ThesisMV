\chapter{Glossary}
\label{chap:glossary}

\section{Astronomical constants}
\label{sec:astronomical_constants}

Astronomical unit: 1~au = 149\,597\,870\,700~m \citep{USNO2015}\\ %Astronomical constants
Solar mass: $M_\odot = 1.9884(2)\times10^{30}$~kg \citep{USNO2015}\\ %see also Mamajek2015
Nominal solar radius (photosphere): $R_\odot = \SI{695700}{\km}$ \citep{Mamajek2015}\\ %Astronomical constants
Sun escape velocity: $v_\text{esc} = 617.6$~km/s (Sun Fact Sheet...)\\
Solar rotation axis tilt: $i_\odot = 7.25$\textdegree{} \citep{USNO2015}\\ %C3; Obliquity to ecliptic
Solar surface rotation period at equator, sidereal: 25.05~d (Sun Fact Sheet...)\\
Nominal solar effective temperature (photosphere): $T_{\text{eff}\odot} = \SI{5772}{\kelvin}$ \citep{Mamajek2015}\\

%Astronomical constants
%http://asa.usno.navy.mil/static/files/2016/Astronomical_Constants_2016.pdf


\section{Symbols}
\label{sec:symbols}

s/c - spacecraft\\
$B$ - magnetic field strength\\
$n$ - number density\\
...\\


\section{Abbreviations}
\label{sec:abbreviations}

Projects:
\begin{description*}
	\item[AFFECTS] Advanced Forecast For Ensuring Communications Through Space
	\item[HELCATS] Heliographic Cataloging, Analysis and Techniques Service
	\item[FP7] Framework Programme 7
	\item[CGAUSS] Coronagraphic German And US SolarProbePlus Survey
	\item[OPTIMAP] OPerational Tool for Ionospheric Mapping And Prediction
\end{description*}

Spacecraft:\\
SPP -- Solar Probe Plus\\
WISPR -- Wide-field Imager for Solar Probe\\
ACE -- Advanced Composition Explorer\\
	MAG -- Magnetometer\\
	SWEPAM -- Solar Wind Electron Proton Alpha Monitor\\
	RTSW -- Real Time Solar Wind\\
SDO -- Solar Dynamics Observatory\\
SOHO -- Solar and Heliospheric Observatory\\
STEREO -- Solar TErrestrial RElations Observatory\\

Organizations:\\
NASA -- National Aeronautics and Space Administration\\
	SPDF -- Space Physics Data Facility\\
NOAA -- National Oceanic and Atmospheric Administration\\
	SWPC -- Space Weather Prediction Center\\
UGOE -- University of Göttingen\\
IAG -- Institute for Astrophysics Göttingen\\
GFZ -- GeoForschungsZentrum\\
WDC-SILSO -- World Data Center-Sunspot Index and Long-term Solar Observations\\

Sun:\\
DB -- disparition brusques (disappearing filaments?; quiescent filaments?)\\
SSN -- sunspot number\\

Solar wind:\\
IMF -- interplanetary magnetic field\\
CME -- coronal mass ejection\\
ICME -- interplanetary coronal mass ejection\\
MC -- magnetic cloud\\
HSS -- high speed stream\\
CIR -- corotating interaction region\\
SIR -- stream interaction region\\
SB -- sector boundary\\
BDE -- bidirectional electrons\\
HCS -- heliospheric current sheet\\
HPS -- heliospheric plasma sheet\\

Earth:\\
Kp -- planetare Kennziffer\\
Dst -- Disturbance storm time\\

Coordinate systems:\\
GSE -- geocentric solar ecliptic\\
GSM -- geocentric solar magnetospheric\\

Theories and techniques:\\
MVA -- minimum variance analysis\\
MHD -- magnetohydrodynamic\\
GCS -- Graduated Cylindrical Shell\\
CAT -- CME Analysis Tool\\
