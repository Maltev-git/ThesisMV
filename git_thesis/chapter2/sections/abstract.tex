
\titley{Solar wind and CME influence on the magnetosphere}
\subtitley{Impact estimations derived from empirical correlations between in-situ solar-wind measurements and the geomagnetic \Kp{}~index}

% \author{M.~S.~Venzmer}
% 
% \institute{University of Goettingen, Institute for Astrophysics, Friedrich-Hund-Platz~1, 37077~Göttingen, Germany}
% 
% \date{First draft 28 April 2017; received date; accepted date }

\abstracty
{Variations in the Earth's magnetosphere are largely evoked by influence through the solar wind. These magnetospheric disturbances have diverse effects on the terrestrial environment. Especially the effects of severe geomagnetic storms created by coronal mass ejections (CMEs) pose various threats to sensitive technical systems and exposed humans. Thus, the development of quantitative forecasts for magnetospheric impacts caused by solar wind and CMEs is of major importance.}	%context
{This study's goals are to estimate the magnetospheric impact from solar activity in general, from solar wind and also to predict it for CMEs in particular. We present empirical dependencies between specific solar-wind parameters and the magnetospheric disturbance index~\Kp{}. These dependencies allow to nowcast the \Kp~index from upstream (L1) solar-wind in-situ measurements. Hence, also the magnetospheric impact of CMEs is estimated solely based on their arrival velocities, predicted from coronagraph observations. The prediction of solar-wind stream velocities, e.g., from coronal hole observations, enables to estimate their impact as well.}	%aims
{First, we estimate the long-term variations of the yearly average \Kp{}~values, which are contributed by solar activity. This is achieved via logarithmic fitting of a yearly sunspot number (SSN) dependency. In order to nowcast the \Kp~index from general solar-wind conditions, we use a correlation with the product of the parameters velocity and magnetic field z\~component in GSM coordinates (\vBz{}). To forecast the \Kp~index from estimated CME and stream velocities, we furthermore filter the solar-wind data, using flagged CME times from the solar-wind structures (SWS) list provided by \citet{Richardson2012}. The solar-wind data considered in our analyses consists of 35~years (1981--2016) of high-resolution minutely OMNI data, which is composed of multi-spacecraft intercalibrated in-situ measurements from \SI{1}{\au}. We evaluate various data processing methods and choose the methods resulting in the highest correlation coefficients with \Kp{}. We analyze the \Kp{} frequency distributions with respect to the depending parameters \vBz{} and velocity, derive their mean \Kp{} per interval and further compile functional dependencies via logarithmic fitting.}	%methods
{The obtained functional relations enable us to empirically estimate the mean \Kp{} impact from measured solar activity, in-situ solar wind, and remotely determined CME and streamer velocities.}	%results
{}	%conclusions

% \keywords{solar wind -- sun: coronal mass ejections (CMEs) -- earth}
% 
% \maketitle
