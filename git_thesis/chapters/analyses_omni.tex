\chapter{chapter2}

%COFI -- chapter outline and flow integration\\

-> example events CIR/HSS and CME\\

SWS: Rules of the Road: Please contact Ian Richardson about your use of this data.\\

V-Kp correlation similar to \citet{Elliott2013}; different data resolution and averaging method (3-hour maximum of 1~min data)


\section{Solar wind interaction processes with the magnetosphere}
there are several underlying physical mechanisms, whose contribution is not yet quantified'?'\\
physical mechanisms:\\
- reconnection\\
- compression\\
- turbulence\\
- induction?\\

three ways for solar wind momentum and energy transfer into magnetosphere:\\
- sw entering sphere\\
- waves/eddies\\
- reconnection\\

%[the relation of these events was confirmed two decades ago by (who and \citet{Bothmer1993})]\\

\section{Solar wind--magnetosphere coupling functions}
%theory from literature
coupling mechanisms and therefrom derived functions\\
VBzgsm - E-field...\\

Studies finding best coupling function, Newell, etc.\\

\citet{Newell2007}: universal sw-magnetosphere coupling function (opening flux rate)\\
\citet{Newell2008}: coupling function merging and viscous term\\
merging and viscous terms (reconnection and turbulence)\\
merging term: rate magnetic flux is opened at the magnetopause ($d\Phi_\text{MP}/dt$)\\
viscous term: reconnection due to Kelvin-Helmholtz instabilies at the boundary ($n^{1/2} v^2$)\\
equation for the least variance linear prediction of Kp: $Kp = 0.05 + \num{2.244e-4} d\Phi_\text{MP}/dt + \num{2.844e-6} n^{1/2} v^2$\\
combination of both terms works best (r = 0.866)\\

see also \autoref{sec:solar_wind_magnetosphere_coupling}\\


\section{Parameter selection}
%fix usage of Kp index\\
In our analyses we use the planetary geomagnetic disturbance indicator Kp (see Section~XX...), because it is designed to measure solar particle radiation by its magnetic effects (cite? Bartels...?).\\
and has close relation to aurorae, NOAA scale\\

\section{Data selection}
choosing data time range\\
the Kp time series started in XXXX, when there were no spacecraft to measure in situ solar wind --> time range defined by available solar wind data\\
OMNI data set -> longest continuous solar wind data set\\

\section{Solar cycle influence}
solar cycle dependence\\
- parameter time plots\\
- parameter vs SSN matrix-plots\\
- cc time plots\\
- cc vs SSN plots\\

\section{CME correlations}
same analysis for CMEs\\
- parameter time plots\\
- parameter vs SSN matrix-plots\\
- cc time plots\\
- cc vs SSN plots\\

\section{Solar wind structure analyses}
ACE solar wind time series and event list\\

see OPTIMAP events\\
events from ACE CME list...\\

%method: automatic list creation by event detection via parameter thresholds

sample CME analyses (MVA, -> Kp)\\
