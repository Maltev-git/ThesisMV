
\chapter{Solar wind and CME influence on the magnetosphere}
\label{chap:chapter2}

Impact estimations derived from empirical correlations between in-situ solar wind measurements and the geomagnetic \Kp{}~index\\

%\section{Abstract}

%context
Variations in the Earth's magnetosphere are largely evoked by influence through the solar wind. These magnetospheric disturbances have diverse effects on the terrestrial environment. Especially the effects of severe geomagnetic storms created by coronal mass ejections (CMEs) pose various threats to sensitive technical systems and exposed humans. Thus, the development of quantitative forecasts for magnetospheric impacts caused by solar wind and CMEs is of major importance. The analyses in this chapter are based on my work done for the project Advanced Forecast For Ensuring Communications Through Space (AFFECTS).

%aims
The goals of this study are to estimate the magnetospheric impact from solar wind and also to predict it for CMEs and streams in particular. Empirical dependencies between specific solar wind parameters and the magnetospheric disturbance index~\Kp{} are presented. These dependencies allow to nowcast the \Kp~index from upstream (L1) solar wind in-situ measurements. Hence, also the magnetospheric impact of CMEs is estimated solely based on their arrival velocities that may be predicted from coronagraph observations. The prediction of solar wind stream velocities, e.g., obtained from coronal hole observations, enables to estimate their impact as well.

%methods
First, I estimate the long-term variations of the \Kp~index which are contributed by solar activity and seasonal effects. A functional dependency of the average yearly \Kp~index on the yearly sunspot number (SSN) is derived. In order to nowcast the \Kp~index from general solar wind conditions, I apply a correlation with the solar wind electric field -- the product of the parameters velocity and magnetic field z-component in GSM coordinates: $E = v B_\text{z}$. A suitable logarithmic function is constructed and its fit to the data results in a functional relation between $E$ and \Kp{}. For the purpose of forecasting the \Kp~index from estimated CME and stream velocities, I furthermore filter the solar wind data, using flagged CME times from the solar wind structures (SWS) list provided by \citet{Richardson2012}. Logarithmic fits to the data yield in separate \Kp{} relations for CMEs and streams.

The solar wind data considered in these analyses consists of 35~years (1981--2016) of high-resolution minutely OMNI data, which is composed of multi-spacecraft intercalibrated in-situ measurements from \SI{1}{\au}. I analyze the \Kp{} frequency distributions with respect to the depending parameters $E$ and velocity, derive their mean \Kp{} per interval and further compile functional dependencies via logarithmic fitting. I evaluate the relations' prediction performance by analyzing their forecast errors and conparing their true skill statistic.

%results
As the obtained functional relations are simple, they cannot compete with ANNs, \Kp{} persistence, or full-fledged solar wind coupling functions. However, they enable empirical estimations of the mean \Kp{} impact from measured solar activity, in-situ solar wind, and remotely determined CME and stream velocities.

%conclusions


\section{Introduction}
%%% motivation
Since the early 19th~century it is known that variations in the solar wind evoke disturbances in the magnetosphere \citep{Bartels1962}. Especially strong disturbances, called geomagnetic storms, can be provoked by coronal mass ejections (CMEs), which are embedded within the solar wind. The causes of the strongest geomagnetic storms are the compression of the solar wind magnetic field lines within the CME shock front and the enhanced field strenghts of magnetic clouds, which are enclosed in CMEs \citep{Bothmer1993,Bothmer1995}. The consequences of strong geomagnetic disturbances are a threat to sensitive technical systems and exposed humans. Therefore it is important to know when magnetospheric disturbances will occur and how large they will become.\\

\Kp{} is designed for... (cite?). I use this magnetospheric disturbance index to correlate it with near-Earth solar wind measurements. More detailed information on the \Kp{}~index can be found in \autoref{sec:kp_index}.\\

%results
With the results presented in this chapter, I elaborate the step from solar wind properties at Earth to the forecast of the possible impact strength on the terrestrial magnetosphere. I derive empirical correlations between the solar wind properties and the geomagnetic \Kp~index, in order to obtain the capability to forecast \Kp{} values. The derived functional dependencies can be used to nowcast/forecast the \Kp~index.\\

differences to existing studies...\\

%general solar wind nowcast; rt-data
In-situ measurements of solar wind are made almost continuously, e.g., at the first Lagrange point (L1), in front of the magnetosphere. Since 1963 several spacecraft collected more than 50~years of solar wind data. The latest spacecraft, that is, Wind, ACE, and DSCOVR (launched in early 2015), provide real-time solar wind data online\footnote{Wind real-time data website: \urlfoot{https://pwg.gsfc.nasa.gov/windnrt/}} \footnote{ACE real-time data website: \urlfoot{http://www.swpc.noaa.gov/products/ace-real-time-solar wind}} \footnote{DSCOVR real-time data website: \urlfoot{http://www.swpc.noaa.gov/products/real-time-solar wind}}.
% solar wind impacts
These solar wind real-time data are used to nowcast various effects on the Earth's magnetosphere, such as the position of the magnetospheric bow shock in front of the Earth, the magnitude of geomagnetic disturbances, the positions of the polar auroral ovals, the variation of the total electron content (TEC) of the ionosphere, and the positional error of global navigation satellite systems (GNSS).\\


%CME velocity forecast
The velocity and the direction of CMEs can be determined in their early near-Sun stages via remote tracking with coronagraph white-light observations. Using these parameters as input for CME propagation models, their possible arrival time and arrival velocity at Earth can be derived.\\

There are efforts to predict the direction and strength of the magnetic field from the flux rope geometry and orientation \citep{Savani2015}. However, the compressed solar wind magnetic field in the sheath region is of at least similar importance for magnetospheric disturbances \citep{Huttunen2004}. As the solar wind $B_\text{z}$ is quite random with low autocorrelation time, its prediction is not yet implemented in the current forecasts (cite?).\\

%%% synopsis
The objectives of the analyses performed and found in this chapter are to estimate the magnetospheric impact of solar wind and to predict it for CMEs and streams in particular. In \autoref{sec:long_term_variations} I determine the magnitudes of the long-term \Kp{} changes due to solar activity and measure the extent of seasonal variations stemming from the Earth's orbit. In order to nowcast the \Kp{}~index, I quantify the solar wind influence on \Kp{} by deriving a functional relation with the solar wind E-field (\autoref{sec:kp_nowcast}). Finally, for the purpose of enabling \Kp{} forecasts from remote observations, I estimate the \Kp{} impact coming from CMEs and streams separately by deriving functional dependencies with their velocities (\autoref{sec:kp_forecast_from_remote_CME_and_stream_observations}).


\section{Long-term variations of the \Kp{}~index}
\label{sec:long_term_variations}

\subsection{Solar activity influence}
The \Kp{} data is obtained from the GFZ~Potsdam, where the index is currently maintained\footnote{GFZ~Potsdam \Kp~index website: \urlfoot{http://www.gfz-potsdam.de/en/kp-index/}}. The data used in this analysis covers the time period 1932--2016, see also \autoref{sec:kp_data}. Its frequency distribution shows that the highest frequencies occur around low \Kp{} values of 1+, see \autoref{fig:Kp_histogram_b}. Going to higher \Kp{} values, the frequencies decline asymptotically towards zero -- a \Kp{} value of 9o occurred only 29 times in this time interval.\\
\begin{figure}[htb]
	\begin{floatrow}
		\ffigbox{
			\includegraphics[width=0.46\textwidth]{figures_of_mine/chapter2/Kp_histogram_b.pdf}
		}{
			\caption{\Kp{} frequency distribution for the time period 1932--2016. The inset shows a zoomed-in view of the high-value tail. The \Kp{} data is obtained from the GFZ~Potsdam.}
			\label{fig:Kp_histogram_b}
		}
		\ffigbox{
			\includegraphics[width=0.95\Xhsize]{figures_of_mine/chapter2/Kp_histogram_yearlySSN_b.pdf}
		}{
			\caption{Yearly \Kp{} frequency distributions during the period 1932--2016, sorted and colored by yearly SSN. All distributions are normed to be of equal area. The \Kp{} data is obtained from the GFZ~Potsdam and the yearly SSN data from the SILSO World Data Center.}
			\label{fig:Kp_histogram_yearlySSN_b}
		}
	\end{floatrow}
\end{figure}
Obviously, the general \Kp{} distribution as seen in \autoref{fig:Kp_histogram_b} averages over solar activity. Solar activity is generally tracked with the international sunspot number (SSN). SSN data from the time period \citeyear{sidc1917} is used in the present analyses. The data is obtained from the online catalog\footnote{WDC-SILSO website: \urlfoot{http://www.sidc.be/silso/}} provided by the World Data Center -- Sunspot Index and Long-term Solar Observations (WDC-SILSO), Solar Influences Data Analysis Center (SIDC), Royal Observatory of Belgium (ROB).

The \Kp{} frequency distributions' shape varies with solar activity (cite?). This is visible in the yearly distributions, sorted and colored by yearly SSN in \autoref{fig:Kp_histogram_yearlySSN_b}.

The distribution's peak position scales with SSN, that is, a high yearly SSN results in a higher abundance of large \Kp{} values as well (cite?).

The time series of yearly average \Kp{} values from the years 1932--2016 shows a solar cycle imprint, see the top graphs in \autoref{fig:yearly_kp-ssn_correlation_c}.
\begin{figure}
	\fcapside[\FBwidth]{
		\includegraphics[width=0.6\textwidth]{figures_of_mine/chapter2/yearly_kp-ssn_correlation_c.pdf}
	}{
		\caption{Yearly \Kp~index distributions (shaded area) with their mean values for the time period 1932--2016 and yearly SSN with cycle number for the time period 1917--2016 (top panels). The Pearson correlation coefficients with the yearly SSN are calculated for time lags back to \num{-15}~years (bottom panel). The \Kp{} data is obtained from the GFZ~Potsdam and the yearly SSN data from the SILSO World Data Center.}
		\label{fig:yearly_kp-ssn_correlation_c}
	}
\end{figure}
The \Kp{} pattern follows the solar cycle minima and maxima as well as the changes in magnitude between solar cycles (cite?). The yearly mean \Kp{} shifts about 1~\Kp~unit for both variations separately. As expected, the correlation with solar activity shows an 11-year period, see bottom graph in \autoref{fig:yearly_kp-ssn_correlation_c}. The highest correlation coefficient of 0.60 is found with a time lag of $-1$~year, that is, the yearly average \Kp{} follows the SSN of the previous year.
%Kp-ssn cc: 0.5971

The yearly mean \Kp~index with respect to the 1-year lagged SSN shows a raise with increasing SSN, as seen from \autoref{fig:Kp_SSN_fit_d}.
\begin{figure}
	\fcapside[\FBwidth]{
		\includegraphics[width=0.6\textwidth]{figures_of_mine/chapter2/Kp_SSN_fit_d.pdf}
	}{
		\caption{Yearly mean \Kp~index with respect to 1-year lagged SSN (+) with the weighted logarithmic fit (dashed line). The error bars denote the SSN standard deviation and the relative weight from the yearly data coverage. The shaded area represents the fit error band derived from the estimated standard deviations of the fit parameters. The logarithmic function (\autoref{eq:log_fit_function}) is used for the weighted fit. The yearly \Kp{} mean values are calculated from GFZ~Potsdam data and the yearly SSN is obtained from the SILSO World Data Center.}
		\label{fig:Kp_SSN_fit_d}
	}
\end{figure}
In order to obtain an analytical relation for this dependency, perform a least-squares regression fit. \Kp{} itself is a quasi-logarithmic index, so it is apparent to use a logarithmic fit function:
\begin{align}
	f(x) = a \cdot \ln(x) + b	\,.	\label{eq:log_fit_function}
\end{align}
The resulting fit parameters are $a = 0.281(43)$ and $b = 1.05(19)$; the numbers in parentheses are the estimated standard deviations.
They lead to the relation
\begin{align}
	\Kp(ssn) = 0.28 \cdot \ln(ssn) + 1.1	\,,	\label{eq:kp_ssn_relation}
\end{align}
% log fit parameters:
% a 0.281126         +/- 0.04267
% b 1.04923          +/- 0.19
which is plotted in \autoref{fig:Kp_SSN_fit_d}. This means that for an average yearly SSN of 1 the mean \Kp{} is $1.05(20)$ and for a SSN of 300 it is $2.65(31)$. The numbers in parentheses are the errors on the corresponding last digits of the quoted value. They are calculated via error propagation from the estimated standard deviations of the fit parameters.
% SSN	Kp	Kp_err
% 1	1.0492	0.189
% 10	1.6965	0.213
% 50	2.1489	0.252
% 100	2.3438	0.273
% 200	2.5387	0.295
% 300	2.6527	0.308


\subsection{Seasonal variations}
On top of the yearly variations, seasonal variations exist in the magnetospheric disturbances as well. In the months May--August the \Kp{} peak frequency is higher than in the remaining months of the year, whereas in March/April and September/October \Kp{} values larger than 3 are more abundant. This is apparent from looking at the monthly \Kp{} frequency distributions plotted in \autoref{fig:Kp_histogram_monthly}.
\begin{figure}[htb]
	\begin{floatrow}
		\ffigbox{
			\includegraphics[width=0.46\textwidth]{figures_of_mine/chapter2/Kp_histogram_monthly.pdf}
		}{
			\caption{Average monthly \Kp{} frequency distributions of the time period 1932--2016, colored by month of the year. The \Kp{} data is obtained from the GFZ~Potsdam.}
			\label{fig:Kp_histogram_monthly}
		}
		\ffigbox{
			\includegraphics[width=0.46\textwidth]{figures_of_mine/chapter2/Kp_seasonal_e.pdf}
		}{
			\caption{\Kp{} frequency distributions by month for the time period 1932--2016 with median (white dashed) and quartile values (white dotted). The other dotted lines mark the upper eighth, 16th, 32nd, and 64th parts. The bin size is 1~month and \SI{1/3}{\Kp}~unit respectively.}
			\label{fig:Kp_seasonal_e}
		}
	\end{floatrow}
\end{figure}
These \Kp{} changes arise from seasonal variations of the solar wind parameters at Earth, which stem from Earth's yearly changes in orbital distance and heliographic latitude. The Earth's rotation axis tilt adds another seasonal effect, the tilt changes the direction of the Earth's magnetic dipole axis to the Sun over the year.

Earth's distance to the Sun varies over the course of a year by \SI{+-1.67}{\percent}, see appendix \autoref{sec:sun_earth_orbit_geometry}. The solar wind parameters scale via power-law dependencies with solar distance, as it is described in the following \autoref{chap:empirical_solar_wind_model_for_the_inner_heliosphere} and accordingly in \citet{Venzmer2018}. For example, the solar wind proton density scales with about $r^{-2}$, this leads to a yearly variation in density of about \SI{6.7}{\percent}. These yearly solar wind variations have direct influence on the \Kp{}~index.

The Sun's rotation axis tilt angle to the ecliptic is \SI{+-7.25}{\degree} and that for Earth is \SI{+-23.44}{\degree}, see also appendix \autoref{sec:sun_earth_orbit_geometry}. The solar wind magnetic field strength varies with heliographic latitude (cite?). The solar wind influence on the \Kp{}~index depends on its coupling efficiency with the magnetosphere. Furthermore, the rate of magnetic reconnection between solar wind and the Earth's magnetosphere depends on both fields' orientation to each other (parallel/antiparallel). Additionally, the tilt of the magnetic dipole axis to the rotation axis -- a few degrees for the Sun during cycle minima (extreme during solar maxima...) and about \SI{10}{\degree} for the Earth -- complicates this system even more.
%geomagnetic dipole tilt: Planetary fact sheet: 11.2 (Model GSFC-12/83)
%variation range in the time period 1930--2016: 9.6--11.5° (http://wdc.kugi.kyoto-u.ac.jp/poles/polesexp.html)

So the \Kp{} variation effects originate from the seasonal change in the solar tilt, the Earth's tilt, and the Earth's distance. Thorough analyses of the seasonal variations were already performed by \citep{Cortie1912} (more cites?). Thus, I just quantify the bulk magnitude of these effects in order to consider them as relative uncertainties when using the solar activity relation (\ref{eq:kp_ssn_relation}). Looking at the \Kp{} frequency distributions by month -- seen in \autoref{fig:Kp_seasonal_e} -- it is apparent that for high values ($\Kp{} > 4$), there exist yearly frequency maxima at the equinoxes and frequency minima at the solstices, as was early described by \citet{Cortie1912}. It is evident from the plotted quantiles that this semiannual variation amounts to $1/3$~\Kp~unit at small \Kp~values and up to 4/3~\Kp~units at higher \Kp~values.


\section{\Kp{} nowcast from in-situ solar wind measurements}
\label{sec:kp_nowcast}

\subsection{Coupling function}

The coupling between the solar wind and the magnetosphere is governed by reconnection and compression of the magnetic field lines, as described in \autoref{sec:solar_wind_coupling_mechanisms}.\\

%vBzgsm relation
In this work I settle and work with the electric field as the coupling function, \autoref{eq:coupling_vxBz}, because...\\
For which reason do I choose this coupling function over others?\\

The solar wind electric field y"~component $E_\text{y}$ approximates $\vect{E}$ under special circumstances, see derivation in \autoref{sec:electric_field_at_the_magnetopause}. It is the product of the radial proton velocity $v_\text{x}$ and the magnetic field z"~component $B_\text{z}$:
\begin{align}
	E_\text{y} = -v_\text{x} \, B_\text{z}\,.
\end{align}
If not specified otherwise, $B_\text{z}$ is always meant to be in GSM coordinates hereafter.\\


%velocity relation
The solar wind velocity sticks mostly to its radial flow direction, that is, it rarely deviates up to \SI{0.0}{\degree} (correct and cite...). Thus, the absolute flow speed $v$ can be used instead of the vector component.\\

argue for \vBz:\\
- 3hmin(\vBz) performs in rank correlation slightly better than the sophisticated Newell formula. really?\\
- simple to calculate\\
- ...\\


Using \vBz{} implies that both quantities are sufficiently independent. $B$ and $v$ are dependent, as can be seen from fig~XX, however, as $v$ and $\theta_c$ are, so are $v$ and $B_\text{z}$, see fig~XX.\\

% velocity relation
It is also known that the solar wind velocity itself already correlates strongly with the \Kp~index.\\

I use this velocity relationship for obtaining \Kp{} proxies from CME and solar wind stream data.\\


\subsection{Data correlation}
\label{sec:data_correlation}
%determine data basis
The \Kp{} time series started in 1932 when there existed no spacecraft to measure solar wind in~situ. Thus, the maximal surveyed time range is defined by the available in-situ solar wind data.\\

Savani2017:\\
``Although Kp is defined for 3 h periods, predicted variations in the IMF are made on shorter time scales and can influence the level of geomagnetic activity (e.g., geomagnetically induced currents). Hence, averaging the IMF over intervals that match those of Kp may suppress features that are important drivers of geomagnetic activity. Nevertheless, for the purpose of this paper, we choose to average the Kp estimate from the field vectors predicted by the BSS model over the same 3 h intervals as the Kp values, as this currently represents the service provided by NOAA/SWPC.''\\

%argue for averaging method
The \Kp{}~index represents maximal variations within 3-hour time intervals (see \autoref{sec:kp_index}). Any solar wind parameter that will be correlated with \Kp{}, obviously also has to have the same time resolution. In addition to adapting the time resolution, it has to be considered by which means this should be done. Simple 3-hour average values are expected to have a weaker correlation than the solar wind parameter's 3-hourly maximal variation.
%argue for high resolution, deliberate between hourly and minutely data
It is self-evident that the 3"~hour maximal variations are higher when using high resolution data. Thus, to be able to correlate \Kp{} with solar wind data, high resolution data, that is, much shorter than the 3"~hour resolution, are needed to determine the maximal solar wind variations within each 3"~hour interval.\\

The OMNI data collection constitutes the longest continuous solar wind measurements made at \SI{1}{\au}. There exist two OMNI data sets with different time resolution -- the hourly version extends back to 1963 and the minutely version extends back to 1981. Although it is shorter, I choose to apply the minutely data set, in order to benefit from the higher resolution. Thus, the work presented in this chapter is based on the minutely OMNI data set with a time duration spanning from 1981 until end of 2016.\\

vBz...\\

The data is reduced to 3"~hour averages and 3"~hour minima in order to match the \Kp{} data resolution and evaluate the advantage of high resolution data. The \Kp{}-\vBz{} Pearson correlation coefficients for the two differently processed data versions are plotted over time shift in \autoref{fig:cc_lag_data_d_KpvsVBzgsm}. The data with 3"~hour minimum processing shows a much better correlation than the 3"~hour average data. Both curves show a negative correlation and their minima lie at a time shift of zero, with coefficients of $r_\text{min} = -0.72$ and $r_\text{avg} = -0.36$.

The OMNI data represents the solar wind at the location of the magnetospheric bow shock. It takes the plasma only a couple of minutes to arrive at the magnetopause and influence the geomagnetic field. Therefore the best correlation at the time shift of zero is expected as it is merely tested in coarse 3"~hour steps.\\
\begin{figure}[htb]
	\begin{floatrow}
		\ffigbox{
			\includegraphics[width=0.46\textwidth]{figures_of_mine/chapter2/cc_lag_data_d_KpvsVBzgsm.pdf}
		}{
			\caption{\Kp-\vBz{} correlation coefficients for different time shifts up to \num{+-24}~hours. The minutely OMNI data from 1981--2016 is reduced to 3"~hour averages (black line) and 3"~hour minima (red line).}
			\label{fig:cc_lag_data_d_KpvsVBzgsm}
		}
%highest correlation coefficients:
%min:	0.00000    -0.717240
%mean:	0.00000    -0.362237
%max:	0.00000     0.293137
		\ffigbox{
			\includegraphics[width=0.46\textwidth]{figures_of_mine/chapter2/histogram_VBzgsm.pdf}
		}{
			\caption{Frequency distributions for the \vBz{} product. The minutely OMNI data from 1981--2016 is reduced to 3"~hour averages (black line) and 3"~hour minima (red line).}
			\label{fig:histogram_VBzgsm}
		}
%vBz frequency shifts:
% min shift: -1250
% mean shift: -250
% max shift: -750
	\end{floatrow}
\end{figure}

The reduction to 3"~hour minimum values shifts the \vBz{} frequency distribution asymmetrically to negative values, whereas the averaged data is scattered around zero, see \autoref{fig:histogram_VBzgsm}.



\subsection{Functional dependency for solar wind}
%distribution
The frequency distribution in \Kp-\vBz{} space is shaped like a candle flame, inclined to negative values by a light breeze, see top panel in \autoref{fig:Kp_2dhistogram_VBzgsm_sws_e}.
\begin{figure}
	\fcapside[\FBwidth]{
		\includegraphics[width=0.6\textwidth]{figures_of_mine/chapter2/Kp_2dhistogram_VBzgsm_sws_e.pdf}
	}{
		\caption{\Kp{} versus \vBz{} frequency distribution (top) and its relative distribution (bottom) with the mean \Kp{} values (solid line) and their mean absolute deviation (dotted lines). It is 3-hour minimum data from the minutely OMNI data set (1981--2016). The bin size is \SI{500}{\km\per\s\nano\tesla} and \SI{1/3}{\Kp}~unit respectively.}
		\label{fig:Kp_2dhistogram_VBzgsm_sws_e}
	}
\end{figure}
%dependency
In order to determine a functional dependency, I focus on the relative frequencies per \vBz-interval and their mean \Kp{} values, which are plotted in the bottom panel of \autoref{fig:Kp_2dhistogram_VBzgsm_sws_e}. The mean absolute deviation (MAD) of the mean has a mean size of \SI{0.7}{\Kp}~units. This probability distribution is asymmetrically V-shaped around zero, having a larger and steeper negative arm than positive arm. The asymmetry also exists for 3-hour mean data, thus this effect is not a result of the data reducing method (3-hour minimum) (fig...?). Rather the steeper negative arm is a consequence of the half-wave rectifier coupling of the solar wind magnetic field direction to the magnetosphere, as described in \autoref{sec:solar_wind_coupling_mechanisms}.\\
%MAD: 2.211/3 = 0.737 Kp units

stress that it is an empirical fit...\\
using an appropriate type of a fit function...\\

%determine fitting functions
Since the \Kp~index has a quasi-logarithmic scaling (see \autoref{sec:kp_index}), a logarithmic function is the obvious choice as a fit function. Furthermore, the depending argument consists of a product of two solar wind parameters which individually scale logarithmically with \Kp{}. These reasons are why I use the logarithm of a parabola for the fitting approach:
\begin{align}
	f(x) &= \ln\left(x^2\right)	\,.	\label{eq:log_square_function}
\end{align}
I also introduce a horizontal shifting parameter $x'$ because the distribution's center is slightly offset. To be able to replicate the asymmetry in both arms, I further split the fit function into a negative and a positive part:
\begin{align}
	f(x) &=
	\begin{cases}
		\,f_-(x) &\text{for } x < 0	\,,\\
		\,f_+(x) &\text{for } x \ge 0	\,.
	\end{cases}	\label{eq:log_square_fit_function}
\end{align}
This way, both arms can be scaled individually with scaling factors for the negative and positive parts $a_-$ and $a_+$. The resulting logarithmic fit function parts are
\begin{align}
	f_-(x) &= a_- \cdot \ln\left(\left(x + x'\right)^2 + b\right) + y'	\,,\\
	f_+(x) &= a_+ \cdot \left(f_-(x) - f_-\left(-x'\right)\right) + f_-\left(-x'\right)	\,,
\end{align}
with the vertical shifting parameter $y'$ and the depth parameter $b$. The resulting fit curve is plotted in \autoref{fig:Kp_2dhistogram_VBzgsm_sws_fit_e} with the fit coefficients $a_- = 1.258(19)$, $x' = 163(20)$, $b = \num{1.416(68)e6}$, $y' = -17.04(33)$, and $a_+ = 0.467(20)$ for units of [\si{\km\per\s \nano\tesla}].
%high precision values:
% a_- = 1.25788(0.019)\\
% y' = -17.0394(0.33)\\
% a_+ = 0.467039(0.0197)\\
% b = 1.41639e6(0.067795e6)\\
% x' = 162.907(20.642)\\
\begin{figure}
	\fcapside[\FBwidth]{
		\includegraphics[width=0.6\textwidth]{figures_of_mine/chapter2/Kp_2dhistogram_VBzgsm_sws_fit_e.pdf}
	}{
		\caption{Mean \Kp{} values (+) and MAD values (dotted lines) per \vBz~interval. The error bars represent the relative data count. The logarithmic fit (dashed line) is plotted with a mean MAD band (shaded area). The splitted function (\ref{eq:log_square_fit_function}) is used for the weighted fit.}
		\label{fig:Kp_2dhistogram_VBzgsm_sws_fit_e}
	}
\end{figure}
The mean MAD is about 0.7~\Kp{}~units. Thus, the solar wind dependency relation condenses to:
\begin{align}
	\text{\Kp}_-\left(vB_\text{z}\right) &= 1.258 \cdot \ln\left(\left(vB_\text{z} + 163\right)^2 + \num{1.416e6}\right) - 17.04	\,,	\label{eq:kpvsvbz_dependency_function_negative}\\
	\text{\Kp}_+\left(vB_\text{z}\right) &= 0.467 \cdot \left(\text{\Kp}_-\left(vB_\text{z}\right) - \text{\Kp}_-(-163)\right) + \text{\Kp}_-(-163)	\,.	\label{eq:kpvsvbz_dependency_function_positive}
\end{align}
This relation can be used together with real-time in-situ measurements from spacecraft located at L1 (see Section~XX rt sources) to nowcast the actual \Kp~index.\\

In order to demonstrate this relation with actual in-situ data, I calculate the \Kp{} estimate for the example CME from \autoref{fig:ACE_64s_v7_thesis_CME_2013-6-26_6} and include the 3"~hour minimum value of \vBz{} in the plot, see \autoref{fig:example_sw_plot_vBz_b_2013-6-26_6}.
\begin{figure}[htb]
	\centering
	\includegraphics[width=\textwidth]{figures_of_mine/chapter2/example_sw_plot_vBz_b_2013-6-26_6.pdf}
	\caption[\lofimage{figures_of_mine/chapter2/example_sw_plot_vBz_b_2013-6-26_6.pdf}]
	{Solar wind measurements, official \Kp{}~index, and estimated \Kp{} for the time period from 26~June to 2~July 2013. The solar wind parameters are the magnetic field strength, its z"~component in GSM coordinates, and the velocity. I also plot the product \vBz{} with its 3"~hour minimum for illustration. The solar wind data are from the minutely OMNI data set. The official \Kp{}~index is obtained from the GFZ~Potsdam. The \Kp{} estimate is derived via Equations~\ref{eq:kpvsvbz_dependency_function_negative} and \ref{eq:kpvsvbz_dependency_function_negative}. The orange band indicates the mean MAD.}
	\label{fig:example_sw_plot_vBz_b_2013-6-26_6}
	\addtocontents{lof}{\smallskip\protect\center I created the figure myself.\medskip}
\end{figure}
It can be seen that in this period the \Kp{} estimate traces the actual \Kp{}~index pretty well within the mean MAD band. However, deviations are found at the times of the initial shock, and the start and middle of the MC.\\



\section{\Kp{} forecast from remote CME and stream observations}
\label{sec:kp_forecast_from_remote_CME_and_stream_observations}

It makes sense to separately look at CMEs and solar wind streams -> see event comparison fig.\\
same velocity -> different \Kp{} effect\\

velocity forecast methods, see basics...\\

To make use of the heads-up time for CMEs, I neglect its magnetic field part, which is difficult to determine from remote observations and simplify the coupling relation (\ref{eq:coupling_vxBz}) from before. Therefore, the only coupling parameter remaining is the solar wind velocity.

Savani2017:\\
``\textit{Prior evidence has suggested the reason be due to southerly IMF strength being the most dominant solar wind parameter for driving geomagnetic activity at Earth. The strong correlation of solar wind speed to geomagnetic activity is evident during CME conditions but is only considered the most dominant parameter during high-speed streams} [Holappa et al., 2014].''\\


\subsection{Solar Wind Structures list}
For the following analysis, I use the list of solar wind structures (SWS) created and updated by \citet{Richardson2000} and \citet{Richardson2012}, who characterized the near-Earth solar wind structures since 1963. All periods related to ICMEs in the OMNI solar wind data set were identified and flagged.\\

list identifying criteria... look into their paper!\\

The SWS list for 1963--2016 was kindly provided by Ian~Richardson (private communication). into acknowledgments...\\

SWS list for 1963--2015 by \citep{Richardson2000,Richardson2012} is available via registration at CEDARweb\footnote{CEDARweb website for Solar Wind Structures: \urlfoot{http://cedarweb.vsp.ucar.edu/wiki/index.php/Tools_and_Models:Solar_Wind_Structures}}.\\
%List of near-Earth ICMEs since January 1996 by \citet{Cane2003,Richardson2010}. Available as ACE Level~3 data for the period 1995--mid2016\footnote{ACE Level~3 data website -- list of near-Earth ICMEs: \urlfoot{http://www.srl.caltech.edu/ACE/ASC/DATA/level3/icmetable2.htm}}.\\

The CME fraction in the OMNI data is \SI{15.8}{\%} during the time period 1981--2016 (this accumulates to 5.53~years).
%and that for the period 1963--2016 is \SI{17.0}{\%} (9.01~years).\\
This percentage is an average value. It varies heavily with the solar activity cycle and with the appearance of individual active regions on the solar surface. In the following, I refer to the remaining part without CMEs as ``solar wind streams'', because it is composed entirely from a mixture of slow wind flows, fast wind streams, and their interaction regions.

% acknowledgments:\\
% The hourly solar wind structure list was kindly provided by Ian~Richardson of the NASA Goddard Space Flight Center and CRESST/University of Maryland via the CEDAR Database at the National Center for Atmospheric Research, which is supported by the National Science Foundation.\\

from OMNI data only?\\
permission received.\\
%http://cedarweb.vsp.ucar.edu/wiki/index.php/Tools_and_Models:Solar_Wind_Structures


\subsection{Data correlation}
Again, I calculate 3-hour extreme values using the minutely OMNI data to profit from higher correlation coefficients, as done before for the data processing of the \vBz{} analysis in \autoref{sec:data_correlation}. The velocity only has positive values, thus its extreme values are 3-hour maximum values. The comparison between the 3-hour maximum and the 3-hour mean frequency distributions shows that their mean position shifts from a velocity of 405 to \SI{425}{\km\per\s}, see \autoref{fig:histogram_V_b}.
%the SWS1 mean raises from 435 to 455~km/s in 3hmax data...\\
\begin{figure}[htb]
	\begin{floatrow}
		\ffigbox{
			\includegraphics[width=0.5\Xhsize]{figures_of_mine/chapter2/histogram_V_b.pdf}
		}{
			\caption{Solar wind velocity frequency distributions for 3-hour mean (black), maximum (red), and maximum of the CME part (green). The minutely OMNI data from the period 1981--2016 is used.}
			\label{fig:histogram_V_b}
		}
		\ffigbox{
			\includegraphics[width=\Xhsize]{figures_of_mine/chapter2/cc_lag_sws_d.pdf}
		}{
			\caption{\Kp{}-velocity correlation coefficients for time shifts in the range \numrange{-24}{48}~hours. The correlations are plotted for the whole solar wind data (solid), for solar wind streams without CMEs (dashed), and for CMEs only (dotted). The data used is the 3-hour maximum of the minutely high resolution OMNI data from the period 1981--2016.}
			\label{fig:cc_lag_sws_d}
		}
	\end{floatrow}
\end{figure}
The CME part and solar wind stream part of the data can be examined separately, filtering the CME related periods using the SWS list. Their frequency distributions show that the CME share is rising in faster solar wind until eventually in the region above about \SI{900}{\km\per\s} there only exist CMEs, see \autoref{fig:histogram_V_b}.

The CME part of the data is correlated with the \Kp~index independently from the remaining solar wind streams, see \autoref{fig:cc_lag_sws_d}. The correlation for CME related data is lower than that for the regular solar wind (all data). Its maximal correlation coefficient with a value of 0.51 is without time shift, see Table~\ref{tab:correlation_coefficients_kpvsv}.
\begin{table}
	\caption{Time lags with the highest correlation coefficients for the \Kp{}-velocity relation for the whole OMNI data, for stream data, and for CME data. The values are based on the 3-hour maximum of the minutely high resolution OMNI data from the time period 1981--2016.}
	\label{tab:correlation_coefficients_kpvsv}
	\centering
	\begin{tabular}{lcc}
		\hline\hline
		Data	&Time lag [hours]	&Correlation coefficient\\
		\hline
		All data	&6	&0.622\\
		Streams	&9	&0.661\\
		CMEs	&0	&0.511\\
		\hline
	\end{tabular}
\end{table}
% the best lag times are:\\
% sws: +6 h\\
% sws1: 0 h\\
% sws23: +9 h\\
% 
% correlation coefficients\\
% SWS1\\
% 0	0.511093\\
% SWS23\\
% lag	cross	auto x	auto y\\
% -3	0.660694\\
% 0	0.620113\\
% SWS\\
% lag	cross	auto x	auto y\\
% -2	0.621539\\
% 0	0.595784\\
The solar wind streams show a higher correlation with \Kp{} and the maximal coefficient of 0.66 has a positive time shift of 9~hours, that is, the \Kp~index forecasts the velocity of solar wind streams 9~hours in advance.

The positive time shift can be explained with the occurence of interaction regions followed by high speed streams (HSS). When a slow solar wind stream is followed by a fast one, the compression at their interface leads to enhanced solar wind densities and magnetic field strengths. The peak velocity of a HSS naturally appears after the interaction region. Therefore the \Kp-impact from the enhanced magnetic field is correlated with the higher velocity of the HSS, yielding the observed positive time shift.

In fact, \citet{Machol2013} even proposed a linear function of the \Kp~index as a best proxy for corrupted real-time velocity measurements made by the Advanced Composition Explorer (ACE) spacecraft.
 

\subsection{Functional dependency for CME velocity}
The general \Kp-velocity dependency in the solar wind is apparent from the tilt of its distribution, see top panel of \autoref{fig:Kp_2dhistogram_V_sws_d}.
\begin{figure}
	\fcapside[\FBwidth]{
		\includegraphics[width=0.6\textwidth]{figures_of_mine/chapter2/Kp_2dhistogram_V_sws_d.pdf}
	}{
		\caption{\Kp-velocity distributions for all solar wind data, for solar wind streams and for CMEs. The data used is the 3"~hour maximum of the minutely high resolution OMNI data from the time period 1981--2016. The SWS list from \citet{Richardson2012} is used for the separation between CME and stream data. The bin size is \SI{10}{\km\per\s} and \SI{1/3}{\Kp}~unit respectively.}
		\label{fig:Kp_2dhistogram_V_sws_d}
	}
\end{figure}
The distribution is inclined to positive values but very broad. The comparison with the filtered data shows that \Kp{} values \num{>6} and velocities \SI{>850}{\km\per\s} are almost always associated with CME related periods, see middle and bottom panel of \autoref{fig:Kp_2dhistogram_V_sws_d}.

In order to find a functional relation for the mean \Kp{} value, I look at the relative frequencies per velocity interval, which are plotted in the bottom panel of \autoref{fig:Kp_2dhistogram_V_sws1_c2}.
\begin{figure}
	\fcapside[\FBwidth]{
		\includegraphics[width=0.6\textwidth]{figures_of_mine/chapter2/Kp_2dhistogram_V_sws1_c2.pdf}
	}{
		\caption{CME part of the \Kp-velocity distribution (same as third panel of \autoref{fig:Kp_2dhistogram_V_sws_d}) and its relative distribution per velocity interval with the mean \Kp{} values (solid line) and their mean absolute deviation (dotted lines). The bin size is \SI{10}{\km\per\s} and \SI{1/3}{\Kp}~unit respectively.}
		\label{fig:Kp_2dhistogram_V_sws1_c2}
	}
\end{figure}
The mean \Kp{} value seems to scale almost linear with the solar wind velocity. The MAD of the mean has a mean size of about \SI{1.1}{\Kp~units}.
%MAD: 3.338/3 = 1.113 Kp units

%determine fitting function
Again, as the \Kp~index has a quasi-logarithmic scaling, a logarithmic function is the obvious choice for the fitting process. Thus, the logarithmic function
\begin{align}
	f(x) = a \cdot \ln\left(x + x'\right) + y'	\label{eq:log_offset_fit_function}
\end{align}
is used for the fit, with the scaling factor $a$, the location parameter $x'$, and the vertical shifting parameter $y'$. The resulting fit is plotted in \autoref{fig:Kp_2dhistogram_V_sws1_fit_e} and its parameters are $a = \num{10.6(34)}$, $x' = \num{8.1(43)e2}$, and $y' = \num{-73(28)}$, with the velocity in units of [\si{\km\per\s}]. The MAD is about 1.1~\Kp{} units.
%10.6075 (3.4)\\
%-73.1694 (28.)\\
%806.943 (430)\\
\begin{figure}
	\fcapside[\FBwidth]{
		\includegraphics[width=0.6\textwidth]{figures_of_mine/chapter2/Kp_2dhistogram_V_sws1_fit_e.pdf}
	}{
		\caption{Mean \Kp{} values (+) and MAD values (dotted lines) per velocity interval for the CME part of the data. The error bars represent the relative data count. The logarithmic fit (dashed line) is plotted with a mean MAD band (shaded area). The function (\ref{eq:log_offset_fit_function}) is used for the weighted fit.}
		\label{fig:Kp_2dhistogram_V_sws1_fit_e}
	}
\end{figure}
This leads to the CME dependency function
\begin{align}
	\Kp_\text{CME}(v) = 10.6 \cdot \ln(v + 810) - 73	\,,	\label{eq:kpvsv_CME_dependency_function}
\end{align}
which can be used to forecast the \Kp{}~index from the estimated CME arrival velocity.\\


solar wind stream example in \autoref{fig:example_sw_plot_v_c_2013-5-1_65}\\
\begin{figure}[htb]
	\centering
	\includegraphics[width=\textwidth]{figures_of_mine/chapter2/example_sw_plot_v_c_2013-5-1_65.pdf}
	\caption[\lofimage{figures_of_mine/chapter2/example_sw_plot_v_c_2013-5-1_65.pdf}]
	{Solar wind measurements, official \Kp{}~index, and estimated \Kp{} for the time period 1~May to 5~July 2013. The solar wind parameters are the magnetic field strength and the velocity. I also plotted the velocity's 3"~hour maximum for illustration. The solar wind data are from the minutely OMNI data set. The official \Kp{}~index is obtained from the GFZ~Potsdam. I plotted the \Kp{} estimate depending on whether the period is flagged as a CME in the SWS list (red line in top panel) or not, that is, via \autoref{eq:kpvsv_CME_dependency_function} or \autoref{eq:kpvsv_stream_dependency_function}. The orange band indicates the mean MAD.}
	\label{fig:example_sw_plot_v_c_2013-5-1_65}
	\addtocontents{lof}{\smallskip\protect\center I created the figure myself.\medskip}
\end{figure}


It becomes obvious how important the influence of the magnetic field z"~component is, when looking at the same example CME from \autoref{fig:example_sw_plot_vBz_b_2013-6-26_6}. I plot its in-situ magnetic field strength and velocity in \autoref{fig:example_sw_plot_v_b_2013-6-26_6} together with the official \Kp{}~index and the derived \Kp{} estimate.
\begin{figure}[htb]
	\centering
	\includegraphics[width=\textwidth]{figures_of_mine/chapter2/example_sw_plot_v_b_2013-6-26_6.pdf}
	\caption[\lofimage{figures_of_mine/chapter2/example_sw_plot_v_b_2013-6-26_6.pdf}]
	{Solar wind measurements, official \Kp{}~index, and estimated \Kp{} for the time period 26~June to 2~July 2013. The solar wind parameters are the magnetic field strength and the velocity. I also plotted the velocity's 3"~hour maximum for illustration. The solar wind data are from the minutely OMNI data set. The official \Kp{}~index is obtained from the GFZ~Potsdam. I plot the \Kp{} estimate depending on whether the period is flagged as a CME in the SWS list (red line in top panel) or not, that is, via \autoref{eq:kpvsv_CME_dependency_function} or \autoref{eq:kpvsv_stream_dependency_function}. The orange band indicates the mean MAD.}
	\label{fig:example_sw_plot_v_b_2013-6-26_6}
	\addtocontents{lof}{\smallskip\protect\center I created the figure myself.\medskip}
\end{figure}
The latter do not coincide well during this period. The \Kp{} estimate performs okay during the initial sheath region and around the peak of the trailing HSS. The rarefaction regions of declining velocity at the start and end of the considered interval are overestimated, whereas the MC and the trailing compression region are underestimated.\\


\subsection{Functional dependency for stream velocity}
The procedure in this section is similar to that in the previous section. The correlation coefficient is higher for solar wind stream velocities when the data is shifted by 9~hours, see \autoref{fig:cc_lag_sws_d}.
I use the shifted data and look at the relative frequencies per velocity interval in order to find a functional dependency for the mean \Kp{} value, see bottom panel of \autoref{fig:Kp_2dhistogram_V_sws23_d}.
\begin{figure}
	\fcapside[\FBwidth]{
		\includegraphics[width=0.6\textwidth]{figures_of_mine/chapter2/Kp_2dhistogram_V_sws23_d.pdf}
	}{
		\caption{Stream part of the \Kp-velocity distribution (similar to second panel of \autoref{fig:Kp_2dhistogram_V_sws_d}, but with the data shifted by 9~hours) and its relative distribution per velocity interval with the mean \Kp{} values (solid line) and their mean absolute deviation (dotted lines). The bin size is \SI{10}{\km\per\s} and \SI{1/3}{\Kp}~unit respectively.}
		\label{fig:Kp_2dhistogram_V_sws23_d}
	}
\end{figure}
Again, the mean \Kp{} value scales almost linear with the velocity. The MAD of the mean has a mean size of about \SI{0.7}{\Kp~units}.
%MAD: 2.226/3 = 0.742 Kp units
%MAD: 2.332/3 = 0.777 Kp units for 300--900km/s
%MAD: 2.389/3 = 0.796 Kp units for 350--900km/s
%MAD: 2.454/3 = 0.818 Kp units for 350--850km/s

%determine fitting function
Again, I use the logarithmic function (\ref{eq:log_offset_fit_function}) for the fitting process. The resulting fit is plotted in \autoref{fig:Kp_2dhistogram_V_sws23_fit_e} and the fit parameters are $a = \num{5.88(38)}$, $x' = \num{2.99(49)e2}$, and $y' = \num{-3.70(29)e1}$, with the velocity in units of [\si{\km\per\s}]. The MAD is about \SI{0.7}{\Kp}~units.
%a1 = 5.88(38)
%b1 = -37.0(29)
%x1 = 299(49)
\begin{figure}
	\fcapside[\FBwidth]{
		\includegraphics[width=0.6\textwidth]{figures_of_mine/chapter2/Kp_2dhistogram_V_sws23_fit_e.pdf}
	}{
		\caption{Mean \Kp{} values (+) and MAD values (dotted lines) per velocity interval for the stream part of the data, shifted by 9~hours. The error bars represent the relative data count. The logarithmic fit (dashed line) is plotted with a mean MAD band (shaded area). The function (\ref{eq:log_offset_fit_function}) is used for the weighted fit.}
		\label{fig:Kp_2dhistogram_V_sws23_fit_e}
	}
\end{figure}
This leads to the solar wind stream dependency function
\begin{align}
	\Kp_\text{Stream}(v) = 5.88 \cdot \ln(v + 299) - 37.0	\,,	\label{eq:kpvsv_stream_dependency_function}
\end{align}
which can be used to forecast the \Kp{}~index from the estimated stream velocity, e.g., obtained from remote coronal hole analyses.


\section{Discussion}
The following results/relations are obtained from the analyses:
\begin{itemize*}
	\item solar activity: \Kp{}-SSN relation with an error of about 1/3~\Kp{}~unit
	\item seasonal variations of up to 4/3~\Kp{}~units
	\item solar wind nowcast: \Kp-\vBz{} relation (average and worst case)
	\item CME forecast: \Kp-velocity relation (average and worst case)
	\item stream forecast: \Kp-velocity relation (average and worst case)
\end{itemize*}

% prediction model performances
The derived empirical relations are chosen for their high correlation coefficients. However, the correlation coefficient itself does not represent the performance of a model as the coefficient depends highly on the scatter of the underlying distribution. Therefore \citet{Wing2005} advise to additionally provide scatterplots and skill scores for the evaluation of predictive models. In the following, I present the forecast errors and determine the true skill statistics as measures for the prediction performance of the empirical models.\\

% persistence
The quality of predictions can be assessed by how they compare to a simple persistence model \citep{Detman1999}. In the case at hand the persistence consists of the official \Kp{} value from the previous 3"~hour interval.
% prediction performance
In order to evaluate the models' prediction performances, their forecast errors are calculated. Forecast errors are the difference between the predicted and the actual value. The resulting performances of the persistence and the three models and the corresponding standard deviations are displayed in \autoref{fig:model_performance_c}.
\begin{figure}
	\fcapside[\FBwidth]{
		\includegraphics[width=0.6\textwidth]{figures_of_mine/chapter2/model_performance_c.pdf}
	}{
		\caption[\lofimage{figures_of_mine/chapter2/model_performance_c.pdf}]
		{Prediction performance of \Kp{} persistence and the three derived empirical \Kp{} relations. The forecast errors are the difference between the predicted and the actual value and are calculated from the same data and time range as the relations themselves. The errorbars denote one positive and negative standard deviation. Perfect predictions are indicated by the gray diagonal lines.}
		\label{fig:model_performance_c}
	}
\end{figure}
The persistence performance is obtained from the complete \Kp{} time span 1981--2016. It does fairly well, as its forecast error is less than 0.7~\Kp{} units up to a \Kp{} of 5.7. At and above a \Kp{} of 6 the error is underestimated with a maximum deviation of 1.3~\Kp{} units.

The derived empirical functions do not cover the whole \Kp{} range from 0 to 9 within the distributions of observed values. Only the \Kp{} ranges with more than one observed data point are considered for deriving the relations' prediction performances. This excludes the \Kp{}~9 values and in case of the velocity relation for streams also the values \Kp{}~7.7 and above.

% vBz performance
The derived \vBz{} function (\ref{eq:kpvsvbz_dependency_function_negative} + \ref{eq:kpvsvbz_dependency_function_positive}) does only coincide with significant data in the \Kp{} range 1 to 8.7. Especially because the minimum of the \vBz{} function is larger than 0.7~\Kp{}, as can be seen from \autoref{fig:Kp_2dhistogram_VBzgsm_sws_fit_e}. The model performs reasonably well at and below a \Kp{} of 5 with forecast errors smaller than 1~\Kp~unit. Larger \Kp~values are underestimated and deviate up to 1.7~\Kp~units from the predicted values.

% v CME performance
In the case of the velocity relation for CMEs, the derived function (\ref{eq:kpvsv_CME_dependency_function}) does only coincide with significant data in the \Kp{} range 1.3 to 6.3, as can be seen from \autoref{fig:Kp_2dhistogram_V_sws1_fit_e}. The magnitude is overestimated below a \Kp{} of 3 and underestimated above. In the range between a \Kp{} of 1 and 5 the forecast errors are smaller than 1.3~\Kp~units. Above, the error rises up to 4~\Kp~units at a \Kp{} of 8.7.

% v stream performance
For the velocity relation for streams, the derived function (\ref{eq:kpvsv_stream_dependency_function}) does only coincide with significant data in the \Kp{} range 0.3 to 4.3, as can be seen from \autoref{fig:Kp_2dhistogram_V_sws23_fit_e}. The magnitude is overestimated below a \Kp{} of 2 and underestimated above. Throughout the valid range the forecast errors are smaller than 1.3~\Kp~units. Above, the error rises up to 4~\Kp~units at a \Kp{} of 7.3.

The model based on the \vBz{} relation is more accurate than those based purely on the velocity. This is expected, however, the \vBz{} relation is still eclipsed by the persistence model.\\

% true skill score
In order to further test the derived models for their predictive value, I derive their true skill statistic (TSS) -- a common tool for forecast verification. The TSS is a skill score based on the contingency table that categorizes forecasted and observed events. The score is the difference between the forecast hit rate and the false alarm rate. Its range is between -1 and 1, where 1 indicates an ideal prediction and 0 a random prediction. I give a more detailed description on the TSS in the appendix \autoref{sec:true_skill_statistic}. This method is the prevalent form of forecast verification in \Kp{} models \citep{Detman1999,Wing2005,Savani2017}. It is defined that each single 3"~hour \Kp{} interval represents an event and a hit occurs when both the forecasted and observed \Kp{} exceed a specified threshold. I adopt these criteria to make the results comparable. Therefore the TSS is derived as a function of \Kp{} threshold -- the results for persistence and the three models are plotted in \autoref{fig:true_skill_score}.
\begin{figure}
	\fcapside[\FBwidth]{
		\includegraphics[width=0.6\textwidth]{figures_of_mine/chapter2/true_skill_score.pdf}
	}{
		\caption{True skill scores for the \Kp{} persistence and the three prediction relations as a function of \Kp{} threshold.}
		\label{fig:true_skill_score}
	}
\end{figure}

The E"~field relation reaches its peak at a \Kp{} of 2 with a TSS of 0.63, it then decreases to a minimum at \Kp{} 6 with a TSS of 0.26. To higher \Kp{} values, the TSS increases again and reaches 0.47 at a \Kp{} of 8.7. This increase is due to the dominant number of correct null forecasts that make the TSS approach the hit rate \citep{Doswell1990}. It is a bias inherent to the TSS in case of rare event forecasting. Both velocity relations have throughout their \Kp{] ranges a significantly lower TSS. The CME relation shows a peak TSS of 0.32 at a \Kp{} of 3 and the stream relation a peak TSS of 0.48 at a \Kp{} of 2. Thus, the TSS gain from the additional $B_\text{z}$ component is about 1.5--2~\Kp{} units.\\

The persistence forecast clearly outperforms the other prediction models, especially in the high \Kp{} range. However, as \citet{Detman1999} stated, it has no warning value because it can only predict a high \Kp{} value after a high \Kp{} value was already observed. Thus, the persistence always misses the onset of a geomagnetic storm, whereas solar wind based models can provide at least a nowcast if not actually a short lead time from the distance of the in-situ measurement location at L1.\\

% statistics summary table
In order to compare their \Kp{} forecast skill metrics with other models, \citet{Savani2017} use a threshold of $\Kp{} \geq 5$. They choose this value because of the NOAA/SWPC geomagnetic storm scale (G"~scale) definition: The G"~scale starts with G1, which translates to a \Kp{} of 5. I stick to this definition and provide a statistics summary of the \Kp{} persistence and the three derived models in \autoref{tab:model_statistics_table}.
\begin{table}[htb]
	\caption{Statistics of the different prediction models. The metrics are calculated for a threshold hit criteria of $\Kp{} \geq 5$.}
	\label{tab:model_statistics_table}
	\centering
	\begin{tabular}{lcccc}
		\hline\hline
		\multirow{2}{*}{Parameter}	&\Kp{} persistence	&E-field	&CME velocity 	&Stream velocity\\
			&$\Kp(t-1)$	&$\Kp(\text{\vBz})$	&$\Kp(v_\text{CME})$	&$\Kp(v_\text{Streams})$\\
		\hline
		Time shift [hours]	&3	&0	&0	&9\\
		Event count	&\num{105192}	&\num{79276}	&\num{12116}	&\num{65774}\\
		Correlation coefficient	&0.81	&\num{-0.72}	&0.51	&0.66\\
		\hline
		Proportion correct	&0.96	&0.97	&0.88	&0.98\\
		Hit rate	&0.55	&0.33	&0.13	&0\\
		False alarm rate	&0.02	&0	&0.01	&0\\
% 		FAR	&0.45	&0.24	&0.35	&1\\
% 		DFR	&0.02	&0.03	&0.11	&0.02\\
		True skill score	&0.53	&0.33	&0.12	&0\\
		Valid \Kp{} range	&\numrange{0.3}{8.7}	&\numrange{1.0}{8.7}	&\numrange{1.3}{6.3}	&\numrange{0.3}{4.3}\\
		\hline
	\end{tabular}
\end{table}
Obviously the defined \Kp{} threshold is of little use for the stream velocity relation as by this definition geomagnetic storms cannot be predicted.\\

% comparison
The E"~field model has a TSS of 0.33 at a \Kp{} of 5, this can be compared with the Wing APL models \citep[Fig.~13]{Wing2005}. The Wing APL model~3 predicts \Kp{} 1~hour in advance and uses the solar wind parameters $n, v_\text{x}, |\vect{B}|$, and $B_\text{z}$ as input sources for a neural network based model. The additionally considered parameters density and magnetic field strength boost the corresponding TSS to about 0.7. The Wing APL model~3 incorporates \Kp{} nowcast values in addition to the solar wind parameters and achieves this way a TSS of about 0.8.\\

BSS model for eight CME events has $\mathit{TSS} = 0.34$ \citep[Tab.~3]{Savani2017}\\


% Costello model (neural network, input: sw): about 0.45\\
% Wing APL model 1 (1 h, neural network, input: Kp nowcast + sw): about 0.8\\
% Wing APL model 2 (4 h, neural network, input: Kp nowcast + sw): about 0.7\\
% Wing APL model 3 (1 h, neural network, input: sw): about 0.7\\
% BSS model for eight CME events has $\mathit{TSS} = 0.34$ \citep[Tab.~3]{Savani2017}\\


% extreme CME speeds
CMEs can be faster than the maximal velocities included in the OMNI data. Instrumental effects (specify...) lead to data gaps during these periods. Yet, CME speeds of up to XX~km/s at \SI{1}{\au} were observed (cite?). According to \autoref{eq:kpvsv_CME_dependency_function}, a CME speed of \SI{2000}{\km\per\s} at L1 would lead on average to a theoretical \Kp{} of 12.2, however, the \Kp{} scale is capped at 9o. A \Kp{} of 9o is reached already at a velocity of \SI{1489}{\km\per\s}, see \autoref{fig:Kp_2dhistogram_V_sws123_fit_f}.
% 2000 km/s -> 12.17~Kp
\begin{figure}
	\fcapside[\FBwidth]{
		\includegraphics[width=0.6\textwidth]{figures_of_mine/chapter2/Kp_2dhistogram_V_sws123_fit_f.pdf}
	}{
		\caption{Logarithmic fit curves for CMEs (dashed line) and streams (dotted line) with their corresponding mean MAD bands (shaded areas).}
		\label{fig:Kp_2dhistogram_V_sws123_fit_f}
	}
\end{figure}

The maximal velocity of regular solar wind is limited by the coronal temperatures \citep{Parker1958} and is observed to be around \SI{900}{\km\per\s}. That is why streams on average provoke \Kp{} values below 5o.\\

comparison:\\
\Kp-velocity correlation\\
similar to \citet{Elliott2013}; different data time period, resolution and averaging method (3-hour maximum of 1~min data)\\
see Akasofu1981 p.~126, table\\


It is obvious that the derived relations are not suited to be used inversely, i.e., to deduce solar wind properties from existing \Kp{}~data.\\


\section{Applications and outlook}

'scientists develop an advanced prototype space weather warning system to ensure the operation of telecommunication and navigation systems on Earth to the threat of solar storms' from AFFECTS\\

Prototype/precursor relations are integrated into applications developed within the Advanced Forecast For Ensuring Communications Through Space (\mbox{AFFECTS}) project which ran from 2011 to 2013. The following services, accessible via the \mbox{AFFECTS} website\footnote{AFFECTS website: \urlfoot{http://www.affects-fp7.eu/services/}}, contain results from this \Kp{} study:
\begin{itemize*}
	\item Real-time plot: \href{http://www.affects-fp7.eu/rssfeeds/ace_ap_forecast_plot/ace_realtime_ap_CH_GFT_plot.png}{Solar Wind and \Kp{} forecast plot} -- DSCOVR real-time solar wind and \Kp{} forecast plot. The \Kp{} is estimated from the solar wind relation.
	\item RSS feed: \href{http://www.affects-fp7.eu/rssfeeds/rssfeed_kp/rssfeed_kp.xml}{L1 Kp Alert} -- threshold based RSS feed that gets triggered when a specified \Kp{} is reached.
	\item RSS feed: \href{http://www.affects-fp7.eu/rssfeeds/rssfeed_gnss/rssfeed_gnss.xml}{L1 GNSS Alert} -- threshold based RSS feed that gets triggered when a specified GNSS error is reached... The value is derived from the \Kp{} estimate.
	\item RSS feed: \href{http://www.affects-fp7.eu/rssfeeds/rssfeed_aurora/rssfeed_aurora.xml}{L1 Aurora Alert} -- threshold based RSS feed that gets triggered when a specified auroral location is reached... The value is derived from the \Kp{} estimate.
\end{itemize*}

note: repair RSS feeds before submitting!\\

\noindent Further applications of resulting \Kp{}-relations:
\begin{itemize*}
	\item CME \Kp{} impact (part of UGOE DDC)
	\item iPhone app L1 Alerts (Solar wind latest 2-hour extreme values and derived forecast values): \urltext{http://www.affects-fp7.eu/app-services/L1-Alerts/dataL1Alerts.txt}, \urltext{https://itunes.apple.com/au/app/affects/id893579846}
	\item Android app L1 Alerts... \urltext{https://play.google.com/store/apps/details?id=com.afects.forecasts}
	\item has SW-Display Kp forecast??
\end{itemize*}

outlook:\\
Separate structures such as CIRs and HCSs for their \Kp-impact separately...\\

All provided web links in this work were existent in [date].\\


CME Kp impact as part of DDC\\
- \textit{Kp} nowcast with L1 solar wind measurements (L1 alerts, disseminated as RSS feeds; integrated in smartphone app and space weather display)\\
- Forecast of the possible CME impact on the Earth's magnetosphere (\textit{Kp} index) from the predicted CME arrival velocity (integrated in UGOE CME forecast chain (aka DDC))\\

In the present study, the solar activity is neglected for deriving the empirical solar wind-\Kp{} relations. It would be worth, examining their dependency on the solar cycle, especially as \citet{Wing2005} note that the predictability of \Kp{} slightly scales with solar activity.\\


\section{notes...}

What kind of solar wind structures create the individual regions in this distribution? (B-V-Kp circle plot)\\
What is their individual contribution to the Kp ranges (e.g. high Kp: CMEs 70\% and CIRs 30\%)?\\

ACE solar wind time series and event list (ACE OPTIMAP ``Zeitreihe''-events)\\


How can the impact field strength of CMEs be forecasted (V->B correlation for CMEs)?\\
Internal solar wind correlations: B-V correlation\\
ACE MAGSWE 64~s data -> yearly overlay plot\\


rt data errors/gaps... vs science data (see paper Kp as V replacement)\\
DSCOVR as replacement was launched on 11~Februar 2015. It is NOAA's SWPC real-time solar wind prime source since 27 July 2016.\footnote{\urlfoot{http://www.swpc.noaa.gov/products/real-time-solar wind}}\\


