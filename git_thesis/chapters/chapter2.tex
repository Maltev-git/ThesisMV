
\chapter{Solar wind and CME influence on geomagnetic activity}
\label{chap:chapter2}

%context
Variations in the Earth's magnetosphere are largely evoked by influence through the solar wind. These magnetospheric disturbances have diverse effects on the terrestrial environment. Especially the effects of severe geomagnetic storms created by coronal mass ejections (CMEs) pose various threats to sensitive technical systems and exposed humans. Thus, the development of quantitative forecasts for magnetospheric impacts caused by solar wind and CMEs is of major importance. The analyses in this chapter are based on my work done for the EU~FP7 project Advanced Forecast For Ensuring Communications Through Space (AFFECTS).

%aims
The goals of this study are to estimate the magnetospheric impact from solar wind and also to predict it for remotely forecasted CMEs and streams in particular. Empirical dependencies between the solar wind and the magnetospheric disturbance index~\Kp{} are presented. These relations allow nowcasts from upstream (L1) solar wind in-situ measurements and allow forecasts from remote observations of the corona. The \Kp{} nowcast is derived via a relation with the solar wind electric field. The \Kp{} forecasts are based on solar wind velocity and split into CME and stream forecasts. The magnetospheric impact of CMEs is estimated solely based on their arrival velocities that may be predicted from coronagraph observations. The prediction of solar wind stream velocities, e.g., obtained from coronal hole observations, enables to estimate their impact as well.

% data
The solar wind data considered in these analyses consists of 35~years (1981--2016) of high-resolution minutely OMNI data, which is composed of multi-spacecraft intercalibrated in-situ measurements from \SI{1}{\au}. I analyze the \Kp{} frequency distributions with respect to the depending parameters E"~field and velocity, derive their mean \Kp{} values and further compile functional dependencies via logarithmic fitting.

%methods
In order to nowcast the \Kp~index from general solar wind conditions, I apply a correlation with the solar wind electric field -- the product of the parameters velocity and magnetic field z-component in GSM coordinates: $E = v B_\text{z}$. A suitable logarithmic function is constructed and its fit to the data results in a functional relation between E"~field and \Kp{}. For the purpose of forecasting the \Kp~index from estimated CME and stream velocities, I furthermore filter the solar wind data, using flagged CME times from the solar wind structures (SWS) list provided by \citet{Richardson2012}. Logarithmic fits to the data yield in separate \Kp{} relations for CMEs and streams.

I evaluate the relations' prediction performance by analyzing their forecast errors and comparing their true skill statistic.

%results
As the obtained functional relations are simple, they cannot compete with current models based on artificial neural networks, \Kp{} persistence, or full-fledged solar wind coupling functions. However, they enable empirical estimations of the mean \Kp{} impact for special forecast situations, that is, \Kp{} can directly be quantified from measured solar activity, in-situ solar wind, and remotely determined CME and stream velocities.

%conclusions

%results
With the results presented in this chapter, I elaborate the step from solar wind properties at Earth to the forecast of the possible impact strength on the terrestrial magnetosphere. I derive empirical correlations and functional dependencies between solar wind properties and the geomagnetic \Kp~index, in order to obtain the capability to nowcast/forecast \Kp{} values.

Remote solar observations provide enhanced forecast lead times of CMEs and streams, however, the benefit comes with major limitations on the predicted magnetic field and plasma parameters.
%CME velocity forecast
The velocity and the direction of CMEs can still be determined in their early near-Sun stages via remote tracking with coronagraph white-light observations. Using these parameters as input for CME propagation models, their possible arrival time and arrival velocity at Earth can be derived, see \autoref{sec:solar_wind_nowcast_and_forecast_to_earth}.
% stream velocity forecast
Similarly, the Earth arrival time and velocity of solar wind streams can be estimated remotely. Images of the solar surface and the corona reveal the distinct sources of solar wind and indicate the emitted type of solar wind and its properties.\\

%%% synopsis
The objectives of the analyses performed and presented in this chapter are to estimate the magnetospheric impact of solar wind and to predict it for CMEs and streams in particular. In \autoref{sec:long_term_variations} I determine the magnitudes of the long-term \Kp{} changes due to solar activity and measure the extent of seasonal variations stemming from the Earth's orbit. In order to nowcast the \Kp{}~index, I quantify the solar wind influence on \Kp{} by deriving a functional relation with the solar wind E"~field in \autoref{sec:relation_between_sw_efield_and_kp}. For the purpose of enabling \Kp{} forecasts from remote observations, I estimate the \Kp{} impact coming from CMEs and streams separately by deriving functional dependencies with their velocities in \autoref{sec:relations_between_cme_stream_v_and_kp}. Finally, the obtained \Kp{} relations are evaluated for their prediction performance in \autoref{sec:prediction_performance}. At the end of this chapter, the results are further discussed, applications are highlighted, and a short outlook is given.


\section{The \Kp{}~index and its long-term variations}
\label{sec:long_term_variations}

% intro
As magnetospheric activity is driven by the approaching solar wind, it also reflects the solar wind's long-term variations. The long-term variations of \Kp{} originate from the change in solar activity and the changes due to the Earth's orbit around the Sun. Here the influence of both effects is quantified.\\

I estimate the long-term variations of the \Kp~index which are contributed by solar activity and seasonal effects. A functional dependency of the average yearly \Kp~index on the yearly sunspot number (SSN) is derived.\\

% data
\Kp{} is designed to measure solar particle radiation by its magnetic effects. I use this magnetospheric disturbance index to correlate it with near-Earth solar wind measurements. More detailed information on the \Kp{}~index can be found in \autoref{sec:kp_index}.\\

The \Kp{} data is obtained from the GFZ~Potsdam, where the index is currently maintained\footnote{GFZ~Potsdam \Kp~index website: \urlfoot{http://www.gfz-potsdam.de/en/kp-index/}}. The data used in this analysis covers the time period 1932--2016, see also \autoref{sec:kp_data}. Its frequency distribution shows that the highest frequencies occur around low \Kp{} values of 1+, see \autoref{fig:Kp_histogram_b}. Going to higher \Kp{} values, the frequencies decline asymptotically towards zero -- a \Kp{} value of 9o occurred only 29 times in this time interval.\\
\begin{figure}[htb]
	\begin{floatrow}
		\ffigbox{
			\includegraphics[width=0.46\textwidth]{figures_of_mine/chapter2/Kp_histogram_b.pdf}
		}{
			\caption{\Kp{} frequency distribution for the time period 1932--2016. The inset shows a zoomed-in view of the high-value tail. The \Kp{} data is obtained from the GFZ~Potsdam.}
			\label{fig:Kp_histogram_b}
		}
		\ffigbox{
			\includegraphics[width=0.95\Xhsize]{figures_of_mine/chapter2/Kp_histogram_yearlySSN_b.pdf}
		}{
			\caption{Yearly \Kp{} frequency distributions during the period 1932--2016, sorted and colored by yearly SSN. All distributions are normed to be of equal area. The \Kp{} data is obtained from the GFZ~Potsdam and the yearly SSN data from the SILSO World Data Center.}
			\label{fig:Kp_histogram_yearlySSN_b}
		}
	\end{floatrow}
\end{figure}

\subsection{Solar activity influence}
Obviously, the general \Kp{} distribution as seen in \autoref{fig:Kp_histogram_b} averages over solar activity. Solar activity is generally tracked with the international sunspot number (SSN). SSN data from the time period 1917--2016 is used in the present analyses. The data is obtained from the International Sunspot Number Monthly Bulletin and online catalog\footnote{WDC-SILSO website: \urlfoot{http://www.sidc.be/silso/}} provided by the World Data Center -- Sunspot Index and Long-term Solar Observations (WDC-SILSO), Solar Influences Data Analysis Center (SIDC), Royal Observatory of Belgium (ROB).

The \Kp{} frequency distributions' shape varies with solar activity (cite?). This is visible in the yearly distributions, sorted and colored by yearly SSN in \autoref{fig:Kp_histogram_yearlySSN_b}.

The distribution's peak position scales with SSN, that is, a high yearly SSN results in a higher abundance of large \Kp{} values as well (cite?).

The time series of yearly average \Kp{} values from the years 1932--2016 shows a solar cycle imprint, see the top graphs in \autoref{fig:yearly_kp-ssn_correlation_c}.
\begin{figure}
	\fcapside[\FBwidth]{
		\includegraphics[width=0.6\textwidth]{figures_of_mine/chapter2/yearly_kp-ssn_correlation_c.pdf}
	}{
		\caption{Yearly \Kp~index distributions (shaded area) with their mean values for the time period 1932--2016 and yearly SSN with cycle number for the time period 1917--2016 (top panels). The Pearson correlation coefficients with the yearly SSN are calculated for time lags back to \num{-15}~years (bottom panel). The \Kp{} data is obtained from the GFZ~Potsdam and the yearly SSN data from the SILSO World Data Center.}
		\label{fig:yearly_kp-ssn_correlation_c}
	}
\end{figure}
The \Kp{} pattern follows the solar cycle minima and maxima as well as the changes in magnitude between solar cycles. The yearly mean \Kp{} shifts about 1~\Kp~unit for both variations separately. As expected, the correlation with solar activity shows an 11-year period, see bottom graph in \autoref{fig:yearly_kp-ssn_correlation_c}. The highest correlation coefficient of 0.60 is found with a time lag of $-1$~year, that is, the yearly average \Kp{} follows the SSN of the previous year.
%Kp-ssn cc: 0.5971

The yearly mean \Kp~index with respect to the 1-year lagged SSN shows a raise with increasing SSN, as seen from \autoref{fig:Kp_SSN_fit_d}.
\begin{figure}
	\fcapside[\FBwidth]{
		\includegraphics[width=0.6\textwidth]{figures_of_mine/chapter2/Kp_SSN_fit_d.pdf}
	}{
		\caption{Yearly mean \Kp~index with respect to 1-year lagged SSN (+) with the weighted logarithmic fit (dashed line). The error bars denote the SSN standard deviation and the relative weight from the yearly data coverage. The shaded area represents the fit error band derived from the estimated standard deviations of the fit parameters. The logarithmic function (\autoref{eq:log_fit_function}) is used for the weighted fit. The yearly \Kp{} mean values are calculated from GFZ~Potsdam data and the yearly SSN is obtained from the SILSO World Data Center.}
		\label{fig:Kp_SSN_fit_d}
	}
\end{figure}
In order to obtain an analytical relation for this dependency, I perform a least-squares regression fit. \Kp{} itself is a quasi-logarithmic index, so it is apparent to use a logarithmic fit function:
\begin{align}
	f(x) = a \cdot \ln(x) + b	\,.	\label{eq:log_fit_function}
\end{align}
The resulting fit parameters are $a = 0.281(43)$ and $b = 1.05(19)$; the numbers in parentheses are the estimated standard deviations.
They lead to the relation
\begin{align}
	\Kp(ssn) = 0.28 \cdot \ln(ssn) + 1.1	\,,	\label{eq:kp_ssn_relation}
\end{align}
% log fit parameters:
% a 0.281126         +/- 0.04267
% b 1.04923          +/- 0.19
which is plotted in \autoref{fig:Kp_SSN_fit_d}. This means that for an average yearly SSN of 1 the mean \Kp{} is $1.05(20)$ and for a SSN of 300 it is $2.65(31)$. The numbers in parentheses are the errors on the corresponding last digits of the quoted value. They are calculated via error propagation from the estimated standard deviations of the fit parameters.
% SSN	Kp	Kp_err
% 1	1.0492	0.189
% 10	1.6965	0.213
% 50	2.1489	0.252
% 100	2.3438	0.273
% 200	2.5387	0.295
% 300	2.6527	0.308

This relation is a practical resource for estimating the yearly average \Kp~index from the last year's average SSN. In 2017 the yearly SSN was \num{21.7(25)} -- applying this into the relation gives an average \Kp{} of \num{1.91} for 2018.

\subsection{Seasonal variations}
On top of the yearly variations, seasonal variations exist in the magnetospheric disturbances as well. In the months May--August the \Kp{} peak frequency is higher than in the remaining months of the year, whereas in March/April and September/October \Kp{} values larger than 3 are more abundant. This is apparent from looking at the monthly \Kp{} frequency distributions plotted in \autoref{fig:Kp_histogram_monthly}.
\begin{figure}[htb]
	\begin{floatrow}
		\ffigbox{
			\includegraphics[width=0.46\textwidth]{figures_of_mine/chapter2/Kp_histogram_monthly.pdf}
		}{
			\caption{Average monthly \Kp{} frequency distributions of the time period 1932--2016, colored by month of the year. The \Kp{} data is obtained from the GFZ~Potsdam.}
			\label{fig:Kp_histogram_monthly}
		}
		\ffigbox{
			\includegraphics[width=0.46\textwidth]{figures_of_mine/chapter2/Kp_seasonal_e.pdf}
		}{
			\caption{\Kp{} frequency distributions by month for the time period 1932--2016 with median (white dashed) and quartile isolines (white dotted). The other dotted isolines mark the upper eighth, 16th, 32nd, and 64th parts. The bin size is 1~month and \SI{1/3}{\Kp}~unit respectively.}
			\label{fig:Kp_seasonal_e}
		}
	\end{floatrow}
\end{figure}
These \Kp{} changes arise from seasonal variations of the solar wind parameters at Earth, which stem from Earth's yearly changes in orbital distance and heliographic latitude. The Earth's rotation axis tilt adds another seasonal effect, the tilt changes the direction of the Earth's magnetic dipole axis to the Sun during the year.

Earth's distance to the Sun varies over the course of a year by \SI{+-1.67}{\percent}, see Appendix~\ref{sec:sun_earth_orbit_geometry}. The solar wind parameters scale via power-law dependencies with solar distance, as it is described in the following \autoref{chap:empirical_solar_wind_model_for_the_inner_heliosphere} and accordingly in \citet{Venzmer2018}. For example, the solar wind proton density scales with about $r^{-2}$, this leads to a yearly variation in density of about \SI{6.7}{\percent}. These yearly solar wind variations have direct influence on the \Kp{}~index.

The Sun's rotation axis tilt angle to the ecliptic is \SI{+-7.25}{\degree} and that for Earth is \SI{+-23.44}{\degree}, see also Appendix~\ref{sec:sun_earth_orbit_geometry}. The solar wind magnetic field strength varies with heliographic latitude (cite?). The solar wind influence on the \Kp{}~index depends on its coupling efficiency with the magnetosphere. Furthermore, the rate of magnetic reconnection between solar wind and the Earth's magnetosphere depends on both fields' orientation to each other (parallel/antiparallel). Additionally, the tilt of the magnetic dipole axis to the rotation axis -- only a few degrees for the Sun during cycle minima and extreme during solar maxima; and about \SI{10}{\degree} for the Earth -- complicates this system even more.
%geomagnetic dipole tilt: Planetary fact sheet: 11.2 (Model GSFC-12/83)
%variation range in the time period 1930--2016: 9.6--11.5° (http://wdc.kugi.kyoto-u.ac.jp/poles/polesexp.html)

So the \Kp{} variation effects originate from the seasonal change in the solar tilt, the Earth's tilt, and the Earth's solar distance. Thorough analyses of the seasonal variations were already early performed by \citet{Cortie1912} (more cites?). Thus, I just quantify the bulk magnitude of these effects in order to consider them as relative uncertainties when using the solar activity relation (\ref{eq:kp_ssn_relation}). Looking at the \Kp{} frequency distributions by month -- seen in \autoref{fig:Kp_seasonal_e} -- it is apparent that for high values ($\Kp{} > 4$), there exist yearly frequency maxima at the equinoxes and frequency minima at the solstices, as was early described by \citet{Cortie1912}. It is evident from the plotted quantiles that this semiannual variation amounts to $1/3$~\Kp~unit at small \Kp~values and up to 4/3~\Kp~units at higher \Kp~values.


\section{Relation between solar wind electric field and \Kp~index}
\label{sec:relation_between_sw_efield_and_kp}

\Kp{} nowcast from in-situ solar wind measurements at L1. Real-time solar wind make reliable measurements of all solar wind parameters available. It has a short lead-time and can therefore provide a nowcast.\\

The coupling between the solar wind and the magnetosphere is governed by reconnection and compression of the magnetic field lines, as described in \autoref{sec:solar_wind_coupling_mechanisms}. There exist quite a few coupling functions relating solar wind quantities with geomagnetic activity, see \autoref{sec:coupling_functions}. However, in this work I settle and work with the solar wind electric field as the coupling function. The reasons for using the electric field are that it is a simple coupling function and still reaches a high correlation...\\

% electric field
see \autoref{eq:coupling_vxBz}\\

The solar wind electric field y"~component $E_\text{y}$ approximates $\vect{E}$ under special circumstances, see derivation in \autoref{sec:electric_field_at_the_magnetopause}. It is the product of the radial proton velocity $v_\text{x}$ and the magnetic field z"~component $B_\text{z}$:
\begin{align}
	E_\text{y} = -v_\text{x} \, B_\text{z}\,.
\end{align}
If not specified otherwise, $B_\text{z}$ is always meant to be in GSM coordinates hereafter.\\


%velocity relation
The solar wind velocity sticks mostly to its radial flow direction, that is, it rarely deviates up to \SI{0.0}{\degree} (correct and cite...). Thus, the absolute flow speed $v$ can be used instead of the vector component.\\

argue for \vBz:\\
vs vBs vs vBT (Newell2007)\\
large northward fields still contribute to geomagnetic activity.\\
3hmin(\vBz) performs in rank correlation slightly better than the sophisticated Newell formula. really?\\


\subsection{Data correlation}
\label{sec:data_correlation}
%determine data basis
The \Kp{} time series started in 1932 when there existed no spacecraft to measure solar wind in~situ. Thus, the maximal surveyed time range is defined by the available in-situ solar wind data.\\

Savani2017:\\
``Although Kp is defined for 3 h periods, predicted variations in the IMF are made on shorter time scales and can influence the level of geomagnetic activity (e.g., geomagnetically induced currents). Hence, averaging the IMF over intervals that match those of Kp may suppress features that are important drivers of geomagnetic activity. Nevertheless, for the purpose of this paper, we choose to average the Kp estimate from the field vectors predicted by the BSS model over the same 3 h intervals as the Kp values, as this currently represents the service provided by NOAA/SWPC.''\\

Commonly the \Kp{} is estimated from 3-hour and 1-hour averages \citep{Savani2017} (more cites...).\\

%argue for averaging method
The \Kp{}~index represents maximal variations within 3-hour time intervals (see \autoref{sec:kp_index}). Any solar wind parameter that will be correlated with \Kp{}, obviously also has to have the same time resolution. In addition to adapting the time resolution, it has to be considered by which means this should be done. Simple 3-hour average values are expected to have a weaker correlation than the solar wind parameter's 3-hourly maximal variation.
%argue for high resolution, deliberate between hourly and minutely data
It is self-evident that the 3"~hour maximal variations are higher when using high resolution data. Thus, to be able to correlate \Kp{} with solar wind data in a proper way, high resolution data, that is, much shorter than the 3"~hour resolution, are needed to determine the maximal solar wind variations within each 3"~hour interval.\\

The OMNI data collection constitutes the longest continuous solar wind measurements made at \SI{1}{\au}. There exist two OMNI data sets with different time resolution -- the hourly version extends back to 1963 and the minutely version extends back to 1981. Although it is shorter, I choose to apply the minutely data, in order to benefit from the higher resolution. Thus, the work presented in this chapter is based on the minutely OMNI data set with a time duration spanning from 1981 until end of 2016.\\

vBz data...\\
The reduction to 3"~hour minimum values shifts the \vBz{} frequency distribution asymmetrically to negative values, whereas the averaged data is scattered around zero, see \autoref{fig:histogram_VBzgsm}.\\
\begin{figure}[htb]
	\begin{floatrow}
		\ffigbox{
			\includegraphics[width=0.46\textwidth]{figures_of_mine/chapter2/histogram_VBzgsm.pdf}
		}{
			\caption{Frequency distributions for the \vBz{} product. The minutely OMNI data from 1981--2016 is reduced to 3"~hour averages (black line) and 3"~hour minima (red line).}
			\label{fig:histogram_VBzgsm}
		}
%vBz frequency shifts:
% min shift: -1250
% mean shift: -250
% max shift: -750
		\ffigbox{
			\includegraphics[width=0.46\textwidth]{figures_of_mine/chapter2/cc_lag_data_d_KpvsVBzgsm.pdf}
		}{
			\caption{\Kp--\vBz{} correlation coefficients for different time shifts up to \num{+-24}~hours. The minutely OMNI data from 1981--2016 is reduced to 3"~hour averages (black line) and 3"~hour minima (red line).}
			\label{fig:cc_lag_data_d_KpvsVBzgsm}
		}
%highest correlation coefficients:
%min:	0.00000    -0.717240
%mean:	0.00000    -0.362237
%max:	0.00000     0.293137
	\end{floatrow}
\end{figure}

The data is reduced to 3"~hour averages and 3"~hour minima in order to match the \Kp{} data resolution and evaluate the advantage of high resolution data. The \Kp{}--\vBz{} Pearson correlation coefficients for the two differently processed data versions are plotted over time shift in \autoref{fig:cc_lag_data_d_KpvsVBzgsm}. The data with 3"~hour minimum processing shows a much better correlation than the 3"~hour average data. Both curves show a negative correlation and their minima lie at a time shift of zero, with coefficients of $r_\text{min} = -0.72$ and $r_\text{avg} = -0.36$.

The best correlation at the time shift of zero is expected as the OMNI data represents the solar wind at the location of the magnetospheric bow shock. It takes the plasma only a couple of minutes to arrive at the magnetopause and influence the geomagnetic field.\\

Zhang2015 do with their data: ``The (OMNI) data have been lagged by 5~min to allow for propagation from the nose of bow shock to the magnetopause.''\\


\subsection{Functional dependency for solar wind electric field}
An empirical relation between \vBz{} and \Kp{} is sought by processing the data distribution and fitting an appropriate function to it.
%distribution
The frequency distribution in \Kp--\vBz{} space is shaped like a candle flame, inclined to negative values by a light breeze, see top panel in \autoref{fig:Kp_2dhistogram_VBzgsm_sws_e}. The negative correlation is already apparent.
\begin{figure}
	\fcapside[\FBwidth]{
		\includegraphics[width=0.6\textwidth]{figures_of_mine/chapter2/Kp_2dhistogram_VBzgsm_sws_e.pdf}
	}{
		\caption{\Kp{} versus \vBz{} frequency distribution (top panel) and its relative distribution (bottom panel) with the mean \Kp{} values (solid line) and their mean absolute deviation (dotted lines). It is 3-hour minimum data from the minutely OMNI data set (1981--2016). The bin size is \SI{500}{\km\per\s\nano\tesla} and \SI{1/3}{\Kp}~unit respectively.}
		\label{fig:Kp_2dhistogram_VBzgsm_sws_e}
	}
\end{figure}
%dependency
In order to determine a functional dependency, I focus on the relative frequencies per \vBz-interval and their mean \Kp{} values, which are plotted in the bottom panel of \autoref{fig:Kp_2dhistogram_VBzgsm_sws_e}. The mean absolute deviation (MAD) of the mean has a mean size of \SI{0.7}{\Kp}~units. This probability distribution is asymmetrically V-shaped around zero, having a larger and steeper negative arm than positive arm.
%MAD: 2.211/3 = 0.737 Kp units

The asymmetry also exists for 3"~hour mean data, thus this effect is not a result of the data reducing method (3-hour minimum). Rather the steeper negative arm is a consequence of the asymmetric coupling of the solar wind magnetic field direction to the magnetosphere, as described in \autoref{sec:solar_wind_coupling_mechanisms}.\\

%determine fitting functions
The appropriate type of function has to be constructed for the empirical fit. Since the \Kp~index has a quasi-logarithmic scaling (see \autoref{sec:kp_index}), a logarithmic function is the obvious choice as a fit function. Furthermore, the depending argument consists of a product of two solar wind parameters which individually scale logarithmically with \Kp{}. These reasons are why I use the logarithm of a parabola for the fitting approach:
\begin{align}
	f(x) &= \ln\left(x^2\right)	\,.	\label{eq:log_square_function}
\end{align}
I also introduce a horizontal shifting parameter $x'$ because the distribution's center is slightly offset. To be able to replicate the asymmetry in both arms, I further split the fit function at the minimum ($x + x'$) into arms of negative and positive slope:
\begin{align}
	f(x) &=
	\begin{cases}
		\,f_-(x) &\text{for } x + x' < 0	\,,\\
		\,f_+(x) &\text{for } x + x' \ge 0	\,.
	\end{cases}	\label{eq:log_square_fit_function}
\end{align}
This way, both arms can be scaled individually with scaling factors for the negative and positive parts $a_-$ and $a_+$. The resulting logarithmic fit function parts are
\begin{align}
	f_-(x) &= a_- \cdot \ln\left(\left(x + x'\right)^2 + b\right) + y'	\,,\\
	f_+(x) &= a_+ \cdot \left(f_-(x) - f_-\left(-x'\right)\right) + f_-\left(-x'\right)	\,,
\end{align}
with the vertical shifting parameter $y'$ and the depth parameter $b$. The resulting fit curve is plotted in \autoref{fig:Kp_2dhistogram_VBzgsm_sws_fit_e} with the fit coefficients $a_- = 1.258(19)$, $x' = 163(20)$, $b = \num{1.416(68)e6}$, $y' = -17.04(33)$, and \mbox{$a_+ = 0.467(20)$} for units of [\si{\km\per\s \nano\tesla}].
%high precision values:
% a_- = 1.25788(0.019)\\
% y' = -17.0394(0.33)\\
% a_+ = 0.467039(0.0197)\\
% b = 1.41639e6(0.067795e6)\\
% x' = 162.907(20.642)\\
\begin{figure}
	\fcapside[\FBwidth]{
		\includegraphics[width=0.6\textwidth]{figures_of_mine/chapter2/Kp_2dhistogram_VBzgsm_sws_fit_e.pdf}
	}{
		\caption{Mean \Kp{} values (+) and MAD values (dotted lines) per \vBz~interval. The error bars represent the relative data count. The logarithmic fit (dashed line) is plotted with a mean MAD band (shaded area). The splitted function (\ref{eq:log_square_fit_function}) is used for the weighted fit.}
		\label{fig:Kp_2dhistogram_VBzgsm_sws_fit_e}
	}
\end{figure}
Thus, the solar wind dependency relation 'condenses' to:
\begin{align}
	\text{\Kp}_-\left(vB_\text{z}\right) &= 1.258 \cdot \ln\left(\left(vB_\text{z} + 163\right)^2 + \num{1.416e6}\right) - 17.04	\,,	\label{eq:kpvsvbz_dependency_function_negative}\\
	\text{\Kp}_+\left(vB_\text{z}\right) &= 0.467 \cdot \left(\text{\Kp}_-\left(vB_\text{z}\right) - \text{\Kp}_-(-163)\right) + \text{\Kp}_-(-163)	\,.	\label{eq:kpvsvbz_dependency_function_positive}
\end{align}

% warning lead time
This derived E"~field relation can well be applied with the solar wind real-time measurements that are continuously being made by spacecraft located at L1, see \autoref{sec:solar_wind_nowcast_and_forecast_to_earth}. However, the heads-up time for warnings from this location is relatively short with a few tens of minutes, whereas remote observations via solar imagers and coronagraphs allow for longer lead times of a few days.

The E"~field relation derived above could also be applied to remote forecast situations, however, IMF strength and orientation are not yet properly predictable from current prediction methods that rely on remote observations.


\section{Relations between CME/stream velocities and \Kp~index}
\label{sec:relations_between_cme_stream_v_and_kp}

In order to still make use of the enhanced warning lead times from remote observations, the solar wind velocity is the only parameter left which is to a certain degree reliably forecastable. In fact, solar wind velocity and \Kp~index correlate quite well, \citet{Machol2013} even proposed a linear function of the \Kp~index as a best proxy for corrupted real-time velocity measurements made by the ACE spacecraft. For one, the velocity can even be used for persistence forecast, as it has the highest autocorrelation time of all major solar wind parameters -- with \SI{59}{\hour} it is much higher than that with the lowest autocorrelation time of \SI{4}{\hour}, which is indeed the $B_\text{z}$~GSM component \citep{Elliott2013}.

It is clear that the strength of the southward IMF component is the most dominant solar wind parameter for the driving of geomagnetic activity. The velocity shows a strong correlation during CME conditions as well, however, infact during HSSs the velocity is considered the most dominant parameter for driving the geomagnetic activity \citep{Holappa2014}.\\

% separating CMEs and streams
The distinct generation mechanisms of solar wind streams and CMEs show in their different appearance: streams are of continuous nature and CMEs are event-like. Therefore they require completely different forecast methods, see also \autoref{sec:solar_wind_nowcast_and_forecast_to_earth}. Thus, in order to forecast the \Kp~index from remotely derived velocity values, it is apparent and makes sense to analyze CMEs and solar wind streams separately.\\

% velocity relation
It is also known that the solar wind velocity itself already correlates strongly with the \Kp~index.\\

I use this velocity relationship for obtaining \Kp{} proxies from CME and solar wind stream data.\\


\subsection{Solar Wind Structures list}
For the separation of CME and stream data, I use the list of solar wind structures (SWS) created and updated by \citet{Richardson2000} and \citet{Richardson2012}. They characterized the near-Earth solar wind into periods related to slow wind, fast wind, and CMEs. Their list extends back to 1963 and is mostly based on 1"~hour averages of solar wind parameters from the OMNI data set. However, in the cases where there are gaps in the in-situ data, they consider other indicators for identifying solar wind structures. They achieve a fairly complete classification with the inclusion of data from geomagnetic activity, energetic particles, and cosmic rays.
Their work identifies solar wind structures and flags the time series into four categories: HSSs, slow solar wind, CME-associated flows, and undetermined intervals. Periods related to CMEs are defined to also comprise the associated ambient solar wind plasma, which consists of upstream shocks and compressed material. The criterion for differentiating between slow and fast solar wind is a velocity threshold of \SI{400}{\km\per\s}.

The SWS list is made available via registration at CEDARweb: \urltext{http://cedarweb.vsp.ucar.edu/wiki/index.php/Tools_and_Models:Solar_Wind_Structures}.
%List of near-Earth ICMEs since January 1996 by \citet{Cane2003,Richardson2010}. Available as ACE Level~3 data for the period 1995--mid2016\footnote{ACE Level~3 data website -- list of near-Earth ICMEs: \urlfoot{http://www.srl.caltech.edu/ACE/ASC/DATA/level3/icmetable2.htm}}.\\
The updated SWS list (until the end of 2016) was kindly provided by Ian~Richardson.

According to the SWS list, the CME fraction during the 36-year time period considered in the present study, 1981--2016, is \SI{15.4}{\%}, which accumulates to 5.53~years. %and that for the period 1963--2016 is \SI{16.9}{\%} (9.01~years).\\
This percentage is an average value, as the actual fraction varies heavily with the solar activity cycle and with the appearance of individual active regions on the solar surface.

The SWS definition of CME-associated flows makes sense for the following analysis as the causes of the strongest geomagnetic storms are both the compression of the magnetic field within the shock fronts of fast CMEs and the enhanced field strength of the driving magnetic clouds \citep{Bothmer1995}. In the following part of the study, I refer to the combination of the SWS categories slow and fast solar wind simply as solar wind streams. These periods are composed entirely from a mixture of slow wind flows, fast wind streams, and their interaction regions.


\subsection{Data correlation}
Again, as done before for the data processing of the E"~field analysis in \autoref{sec:data_correlation}, I calculate 3-hour extreme values using the minutely OMNI data in order to profit from the higher correlation. The comparison between the 3-hour maximum and the 3-hour mean frequency distributions shows that their mean positions are shifted slightly, having velocities of \SI{405}{\km\per\s} and \SI{425}{\km\per\s} respectively, as is shown in \autoref{fig:histogram_V_b}.
%the SWS1 mean raises from 435 to 455~km/s in 3hmax data...\\
\begin{figure}[htb]
	\begin{floatrow}
		\ffigbox{
			\includegraphics[width=0.5\Xhsize]{figures_of_mine/chapter2/histogram_V_b.pdf}
		}{
			\caption{Solar wind velocity frequency distributions for 3-hour mean, 3-hour maximum, and 3-hour maximum of the CME part. The minutely OMNI data from the period 1981--2016 is used.}
			\label{fig:histogram_V_b}
		}
		\ffigbox{
			\includegraphics[width=\Xhsize]{figures_of_mine/chapter2/cc_lag_sws_d.pdf}
		}{
			\caption{\Kp{}--velocity correlation coefficients for time shifts in the range from $-24$ to 48~hours. The correlations are plotted for the whole solar wind data (solid line), for solar wind streams without CMEs (dashed line), and for CMEs only (dotted line). The data used is the 3-hour maximum of the minutely high resolution OMNI data from the period 1981--2016. add 3-hour mean curves...}
			\label{fig:cc_lag_sws_d}
		}
	\end{floatrow}
\end{figure}
The CME part of the data is examined separately, filtering the related periods using the SWS list. The frequency distribution shows that the CME share is rising in faster solar wind, until eventually in the region above about \SI{900}{\km\per\s} there only exist CMEs.\\

The CME and stream parts of the data are correlated independently with the \Kp~index. The correlation for CME related data is lower than that for all solar wind, see \autoref{fig:cc_lag_sws_d}. Its maximal correlation coefficient has a value of 0.51 and is without time shift, see Table~\ref{tab:correlation_coefficients_kpvsv}.
\begin{table}
	\caption{Time lags with the highest correlation coefficients for the \Kp{}--velocity relation for all solar wind data, for stream data, and for CME data. The values are based on the 3-hour maximum of the minutely high resolution OMNI data from the time period 1981--2016.}
	\label{tab:correlation_coefficients_kpvsv}
	\centering
	\begin{tabular}{lcc}
		\hline\hline
		Data	&Time lag [hours]	&Correlation coefficient\\
		\hline
		All data	&6	&0.622\\
		Streams	&9	&0.661\\
		CMEs	&0	&0.511\\
		\hline
	\end{tabular}
\end{table}
% the best lag times are:\\
% sws: +6 h\\
% sws1: 0 h\\
% sws23: +9 h\\
% 
% correlation coefficients\\
% SWS1\\
% 0	0.511093\\
% SWS23\\
% lag	cross	auto x	auto y\\
% -3	0.660694\\
% 0	0.620113\\
% SWS\\
% lag	cross	auto x	auto y\\
% -2	0.621539\\
% 0	0.595784\\
Solar wind streams show a higher correlation with \Kp{} and the correlation's maximum value 0.66 has a positive time shift of 9~hours. That means the \Kp~index forecasts the velocity of solar wind streams 9~hours in advance.

This positive time shift can be explained from the occurence of interaction regions followed by HSSs. When a slow solar wind stream is followed by a fast one, the compression at their interface leads to enhanced solar wind densities and magnetic field strengths. The peak velocity of a HSS naturally appears after the interaction region. Therefore the \Kp-impact from the enhanced magnetic field is correlated with the higher velocity of the HSS, resulting in the observed positive time shift.\\


\subsection{Functional dependency for CME velocity}
The general \Kp--velocity dependency in the solar wind is apparent from the tilt of its distribution, see top panel of \autoref{fig:Kp_2dhistogram_V_sws_d}.
\begin{figure}
	\fcapside[\FBwidth]{
		\includegraphics[width=0.6\textwidth]{figures_of_mine/chapter2/Kp_2dhistogram_V_sws_d.pdf}
	}{
		\caption{\Kp--velocity distributions for all solar wind data, for solar wind streams, and for CMEs. The data used is the 3"~hour maximum of the minutely high resolution OMNI data from the time period 1981--2016. The SWS list from \citet{Richardson2012} is used for the separation between CME and stream data. The bin size is \SI{10}{\km\per\s} and \SI{1/3}{\Kp}~unit respectively.}
		\label{fig:Kp_2dhistogram_V_sws_d}
	}
\end{figure}
The distribution is inclined to positive values but very broad, that is, in the typical solar wind velocity range it spans over more than half of the total \Kp{} range. The comparison with the filtered data shows that \Kp{} values \num{>6} and velocities \SI{>850}{\km\per\s} are almost always associated with CME related periods, see middle and bottom panel of \autoref{fig:Kp_2dhistogram_V_sws_d}.

In order to determine a functional relation between \Kp{} and velocity, I look at the relative \Kp{} frequencies for each \SI{10}{\km\per\s} velocity interval. The relative frequencies are plotted in the bottom panel of \autoref{fig:Kp_2dhistogram_V_sws1_c2}.
\begin{figure}
	\fcapside[\FBwidth]{
		\includegraphics[width=0.6\textwidth]{figures_of_mine/chapter2/Kp_2dhistogram_V_sws1_c2.pdf}
	}{
		\caption{CME part of the \Kp--velocity distribution (same as third panel of \autoref{fig:Kp_2dhistogram_V_sws_d}) and its relative distribution per velocity interval with the mean \Kp{} values (solid line) and their mean absolute deviation (dotted lines). The bin size is \SI{10}{\km\per\s} and \SI{1/3}{\Kp}~unit respectively.}
		\label{fig:Kp_2dhistogram_V_sws1_c2}
	}
\end{figure}
The mean \Kp{} value seems to scale almost linear with the solar wind velocity. The MAD of the mean has a mean size of about \SI{1.1}{\Kp~units}.
%MAD: 3.338/3 = 1.113 Kp units

%determine fitting function
Again, as the \Kp~index has a quasi-logarithmic scaling, a logarithmic function is the obvious choice for the fitting process. Thus, the logarithmic function
\begin{align}
	f(x) = a \cdot \ln\left(x + x'\right) + y'	\label{eq:log_offset_fit_function}
\end{align}
is used for the fit, with the scaling factor $a$, the location parameter $x'$, and the vertical shifting parameter $y'$. The resulting fit is plotted in \autoref{fig:Kp_2dhistogram_V_sws1_fit_e} and its parameters are $a = \num{10.6(34)}$, $x' = \num{8.1(43)e2}$, and $y' = \num{-73(28)}$, with the velocity in units of [\si{\km\per\s}].
%10.6075 (3.4)\\
%-73.1694 (28.)\\
%806.943 (430)\\
\begin{figure}
	\fcapside[\FBwidth]{
		\includegraphics[width=0.6\textwidth]{figures_of_mine/chapter2/Kp_2dhistogram_V_sws1_fit_e.pdf}
	}{
		\caption{Mean \Kp{} values (+) and MAD values (dotted lines) per velocity interval for the CME part of the data. The error bars represent the relative data count. The logarithmic fit (dashed line) is plotted with a mean MAD band (shaded area). The function (\ref{eq:log_offset_fit_function}) is used for the weighted fit.}
		\label{fig:Kp_2dhistogram_V_sws1_fit_e}
	}
\end{figure}
This leads to the CME dependency function
\begin{align}
	\text{\Kp}_\text{CME}(v) = 10.6 \cdot \ln(v + 810) - 73	\,,	\label{eq:kpvsv_CME_dependency_function}
\end{align}
which can be used to forecast the \Kp{}~index from the estimated CME arrival velocity, e.g., obtained from analyses of remote coronagraph observations.


\subsection{Functional dependency for stream velocity}
The procedure in this section is the same as that in the previous section. However, in the case of solar wind stream velocities, the correlation coefficient is higher when the data is shifted by 9~hours, see \autoref{fig:cc_lag_sws_d}.
I use this shifted data and look at the relative frequencies per velocity interval in order to find a functional dependency between \Kp{} and velocity, see bottom panel of \autoref{fig:Kp_2dhistogram_V_sws23_d}.
\begin{figure}
	\fcapside[\FBwidth]{
		\includegraphics[width=0.6\textwidth]{figures_of_mine/chapter2/Kp_2dhistogram_V_sws23_d.pdf}
	}{
		\caption{Stream part of the \Kp--velocity distribution (similar to second panel of \autoref{fig:Kp_2dhistogram_V_sws_d}, but with the data shifted by 9~hours) and its relative distribution per velocity interval with the mean \Kp{} values (solid line) and their mean absolute deviation (dotted lines). The bin size is \SI{10}{\km\per\s} and \SI{1/3}{\Kp}~unit respectively.}
		\label{fig:Kp_2dhistogram_V_sws23_d}
	}
\end{figure}
Again, the mean \Kp{} value scales almost linear with the velocity. The distribution is much more narrower than that for CMEs -- the MAD of the mean has only a mean size of about \SI{0.7}{\Kp~units}.
%MAD: 2.226/3 = 0.742 Kp units
%MAD: 2.332/3 = 0.777 Kp units for 300--900km/s
%MAD: 2.389/3 = 0.796 Kp units for 350--900km/s
%MAD: 2.454/3 = 0.818 Kp units for 350--850km/s

%determine fitting function
Again, I use the logarithmic function (\ref{eq:log_offset_fit_function}) for the fitting process. The resulting fit is plotted in \autoref{fig:Kp_2dhistogram_V_sws23_fit_e} and the fit parameters are $a = \num{5.88(38)}$, $x' = \num{2.99(49)e2}$, and $y' = \num{-3.70(29)e1}$, with the velocity in units of [\si{\km\per\s}]. The MAD is about \SI{0.7}{\Kp}~units.
%a1 = 5.88(38)
%b1 = -37.0(29)
%x1 = 299(49)
\begin{figure}
	\fcapside[\FBwidth]{
		\includegraphics[width=0.6\textwidth]{figures_of_mine/chapter2/Kp_2dhistogram_V_sws23_fit_e.pdf}
	}{
		\caption{Mean \Kp{} values (+) and MAD values (dotted lines) per velocity interval for the stream part of the data, shifted by 9~hours. The error bars represent the relative data count. The logarithmic fit (dashed line) is plotted with a mean MAD band (shaded area). The function (\ref{eq:log_offset_fit_function}) is used for the weighted fit.}
		\label{fig:Kp_2dhistogram_V_sws23_fit_e}
	}
\end{figure}
This leads to the solar wind stream dependency function
\begin{align}
	\text{\Kp}_\text{Stream}(v) = 5.88 \cdot \ln(v + 299) - 37.0	\,,	\label{eq:kpvsv_stream_dependency_function}
\end{align}
which can be used to forecast the \Kp{}~index from the estimated stream velocity, e.g., obtained from remote coronal hole analyses.


\section{Prediction performance}
\label{sec:prediction_performance}
% prediction model performances
The derived empirical relations are chosen for their high correlation coefficients. However, the correlation coefficient itself does not represent the performance of a model as the coefficient depends highly on the scatter of the underlying distribution. Therefore \citet{Wing2005} advise to additionally provide scatterplots and skill scores for the evaluation of predictive models. In the following, I present the forecast errors and determine the true skill statistics as measures for the prediction performance of the drived empirical models.

% persistence
The quality of predictions can be assessed by how they compare to a simple persistence model \citep{Detman1999}. In the case at hand, the persistence consists of the official \Kp{} value from the previous 3"~hour interval.
% prediction performance
In order to evaluate the models' prediction performances, their forecast errors are calculated. Forecast errors are the differences between the predicted and the actual values. The resulting performances of the persistence and the three models are displayed in \autoref{fig:model_performance_c}, together with the corresponding standard deviations.
\begin{figure}
	\fcapside[\FBwidth]{
		\includegraphics[width=0.6\textwidth]{figures_of_mine/chapter2/model_performance_c.pdf}
	}{
		\caption[\lofimage{figures_of_mine/chapter2/model_performance_c.pdf}]
		{Prediction performance of \Kp{} persistence and the three derived empirical \Kp{} relations. The forecast errors are the differences between the predicted and the actual values. They are calculated from the same data and time range as the relations themselves. The errorbars denote one positive and one negative standard deviation. Perfect predictions are indicated by the gray diagonal lines.}
		\label{fig:model_performance_c}
	}
\end{figure}
The persistence performance is obtained from the complete \Kp{} time span 1981--2016. It does fairly well, as its forecast error is less than 0.7~\Kp{} units up to a \Kp{} of 5.7. At and above a \Kp{} of 6 the error is underestimated with a maximum deviation of 1.3~\Kp{} units.

The derived empirical functions do not cover the whole \Kp{} range from 0 to 9 within the distributions of observed values. Only the \Kp{} values with more than one observed data point are considered for deriving the relations' prediction performances. This excludes the \Kp{}~9 values and in case of the velocity relation for streams also the values \Kp{}~7.7 and above.

% vBz performance
The derived \vBz{} function (\ref{eq:kpvsvbz_dependency_function_negative} + \ref{eq:kpvsvbz_dependency_function_positive}) does only coincide with significant data in the \Kp{} range 1 to 8.7. Especially because the minimum of the \vBz{} function is larger than 0.7~\Kp{}, as can be seen from \autoref{fig:Kp_2dhistogram_VBzgsm_sws_fit_e}. The model performs reasonably well at and below a \Kp{} of 5 with forecast errors smaller than 1~\Kp~unit. Larger \Kp~values are underestimated and deviate up to 1.7~\Kp~units from the predicted values.

% v CME performance
In the case of the velocity relation for CMEs, the derived function (\ref{eq:kpvsv_CME_dependency_function}) does only coincide with significant data in the \Kp{} range 1.3 to 6.3, as can be seen from \autoref{fig:Kp_2dhistogram_V_sws1_fit_e}. The magnitude is overestimated below a \Kp{} of 3 and underestimated above. In the range between a \Kp{} of 1 to 5 the forecast errors are smaller than 1.3~\Kp~units. Above, the error rises up to 4~\Kp~units at a \Kp{} of 8.7.

% v stream performance
For the velocity relation for streams, the derived function (\ref{eq:kpvsv_stream_dependency_function}) does only coincide with significant data in the \Kp{} range 0.3 to 4.3, as can be seen from \autoref{fig:Kp_2dhistogram_V_sws23_fit_e}. The magnitude is overestimated below a \Kp{} of 2 and underestimated above. The forecast errors are smaller than 1.3~\Kp~units throughout the valid range. Above, the error rises up to 4~\Kp~units at a \Kp{} of 7.3.

The model based on the \vBz{} relation is more accurate than those based purely on the velocity. This is expected, however, the \vBz{} relation is still eclipsed by the persistence model.\\

% true skill score
In order to further test the derived models for their predictive value, I derive their true skill statistic (TSS) -- a common tool for forecast verification. The TSS is a skill score based on the contingency table that categorizes forecasted and observed events. The score is the difference between the forecast hit rate and the false alarm rate. Its range goes from $-1$ to 1, where 1 indicates an ideal prediction and 0 a random prediction. I give a more detailed description on the TSS in the Appendix~\ref{sec:true_skill_statistic}. This method is the prevalent form of forecast verification in \Kp{} models \citep{Detman1999,Wing2005,Savani2017}. It is defined that each single 3"~hour \Kp{} interval represents an event and a hit occurs when both the forecasted and observed \Kp{} exceed a specified threshold. I adopt these criteria to make the results comparable. Therefore the TSS is derived as a function of \Kp{} threshold -- the results for persistence and the three models are plotted in \autoref{fig:true_skill_score}.
\begin{figure}
	\fcapside[\FBwidth]{
		\includegraphics[width=0.6\textwidth]{figures_of_mine/chapter2/true_skill_score.pdf}
	}{
		\caption{True skill scores for the \Kp{} persistence and the three prediction relations as a function of \Kp{} threshold. The scores are calculated from the same data and time range as the relations themselves.}
		\label{fig:true_skill_score}
	}
\end{figure}

The E"~field relation reaches its peak at a \Kp{} of 2 with a TSS of 0.63, it then decreases to a local minimum at \Kp{} 6 with a TSS of 0.26. To higher \Kp{} values, the TSS increases again and reaches 0.47 at a \Kp{} of 8.7. This increase is due to the dominant number of correct null forecasts that make the TSS approach the hit rate \citep{Doswell1990}. It is a bias inherent to the TSS in case of rare event forecasting. Both velocity relations have throughout their \Kp{} ranges a significantly lower TSS. The CME relation shows a peak TSS of 0.32 at a \Kp{} of 3 and the stream relation a peak TSS of 0.48 at a \Kp{} of 2. Thus, the TSS gain from the additional $B_\text{z}$ component is about \numrange{0.15}{0.2}.\\

The persistence forecast clearly outperforms the other prediction models, especially in the high \Kp{} range. However, as \citet{Detman1999} stated, it has no warning value because it can only predict a high \Kp{} value after a high \Kp{} value was already observed. Thus, the persistence always misses the onset of a geomagnetic storm, whereas solar wind based models can provide at least a nowcast if not actually a short lead time from the distance of the in-situ measurement location at L1.

% statistics summary table
In order to compare the \Kp{} forecast skill metrics with other models, \citet{Savani2017} use a threshold of $\Kp{} \geq 5$. They choose this value because of the geomagnetic storm scale (G"~scale) developed by NOAA's SWPC. The G"~scale relates the \Kp~index to five levels from G1 to G5\footnote{NOAA Space Weather Scales website: \urlfoot{http://www.swpc.noaa.gov/noaa-scales-explanation}}. The starting level G1 translates to a \Kp{} of 5. I stick to this definition and provide a statistics summary of the \Kp{} persistence and the three derived models in \autoref{tab:model_statistics_table}.
\begin{table}[htb]
	\caption{Statistics of the different prediction models. The metrics are calculated for a threshold hit criteria of $\Kp{} \geq 5$ for the same data and time range as the relations themselves.}
	\label{tab:model_statistics_table}
	\centering
	\begin{tabular}{lcccc}
		\hline\hline
		\multirow{2}{*}{Parameter}	&\Kp{} persistence	&E"~field	&CME velocity 	&Stream velocity\\
			&$\Kp(t-1)$	&$\Kp(\text{\vBz})$	&$\Kp(v_\text{CME})$	&$\Kp(v_\text{Streams})$\\
		\hline
		Time shift [hours]	&3	&0	&0	&9\\
		Event count	&\num{105192}	&\num{79276}	&\num{12116}	&\num{65774}\\
		Correlation coefficient	&0.81	&\num{-0.72}	&0.51	&0.66\\
		\hline
		Proportion correct	&0.96	&0.97	&0.88	&0.98\\
		Hit rate	&0.55	&0.33	&0.13	&0\\
		False alarm rate	&0.02	&0	&0.01	&0\\
% 		FAR	&0.45	&0.24	&0.35	&1\\
% 		DFR	&0.02	&0.03	&0.11	&0.02\\
		True skill score	&0.53	&0.33	&0.12	&0\\
		Valid \Kp{} range	&\numrange{0.3}{8.7}	&\numrange{1.0}{8.7}	&\numrange{1.3}{6.3}	&\numrange{0.3}{4.3}\\
		\hline
	\end{tabular}
\end{table}
Obviously the defined \Kp{} threshold is of little use for the stream velocity relation as its valid \Kp~range is below the defining geomagnetic storm value of \Kp{} 5, that is, storms cannot be predicted via the stream relation.

% comparison
The E"~field model has a TSS of 0.33 at a \Kp{} of 5, this can be compared with the Wing APL models \citep[Fig.~13]{Wing2005}. The Wing APL model~3 predicts \Kp{} 1~hour in advance and uses the solar wind parameters velocity, density, magnetic field strength, and $B_\text{z}$ as input sources for a neural network based model. The additionally considered parameters (density and absolute magnetic field strength) boost the corresponding TSS to about 0.7. The Wing APL model~1 incorporates \Kp{} nowcast values in addition to the solar wind parameters and achieves this way a TSS of about 0.8.

The BSS model developed by \citet{Savani2015,Savani2017} predicts the magnetic field in CMEs arriving at Earth and derives their geomagnetic response. For a total of eight CME events the BSS model shows a TSS of 0.34, when investigating only the periods of ``active solar wind'' \citep[Tab.~3]{Savani2017}.
% Costello model (neural network, input: sw): about 0.45\\
% Wing APL model 1 (1 h, neural network, input: Kp nowcast + sw): about 0.8\\
% Wing APL model 2 (4 h, neural network, input: Kp nowcast + sw): about 0.7\\
% Wing APL model 3 (1 h, neural network, input: sw): about 0.7\\
% BSS model for eight CME events has $\mathit{TSS} = 0.34$ \citep[Tab.~3]{Savani2017}\\



\section{Discussion}
The following results are obtained from the \Kp{} analyses in this chapter:
\begin{itemize*}
	\item A functional dependency for the yearly \Kp{} averages that relates \Kp{} with solar activity via the SSN. The error to it is about 1/3~\Kp{}~unit, whereas seasonal variations further contribute up to 4/3~\Kp{}~units.
	\item A functional dependency for enabling a \Kp{} nowcast. It relates \Kp{} with the solar wind E"~field.
	\item A functional dependency for enabling a CME forecast. It relates \Kp{} with the velocity of CME-associated flows.
	\item A functional dependency for enabling a stream forecast. It relates \Kp{} with the velocity of solar wind streams.
\end{itemize*}


% extreme CME speeds
CMEs can be faster than the maximal velocities included in the OMNI data. Solar wind plasma spectrometers are only able to measure speeds up to about \SI{1100}{\km\per\s} reliably, leading to data gaps during these periods. Yet, CME speeds of up to \SI{2000}{\km\per\s} were observed at \SI{1}{\au} \citep{Russell2013}. According to \autoref{eq:kpvsv_CME_dependency_function}, a CME speed of this value would lead on average to a theoretical \Kp{} of 12.2, however, the \Kp{} scale is capped at 9o and a \Kp{} of 9o is reached already at a velocity of \SI{1489}{\km\per\s}, see \autoref{fig:Kp_2dhistogram_V_sws123_fit_f}.
% 2000 km/s -> 12.17~Kp
\begin{figure}
	\fcapside[\FBwidth]{
		\includegraphics[width=0.6\textwidth]{figures_of_mine/chapter2/Kp_2dhistogram_V_sws123_fit_f.pdf}
	}{
		\caption{Logarithmic fit curves for CMEs (dashed line) and streams (dotted line) with their corresponding mean MAD bands (shaded areas). The curve for solar wind streams is cut because streams do not occur at these high velocities. The curves are the same as in Figures~\ref{fig:Kp_2dhistogram_V_sws23_fit_e} and \ref{fig:Kp_2dhistogram_V_sws23_fit_e}.}
		\label{fig:Kp_2dhistogram_V_sws123_fit_f}
	}
\end{figure}
Future studies have to show how accurate the CME relation is in predicting \Kp{} from extreme CMEs with velocities higher than \SI{1100}{\km\per\s}. (check with Niclas...)\\

The maximal velocity of regular solar wind is limited by the coronal temperatures \citep{Parker1958} and is observed to be around \SI{900}{\km\per\s}. That is why solar wind plasma associated with streams on average provoke \Kp{} values below 5o.\\

In order to demonstrate this relation with actual in-situ data, I calculate the \Kp{} estimate for the example CME from \autoref{fig:ACE_64s_v7_thesis_CME_2013-6-26_6} and include the 3"~hour minimum value of \vBz{} in the plot, see \autoref{fig:example_sw_plot_vBz_b_2013-6-26_6}.
\begin{figure}[htb]
	\centering
	\includegraphics[width=\textwidth]{figures_of_mine/chapter2/example_sw_plot_vBz_b_2013-6-26_6.pdf}
	\caption[\lofimage{figures_of_mine/chapter2/example_sw_plot_vBz_b_2013-6-26_6.pdf}]
	{Solar wind measurements, official \Kp{}~index, and estimated \Kp{} for the time period from 26~June to 2~July 2013. The solar wind parameters are the magnetic field strength, its z"~component in GSM coordinates, and the velocity. I also plot the product \vBz{} with its 3"~hour minimum for illustration. The solar wind data are from the minutely OMNI data set. The official \Kp{}~index is obtained from the GFZ~Potsdam. The \Kp{} estimate is derived via Equations~\ref{eq:kpvsvbz_dependency_function_negative} and \ref{eq:kpvsvbz_dependency_function_negative}. The orange band indicates the mean MAD.}
	\label{fig:example_sw_plot_vBz_b_2013-6-26_6}
	\addtocontents{lof}{\smallskip\protect\center I created the figure myself.\medskip}
\end{figure}
It can be seen that in this period the \Kp{} estimate traces the actual \Kp{}~index pretty well within the mean MAD band. However, deviations are found at the times of the initial shock, and the start and middle of the MC.

It becomes obvious how important the influence of the magnetic field z"~component is, when looking at the same example CME from \autoref{fig:example_sw_plot_vBz_b_2013-6-26_6}. I plot its in-situ magnetic field strength and velocity in \autoref{fig:example_sw_plot_v_b_2013-6-26_6} together with the official \Kp{}~index and the derived \Kp{} estimate.
\begin{figure}[htb]
	\centering
	\includegraphics[width=\textwidth]{figures_of_mine/chapter2/example_sw_plot_v_b_2013-6-26_6.pdf}
	\caption[\lofimage{figures_of_mine/chapter2/example_sw_plot_v_b_2013-6-26_6.pdf}]
	{Solar wind measurements, official \Kp{}~index, and estimated \Kp{} for the time period 26~June to 2~July 2013. The solar wind parameters are the magnetic field strength and the velocity. I also plotted the velocity's 3"~hour maximum for illustration. The solar wind data are from the minutely OMNI data set. The official \Kp{}~index is obtained from the GFZ~Potsdam. I plot the \Kp{} estimate depending on whether the period is flagged as a CME in the SWS list (red line in top panel) or not, that is, via \autoref{eq:kpvsv_CME_dependency_function} or \autoref{eq:kpvsv_stream_dependency_function}. The orange band indicates the mean MAD.}
	\label{fig:example_sw_plot_v_b_2013-6-26_6}
	\addtocontents{lof}{\smallskip\protect\center I created the figure myself.\medskip}
\end{figure}
The latter do not coincide well during this period. The \Kp{} estimate performs okay during the initial sheath region and around the peak of the trailing HSS. The rarefaction regions of declining velocity at the start and end of the considered interval are overestimated, whereas the MC and the trailing compression region are underestimated.

solar wind stream example in \autoref{fig:example_sw_plot_v_c_2013-5-1_65}\\
\begin{figure}[htb]
	\centering
	\includegraphics[width=\textwidth]{figures_of_mine/chapter2/example_sw_plot_v_c_2013-5-1_65.pdf}
	\caption[\lofimage{figures_of_mine/chapter2/example_sw_plot_v_c_2013-5-1_65.pdf}]
	{Solar wind measurements, official \Kp{}~index, and estimated \Kp{} for the time period 1~May to 5~July 2013. The solar wind parameters are the magnetic field strength and the velocity. I also plotted the velocity's 3"~hour maximum for illustration. The solar wind data are from the minutely OMNI data set. The official \Kp{}~index is obtained from the GFZ~Potsdam. I plotted the \Kp{} estimate depending on whether the period is flagged as a CME in the SWS list (red line in top panel) or not, that is, via \autoref{eq:kpvsv_CME_dependency_function} or \autoref{eq:kpvsv_stream_dependency_function}. The orange band indicates the mean MAD.}
	\label{fig:example_sw_plot_v_c_2013-5-1_65}
	\addtocontents{lof}{\smallskip\protect\center I created the figure myself.\medskip}
\end{figure}


The present study adds more empirical \Kp{} relations -- for the specific forecast situations when only the solar wind velocity $v$ or its E"~field are obtainable.\\

differences to existing studies -> Elliott2013\\
It can be looked at how well the solar wind stream velocity relation compares to the compression/rarefaction relations derived by \citet{Elliott2013}.\\

comparison:\\
\Kp--velocity correlation\\
similar to \citet{Elliott2013}; different data time period, resolution and averaging method (3-hour maximum of 1~min data)\\
see Akasofu1981 p.~126, table\\


% % % former outlook

In the present study, the solar activity is neglected for deriving the empirical solar wind--\Kp{} relations. It would be worth examining the solar cycle's influence on the relations, especially as \citet{Wing2005} note that the predictability of \Kp{} slightly scales with solar activity.\\

% low solar activity influence on geomagnetic storms/space weather
\textit{The weaker solar activity cycle 24 seems to have important consequences for CMEs in the heliosphere: CMEs expand anomalously due to reduced heliospheric pressure leading to the increased observed rate of small CMEs, halos originating far from the disk center, and mild space weather (\citep{Gopalswamy2014}, 2015a; Petrie 2015).\\
The total pressure in the heliosphere (magnetic + plasma) is reduced by ca. 40\%, which leads to the anomalous expansion of CMEs explaining the increased slope. The excess CME expansion contributes to the diminished effectiveness of CMEs in producing magnetic storms during cycle 24, both because the magnetic content of the CMEs is diluted and also because of the weaker ambient fields \citep{Gopalswamy2014}.\\
}
maybe the derived relations are less valid during these current conditions?\\

Separate small-scale solar wind structures, such as CIRs and HCSs, for their \Kp{} impact. Filter CME-associated flows into their substructures, such as compressed solar wind plasma and MC plasma, and evaluate their \Kp{} impact individually.\\


list principal conclusions of this study:\\
- \\


\section{Applications}

Prototype \Kp{} relations are integrated into applications developed within the Advanced Forecast For Ensuring Communications Through Space (\mbox{AFFECTS}) project which ran from 2011 to 2013. The following services are accessible via the \mbox{AFFECTS} website\footnote{AFFECTS website: \urlfoot{http://www.affects-fp7.eu/services/}} and contain early results from this \Kp{} study:
\begin{itemize*}
	\item Real-time solar wind plot: \href{http://www.affects-fp7.eu/rssfeeds/ace_ap_forecast_plot/ace_realtime_ap_CH_GFT_plot.png}{Solar Wind and \Kp{} forecast plot} -- ACE/DSCOVR real-time solar wind and \Kp{} forecast plot.
	\item RSS feed: \href{http://www.affects-fp7.eu/rssfeeds/rssfeed_kp/rssfeed_kp.xml}{L1 Kp Alert} -- threshold-based RSS feed that gets triggered when the estimated \Kp{} surpasses a threshold specified as being $7-$.
	\item RSS feed: \href{http://www.affects-fp7.eu/rssfeeds/rssfeed_gnss/rssfeed_gnss.xml}{L1 GNSS Alert} -- threshold-based RSS feed that gets triggered when a specified GNSS error is reached. The value is derived from the \Kp{} estimate.
	\item RSS feed: \href{http://www.affects-fp7.eu/rssfeeds/rssfeed_aurora/rssfeed_aurora.xml}{L1 Aurora Alert} -- threshold-based RSS feed that gets triggered when a specified auroral location is reached. The value is derived from the \Kp{} estimate.
\end{itemize*}

\noindent Further applications of the derived \Kp{}-relations:
\begin{itemize*}
	\item The derived \Kp{} relations are integrated into the UGOE/IAG CME forecast chain (DDC), which consists of CME source region and coronal environment analysis, 3D modeling of the CME structure, CME acceleration and propagation modeling, CME arrival time and parameter prediction, and the subsequent forecast of relevant space weather quantities, such as its impact on the \Kp~index and the ionospheric TEC.
	\item iPhone app L1 Alerts (Solar wind latest 2-hour extreme values and derived forecast values):\\
	\urltext{http://www.affects-fp7.eu/app-services/L1-Alerts/dataL1Alerts.txt}\\
	\urltext{https://itunes.apple.com/au/app/affects/id893579846}
% 	\item Android app L1 Alerts... \urltext{https://play.google.com/store/apps/details?id=com.afects.forecasts}
	\item has SW-Display Kp forecast??
\end{itemize*}



\bigskip
{\small
\textit{Acknowledgments.} Part of the research leading to the results presented in this chapter has received funding from the EU~FP7 project AFFECTS under grant 263506.
The analyses in this chapter rely on the \Kp~index, calculated and made available by the GFZ~Potsdam from data collected at magnetic observatories. Thank goes to the involved national institutes, the INTERMAGNET network and ISGI (\urltext{isgi.unistra.fr}). The author thanks the WDC-SILSO at the SIDC (ROB) for maintaining and providing the international sunspot number series. Additional thank gos to the OMNI PIs/teams for creating and making available the solar wind in-situ data. The OMNI data are supplied by the SPDF at GSFC (NASA). The hourly SWS list, updated until the end of 2016, was kindly provided by Ian~Richardson of the GSFC and CRESST/University of Maryland.
}


\section*{notes...}

Savani2017:\\
The Kp difference of 1.5 is tested as there is evidence that a limitation in accuracy is present in the underlying empirical Kp formulation [Mays et al., 2015a]. [...] which is approximately consistent with recent results by Mays et al. [2015a], the Kp empirical formulation is accurate to about Kp = 1.5.\\

What kind of solar wind structures create the individual regions in this distribution? (B--v--\Kp{} circle plot)\\
What is their individual contribution to the \Kp{} ranges (e.g. high \Kp{}: CMEs 70\% and CIRs 30\%)?\\

How can the impact field strength of CMEs be forecasted (v->B correlation for CMEs)?\\
Internal solar wind correlations: B--v correlation\\
ACE MAGSWE 64~s data -> yearly overlay plot\\

Using \vBz{} implies that both quantities are sufficiently independent. $B$ and $v$ are dependent, as can be seen from fig~XX, however, as $v$ and $\theta_c$ are, so are $v$ and $B_\text{z}$, see fig~XX.\\

rt data errors/gaps... vs science data (see paper Kp as V replacement: \citet{Machol2013})\\
DSCOVR as replacement was launched on 11~Februar 2015. It is NOAA's SWPC real-time solar wind prime source since 27 July 2016.\footnote{\urlfoot{http://www.swpc.noaa.gov/products/real-time-solar wind}}\\


