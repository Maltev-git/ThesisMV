
\chapter{Conclusions}
\label{chap:summary}

%COFI -- chapter outline and flow integration\\
Here, I briefly summarize the principal results of this work and give an outlook on further objectives/questions based on/raised by this work.

\section{Summary}
% retelling of the principal results
% summarizing briefly the evidence for each conclusion
% where the work leaves us

% two topics
In this work I addressed solar wind problems relevant to space weather and to the Parker~Solar~Probe mission. Predictive models related to solar wind were developed in both studies. In the first study I modeled the solar wind's and CMEs' impact on geomagnetic activity and in the second study I developed an empirical solar wind model for the inner heliosphere.\\


% long synopsis chapter2 without refs
The objectives of the analyses performed and presented in this study are to estimate the geomagnetic impact of solar wind and to predict it for CMEs and streams in particular. I determined the magnitudes of the long-term \Kp{} changes due to solar activity and I measured the extent of seasonal variations stemming from the Earth's orbit. In order to nowcast the \Kp~index, I quantified the solar wind influence on \Kp{} by deriving a functional relation with the solar wind E"~field. For the purpose of enabling \Kp{} forecasts from remote observations, I estimated the \Kp{} impact coming from CMEs and streams separately by deriving functional dependencies with their velocities. The obtained \Kp{} relations were evaluated for their prediction performance. The results were further discussed and the conclusions were presented.\\



Impact estimations derived from empirical correlations between in-situ solar wind measurements and the geomagnetic \Kp~index\\

In order to derive the link from near-Earth solar wind properties to its \Kp{}~impact, 35~years of solar wind high-resolution measurements were correlated with the \Kp~index and empirical relations were derived that enable the prediction of \Kp{} from the solar wind velocity and the IMF z"~component. Early predictions of CME and ambient stream magnetic fields obtained from near-Sun observations currently either are not existent or come with high uncertainties. Thus, to provide \Kp{} predictions, I derived relations exclusively from their arrival velocity.\\


conclusions:\\
- \\



% long synopsis chapterPSP without refs
This study is constructed as follows: first, the PSP mission and its scientific goals were described. I introduced the analyses done in the publication \citet{Venzmer2018}, which constitutes the major part of this study. An alternative magnetic field model was derived, which changes the solar distance dependency from the power law in our article to support a Parker magnetic field geometry. Further, possible improvements to the solar wind model were sketched.\\

% synopsis from paper
Our approach is to obtain analytical representations of the shapes of the solar wind parameter’s frequency distributions, of their solar activity dependence, and of their solar distance scaling. The solar wind parameters’ frequency distributions and solar activity dependence were derived from near-Earth solar wind and SSN time series with a duration of almost five solar cycles. Their distance dependency was derived from Helios solar wind measurements covering more than two thirds of the distance to the Sun and more than half a solar cycle. From a combination of the obtained frequency distributions, SSN dependence functions, and solar distance dependence functions, a general solar wind model was built, representing the solar activity and distance behavior. Finally, this empirical model was fed with SSN predictions and extrapolated to PSP’s planned orbital positions.\\



I built an empirical solar wind model for the inner heliosphere in the ecliptic which accounts for the variations of the solar activity cycle and for solar distance. In order to obtain empirical estimates of the solar wind environment PSP is to encounter, this solar wind model was extrapolated down to PSP's planned near-Sun perihelia and was built to consider the expected solar activity during PSP's mission.\\

The newly derived distance dependency, extrapolated to the PSP orbit and combined with the solar activity dependency described in the paper, yields \SI{8}{\%} higher values at the first perihelion and \SI{32}{\%} higher values at the closest perihelion in comparison to the estimates derived in the paper.\\

conclusions:\\
- the velocity extrapolation indicates that sw acceleration takes place up to solar distances of XX~Rs\\
- the temperature extrapolation suggests that plasma heating takes place up to solar distances of XX~Rs\\
- PSP at its perihelia will possibly encounter IMF extreme values of up to 1200 bzw. 1600~nT\\


% results summary handout
results summary handout\\
- equations and under which circumstances they are to be used\\


\subsubsection*{Solar wind and CME influence on the magnetosphere}

Relation between the yearly averages of the previous year's SSN and the \Kp~index:
\begin{align}
	\Kp(ssn) = 0.28 \cdot \ln(ssn) + 1.1	\,.
\end{align}

The seasonal variations in \Kp{} contribute up to 1.3~\Kp~units.\\

Relation between the 3"~hour minimum values of \vBz{} and the \Kp~index:
\begin{align*}
	\text{\Kp}\left(vB_\text{z}\right) &=
	\begin{cases}
		\,\text{\Kp}_-\left(vB_\text{z}\right) &\text{for } (vB_\text{z} + 163) < 0	\,,\\
		\,\text{\Kp}_+\left(vB_\text{z}\right) &\text{for } (vB_\text{z} + 163) \ge 0	\,.
	\end{cases}
\end{align*}
\begin{align*}
	\text{\Kp}_-\left(vB_\text{z}\right) &= 1.258 \cdot \ln\left(\left(vB_\text{z} + 163\right)^2 + \num{1.416e6}\right) - 17.04	\,,\\
	\text{\Kp}_+\left(vB_\text{z}\right) &= 0.467 \cdot \left(\text{\Kp}_-\left(vB_\text{z}\right) - \text{\Kp}_-(-163)\right) + \text{\Kp}_-(-163)	\,.
\end{align*}
The MAD of the mean has a mean size of \SI{0.7}{\Kp}~units.\\
Restrictions: This relation is able to predict \Kp{} in the range \numrange{1.0}{8.7}.\\

Relations between the 3"~hour maximum values of $v$ and the \Kp~index for CME-associated flows (including shocked and compressed solar wind plasma) and solar wind streams (fast and slow wind including interaction regions):
\begin{align*}
	\text{\Kp}_\text{CME}(v) &= 10.6 \cdot \ln(v + 810) - 73	\,,\\
	\text{\Kp}_\text{Stream}(v) &= 5.88 \cdot \ln(v + 299) - 37.0	\,.
\end{align*}
The MAD of the mean has a mean size of about \SI{1.1}{\Kp~units} for the CME case and \SI{0.7}{\Kp}~units for streams.\\
Restrictions: These relations are able to predict \Kp{} in the ranges \numrange{1.3}{6.3} and \numrange{0.3}{4.3} respecively.\\


\subsubsection*{Empirical solar wind model for the inner heliosphere}

Empirical solar wind models are derived for the magnetic field strength, proton velocty, density, and temperature. The models comprise the empirically determined frequency distributions and shifts them with solar distance and solar activity. The distributions are assumed to be of lognormal shape, but the velocity's fast and slow regimes are considered by modeling a double lognormal distribution. The solar activity dependence is based on the SSN.\\

The relevant solar wind relations are presented in \citet[p.~10]{Venzmer2018}.\\
They are valid within the ecliptic plane.\\


\subsubsection*{Parker magnetic field solar distance dependency}
Advantage of the IMF model: considers the different distance scaling of the individual vector components.\\

Solar distance relations for the magnetic field strength's median and the mean:
\begin{align}
	B_\text{med}(r) = 4.0 \cdot \sqrt{\left(r^{-1.85}\right)^2 + \left(r^{-1.0}\right)^2}	\,,\\
	B_\text{avg}(r) = 4.5 \cdot \sqrt{\left(r^{-1.74}\right)^2 + \left(r^{-1.0}\right)^2}	\,.
\end{align}
Modeled median and mean relations for shifting the magnetic field strength's lognormal distribution:
\begin{align}
	B_\text{med}(ssn,r) &= \left(\SI{0.0131}{\nT} \cdot ssn + \SI{4.29}{\nT}\right) \cdot \sqrt{\left(r^{-1.86}\right)^2 + \left(r^{-1.3}\right)^2}	\,,\\
	B_\text{avg}(ssn,r) &= 1.0879 \cdot B_\text{med}(ssn,r)	\,.
\end{align}
Restrictions of the magnetic field model: The relations assume that the azimuthal field component is nonexistent and the Parker spiral angle at \SI{1}{\au} has a value of \SI{-45}{\degree} or \SI{135}{\degree}.\\

The estimated magnetic field strength median value for PSP's first perihelion is with of \SI{94}{\nano\tesla} \SI{8}{\%} higher than that derived in the paper. The median value for the first closest perihelion is with \SI{1241}{\nano\tesla} even \SI{32}{\%} higher.\\


\section{Outlook}

% where the work leads us
% questions raised by this work
% offer testable hypotheses for future researchers
% what approaches I would have chosen, knowing what I now know?
% specific objectives based on unanswered questions in my work



% % % chapter2 outlook
Testing Newell's universal coupling function with high resolution data would be interesting.\\

The \Kp{} scale is capped at $9o$ -- it could be extended. \Kp{} is connected to the disturbance field strength at the \SI{+-50}{\degree} latitudes via the \textit{ap}~index. Via extending the derived \Kp{} relations above $9o$, this way the absolute field disturbances at the ground could be estimated. for the unresolved extreme CME events.\\

The present study adds more empirical \Kp{} relations -- for the cases when only the solar wind velocity $v$ or its electric field \vBz{} are known, see \autoref{chap:chapter2}. The derived \Kp{} relations are one of the final steps in the CME forecast chain developed at the Institute for Astrophysics Göttingen.\\

Separate small-scale solar wind structures, such as CIRs and HCSs, for their \Kp{} impact. Filter CME-associated flows into their substructures, such as compressed solar wind plasma and MC plasma, and evaluate their \Kp{} impact individually.\\



% % % former chapterPSP outlook
Further investigations should be done into the extrapolation of the different solar wind structures, as for example the properties of CMEs and MCs differ from those of the bulk solar wind distance scaling laws \citep{Bothmer1998}. Further, the differing contribution of solar wind structures with solar activity has an influence on the derived fit parameters.\\

Additional solar wind data captured from spacecraft that studied Mercury (perihelion distance of \SI{0.3}{\au}), such as Mariner~10 and MESSENGER, could be included into the analyses to refine the distance dependency of the solar wind model.
%MESSENGER: B-field 2004--2011
However, especially the near-Sun in-situ measurements of PSP itself, starting in fall 2018, will eventually reveal how accurate the predictions of the derived solar wind models are.\\

Using the SSN prediction, the derived solar wind model allows the forecast and extrapolation of the solar wind environment occuring during the PSP mission's near-Sun encounters from end of 2018 on. The anticipated Parker~Solar~Probe data (near-Sun measurements) will reveal how far the solar wind estimates really are from reality and thus is able to locate the outer boundary of the solar wind acceleration region.\\

could be integrated into the sw model, such as flux conservation, seasonal and latitudinal effects, and the distance behavior of different solar wind structure types.\\

other possible spacecraft missions that would be beneficial for space weather forecasting: sub-L1 for earlier in-situ CME magnitude warning (cite?) and L5 for early CME velocity and arrival warning \citep{Vourlidas2015}\\

There exist a lot more aspects related to solar wind that are not covered in this study, i.e., MHD waves, the electron pitch angle, solar energetic particles, dust particles, and neutral energetic atoms. It would be worth looking into these properties as well...\\


%german version of summary?




