
\chapter{Summary and outlook}
\label{chap:summary}

%COFI -- chapter outline and flow integration\\

% already in chapters:\\
% results\\
% discussion\\
% conclusions\\
% 
% end matter:\\
% summary\\
% outlook\\

\section{Summary}

In order to derive the link from near-Earth solar wind properties to its \Kp{}~impact, 35~years of solar wind high-resolution measurements were correlated with the \Kp~index and empirical relations were derived that enable the prediction of \Kp{} from the solar wind velocity and the IMF z"~component. Early predictions of CME and ambient stream magnetic fields obtained from near-Sun observations currently either are not existent or come with high uncertainties. Thus, to provide \Kp{} predictions, I derived relations exclusively from their arrival velocity.\\


I built an empirical solar wind model for the inner heliosphere in the ecliptic which accounts for the variations of the solar activity cycle and for solar distance. In order to obtain empirical estimates of the solar wind environment PSP is to encounter, this solar wind model is extrapolated down to PSP's planned near-Sun perihelia and is built to consider the expected solar activity during PSP's mission.\\



\section{Outlook}

Using the SSN prediction, the derived solar wind model allows the forecast and extrapolation of the solar wind environment occuring during the PSP mission's near-Sun encounters from end of 2018 on. The anticipated Parker~Solar~Probe data (near-Sun measurements) will reveal how far the solar wind estimates really are from reality and thus is able to locate the outer boundary of the solar wind acceleration region.\\

DSCOVR data (advantages over ACE? gain?)\\

other possible spacecraft missions that would be beneficial for space weather forecasting: sub-L1 for earlier in-situ CME magnitude warning (cite?) and L5 for early CME velocity and arrival warning \citep{Vourlidas2015}\\


%german version of summary


