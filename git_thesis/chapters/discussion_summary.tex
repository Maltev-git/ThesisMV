
\chapter{Conclusions}
\label{chap:summary}

%COFI -- chapter outline and flow integration\\
Here, I briefly summarize the principal results of this work and give an outlook on further objectives/questions based on/raised by this work.

\section{Summary}
% retelling of the principal results
% summarizing briefly the evidence for each conclusion
% where the work leaves us

% two topics
In this work I addressed solar wind problems relevant to space weather and to the Parker~Solar~Probe mission. Predictive solar wind models were developed in both studies. In the first study I modeled the solar wind's and CMEs' impact on geomagnetic activity and in the second I developed an empirical solar wind model for the inner heliosphere.\\


% long synopsis chapter2 without refs
long synopsis chapter2 without refs\\

Impact estimations derived from empirical correlations between in-situ solar wind measurements and the geomagnetic \Kp{}~index\\

In order to derive the link from near-Earth solar wind properties to its \Kp{}~impact, 35~years of solar wind high-resolution measurements were correlated with the \Kp~index and empirical relations were derived that enable the prediction of \Kp{} from the solar wind velocity and the IMF z"~component. Early predictions of CME and ambient stream magnetic fields obtained from near-Sun observations currently either are not existent or come with high uncertainties. Thus, to provide \Kp{} predictions, I derived relations exclusively from their arrival velocity.\\


conclusions:\\
- for Kp nowcasts via the E-field the sw data processing is of major importance\\
- average and strong geomagnetic activity is likely underestimated by the predictions from the velocity--Kp relations\\
- from the derived ambient solar wind velocity--Kp relation it is not possible forecast geomagnetic storms\\
- from the derived CME velocity--Kp relation strong geomagnetic storms are likely to be underestimated\\
- \\



% long synopsis chapterPSP without refs
long synopsis chapterPSP without refs\\

I built an empirical solar wind model for the inner heliosphere in the ecliptic which accounts for the variations of the solar activity cycle and for solar distance. In order to obtain empirical estimates of the solar wind environment PSP is to encounter, this solar wind model was extrapolated down to PSP's planned near-Sun perihelia and was built to consider the expected solar activity during PSP's mission.\\

conclusions:\\
- the velocity extrapolation indicates that sw acceleration takes place up to solar distances of XX~Rs\\
- the temperature extrapolation suggests that plasma heating takes place up to solar distances of XX~Rs\\
- PSP at its perihelia will possibly encounter IMF extreme values of up to 1200 bzw. 1600~nT\\


% results summary handout
results summary handout\\
- equations and under which circumstances they are to be used\\


\subsubsection*{Solar wind and CME influence on the magnetosphere}

Relation between the yearly averages of the previous year's SSN and the \Kp~index:
\begin{align}
	\Kp(ssn) = 0.28 \cdot \ln(ssn) + 1.1	\,.
\end{align}

The seasonal variations in \Kp{} contribute up to 1.3~\Kp~units.\\

Relation between the 3"~hour minimum values of \vBz{} and the \Kp~index:
\begin{align*}
	\text{\Kp}\left(vB_\text{z}\right) &=
	\begin{cases}
		\,\text{\Kp}_-\left(vB_\text{z}\right) &\text{for } (vB_\text{z} + 163) < 0	\,,\\
		\,\text{\Kp}_+\left(vB_\text{z}\right) &\text{for } (vB_\text{z} + 163) \ge 0	\,.
	\end{cases}
\end{align*}
\begin{align*}
	\text{\Kp}_-\left(vB_\text{z}\right) &= 1.258 \cdot \ln\left(\left(vB_\text{z} + 163\right)^2 + \num{1.416e6}\right) - 17.04	\,,\\
	\text{\Kp}_+\left(vB_\text{z}\right) &= 0.467 \cdot \left(\text{\Kp}_-\left(vB_\text{z}\right) - \text{\Kp}_-(-163)\right) + \text{\Kp}_-(-163)	\,.
\end{align*}
The MAD of the mean has a mean size of \SI{0.7}{\Kp}~units.\\
Restrictions: This relation is able to predict \Kp{} in the range \numrange{1.0}{8.7}.\\

Relations between the 3"~hour maximum values of $v$ and the \Kp~index for CME-associated flows (including shocked and compressed solar wind plasma) and solar wind streams (fast and slow wind including interaction regions):
\begin{align*}
	\text{\Kp}_\text{CME}(v) &= 10.6 \cdot \ln(v + 810) - 73	\,,\\
	\text{\Kp}_\text{Stream}(v) &= 5.88 \cdot \ln(v + 299) - 37.0	\,.
\end{align*}
The MAD of the mean has a mean size of about \SI{1.1}{\Kp~units} for the CME case and \SI{0.7}{\Kp}~units for streams.\\
Restrictions: These relations are able to predict \Kp{} in the ranges \numrange{1.3}{6.3} and \numrange{0.3}{4.3} respecively.\\


\subsubsection*{Empirical solar wind model for the inner heliosphere}

Empirical solar wind models are derived for the magnetic field strength, proton velocty, density, and temperature. The models comprise the empirically determined frequency distributions and shifts them with solar distance and solar activity. The distributions are assumed to be of lognormal shape, but the velocity's fast and slow regimes are considered by modeling a double lognormal distribution. The solar activity dependence is based on the SSN.\\

The relevant solar wind relations are presented in \citet[p.~10]{Venzmer2018}.\\
They are valid within the ecliptic plane.\\


\subsubsection*{Parker magnetic field solar distance dependency}
Advantage of the IMF model: considers the different distance scaling of the individual vector components.\\

Solar distance relations for the magnetic field strength's median and the mean:
\begin{align}
	B_\text{med}(r) = 4.0 \cdot \sqrt{\left(r^{-1.85}\right)^2 + \left(r^{-1.0}\right)^2}	\,,\\
	B_\text{avg}(r) = 4.5 \cdot \sqrt{\left(r^{-1.74}\right)^2 + \left(r^{-1.0}\right)^2}	\,.
\end{align}
Modeled median and mean relations for shifting the magnetic field strength's lognormal distribution:
\begin{align}
	B_\text{med}(ssn,r) &= \left(\SI{0.0131}{\nT} \cdot ssn + \SI{4.29}{\nT}\right) \cdot \sqrt{\left(r^{-1.86}\right)^2 + \left(r^{-1.3}\right)^2}	\,,\\
	B_\text{avg}(ssn,r) &= 1.0879 \cdot B_\text{med}(ssn,r)	\,.
\end{align}
Restrictions of the magnetic field model: The relations assume that the azimuthal field component is nonexistent and the Parker spiral angle at \SI{1}{\au} has a value of \SI{-45}{\degree} or \SI{135}{\degree}.\\

The estimated magnetic field strength median value for PSP's first perihelion is with of \SI{94}{\nano\tesla} \SI{8}{\%} higher than that derived in the paper. The median value for the first closest perihelion is with \SI{1241}{\nano\tesla} even \SI{32}{\%} higher.\\


\section{Outlook}

% where the work leads us
% questions raised by this work
% offer testable hypotheses for future researchers
% what approaches I would have chosen, knowing what I now know?
% specific objectives based on unanswered questions in my work


% % % chapter2 outlook
Testing Newell's universal coupling function with high resolution data would be interesting.\\

The \Kp{} scale is capped at $9o$ -- it could be extended. \Kp{} is connected to the disturbance field strength at the \SI{+-50}{\degree} latitudes via the \textit{ap}~index. Via extending the derived \Kp{} relations above $9o$, this way the absolute field disturbances at the ground could be estimated. for the unresolved extreme CME events.\\

The present study adds more empirical \Kp{} relations -- for the cases when only the solar wind velocity $v$ or its electric field \vBz{} are known, see \autoref{chap:chapter2}. The derived \Kp{} relations are one of the last parts in the UGOE/IAG CME forecast chain.\\

Separate small-scale solar wind structures, such as CIRs and HCSs, for their \Kp{} impact. Filter CME-associated flows into their substructures, such as compressed solar wind plasma and MC plasma, and evaluate their \Kp{} impact individually.\\


% % % former chapterPSP outlook
Further investigations should be done into the extrapolation of the different solar wind structures, as for example the properties of CMEs and MCs differ from those of the bulk solar wind distance scaling laws \citep{Bothmer1998}. Further, the differing contribution of solar wind structures with solar activity has an influence on the derived fit parameters.\\

Additional solar wind data captured from spacecraft that studied Mercury (perihelion distance of \SI{0.3}{\au}), such as Mariner~10 and MESSENGER, could be included into the analyses to refine the distance dependency of the solar wind model.
%MESSENGER: B-field 2004--2011
However, especially the near-Sun in-situ measurements of PSP itself, starting in fall 2018, will eventually reveal how accurate the predictions of the derived models are.\\

Using the SSN prediction, the derived solar wind model allows the forecast and extrapolation of the solar wind environment occuring during the PSP mission's near-Sun encounters from end of 2018 on. The anticipated Parker~Solar~Probe data (near-Sun measurements) will reveal how far the solar wind estimates really are from reality and thus is able to locate the outer boundary of the solar wind acceleration region.\\

could be integrated into the sw model, such as flux conservation, seasonal and latitudinal effects, and the distance behavior of different solar wind structure types.\\

other possible spacecraft missions that would be beneficial for space weather forecasting: sub-L1 for earlier in-situ CME magnitude warning (cite?) and L5 for early CME velocity and arrival warning \citep{Vourlidas2015}\\


%german version of summary?


\section{notes...}

put fig. somewhere earlier (p.~10?): see \autoref{fig:Hundhausen1977_fig20bcd}\\
\begin{figure}[htb]
	\centering
	\includegraphics[width=0.5\textwidth]{figures_of_others/images/Hundhausen1977_fig20bcd.png}
	\caption[\lofimage{figures_of_others/images/Hundhausen1977_fig20bcd.png}Credit: {\citep[Fig.~20, panels (b--d)]{Hundhausen1977}}, \textcopyright~Colorado Associated University Press, reproduced with permission.]
	{Schemata of different coronal configurations during the solar cycle. Visualized are the locations of closed coronal magnetic fields (lines) and coronal holes (black areas) for a post maximum distorted dipole (top panel), for a pre minimum stable tilted dipole (middle panel), and a post minimum axial dipole (bottom panel). Credit: {\citep[Fig.~20, panels (b--d)]{Hundhausen1977}}, \textcopyright~Colorado Associated University Press, reproduced with permission.}
	\label{fig:Hundhausen1977_fig20bcd}
\end{figure}

causes (see citet{Rangarajan1997} p.~1282 and mention Bartels1963 too)\\

\citet{Sonnerup1967}: The rotational discontinuity seems to occur predominantly during magnetic storms and two of these cases, involving substantial normal-field components, provide compelling evidence that field reconnection takes place during the storm main phase.\\

% mass flux considerations:\\
% mass per second per Earth disc...\\
% R_E = 6371.008 km\\
% A_E = pi * R_E² = 127 516 438.219 km²\\
% m_p = 1.672621898e-27 kg\\
% v = 400 km/s\\
% n = 6.5 cm-3 = 6.5e15 km-3\\
% mass flux_E = m_p * n * v * A_E = 0,554538384917 kg/s\\
% = 555 g/s = 1996 kg/h = 48 t/d = 17487 t/a\\
% mass per second per magnetosphere disc...\\
% R_MP = 15 * R_E
% A_MP = pi * R_MP² = 28 691 198 599.3 km²\\
% mass flux_MP = 124,772770348 kg/s\\
% = 449 t/h = 10 780 t/d = 3 934 834 t/a\\

% Wikipedia: https://en.wikipedia.org/wiki/Solar_wind
% The wind exerts a pressure at 1 AU typically in the range of 1–6 nPa (1–6×10−9 N/m2), although it can readily vary outside that range.
% The ram pressure is a function of wind speed and density. The formula is
% P = mp * n * V2= 1.6726×10−6 * n * V2
% where mp is the proton mass, pressure P is in nPa (nanopascals), n is the density in particles/cm3 and V is the speed in km/s of the solar wind.[citation needed]

% HCS
why HCS? what currents are there...\\
why is the CS not the divide between both polarities?\\

% space weather
"The principal users affected by geomagnetic storms are the electric power grid, spacecraft operations, users of radio signals that reflect off of or pass through the ionosphere, and observers of the aurora." NOAA cite\\

strong radio bursts; are they relevant direct disturbances for space weather?\\

The solar wind total energy flux at Earth ($1.45~\text{mW/m}^2$) is only about one millionth of the solar radiation flux at Earth (see \citet[p.~153]{Schwenn1990}).\\

% magnetosphere
Phan2005, Magnetopause Processes:\\
Cluster findings include:\\
A strong ‘guide field’ detected at a reconnection X-line, i.e., a finite magnetic field along the X-line, has provided direct evidence for component merging.\\
Tailward-of-the-cusp reconnection has been found to occur only when the IMF has a northward component. The occurrence rate of cusp reconnection is nearly 100\% when the IMF has a northward component, implying that cusp reconnection in the northern and southern hemispheres must be common. The high occurrence rate (in contrast to a rate of 50\% at the subsolar magnetopause) is thought to be due to the presence of a plasma depletion layer. In this layer, the plasma beta is reduced, rendering the magnetosheath flow sub-Alfvenic and allowing the establishment of a stable X-line at the high-latitude magnetopause.\\

look into printed paper collection...\\

% geomagnetic indices
\citet{Lockwood2014} even used geomagnetic indices (including $aa$) to reconstruct the near-Earth IMF strength and solar wind flow speed back to 1845.\\

% halo CMEs:
observed coronal transient: 3d structure, directed at Earth, associated with shock wave; \citep{Howard1982} -> halo CME\\

% chapter2
This work adopts these assumptions and treats the solar wind throughout as a proton plasma.\\

% Kp forecast models
% Aleksei's f/c model; Ukraine; developed within AFFECTS project Space Research Institute of NAS of Ukraine and SSA of Ukraine, Ukraine (SRI NASU-NSAU)\\
% http://www.ikd.kiev.ua/index.php?option=com_content&view=article&id=22%3A2011-02-09-12-19-54&catid=12&Itemid=15&lang=en

% The $\text{d}\Phi / \text{d}t$ coupling relation is being implemented into CME forecast procedures \citep{Savani2017}.\\
% NASA's Space Weather Research Center (SWRC); the derived Kp estimates using real-time in-situ data from L1 are currently implemented within the CCMC integrated Space Weather Analysis System (iSWA); Community Coordinated Modeling Center (CCMC) at the Goddard Space Flight Center\\

% coupling functions
\citet{Elliott2013}: The \Kp~index and solar wind speed relationship: Insights for improving space weather forecasts\\

% Kp index
There exist several indicators/quantities that scale or are based on the \Kp{}~index:
\begin{itemize*}
	\item The equatorward auroral boundary position correlates with the \Kp~index (cite?).
	\item The variation of the total electron content (TEC) of the ionosphere correlates with the \Kp~index (cite?). The TEC has influence on global navigation satellite systems (GNSS). A part of their positional error scales directly with TEC (in extreme cases up to about \SI{30}{\m}).
\end{itemize*}

% solar wind impacts
These solar wind real-time data are used to nowcast various effects on the Earth's magnetosphere, such as the position of the magnetospheric bow shock in front of the Earth, the magnitude of geomagnetic disturbances, the positions of the polar auroral ovals, the variation of the total electron content (TEC) of the ionosphere, and the positional error of global navigation satellite systems (GNSS).\\

% % % chapter2 notes
Savani2017:\\
The Kp difference of 1.5 is tested as there is evidence that a limitation in accuracy is present in the underlying empirical Kp formulation [Mays et al., 2015a]. [...] which is approximately consistent with recent results by Mays et al. [2015a], the Kp empirical formulation is accurate to about Kp = 1.5.\\

What kind of solar wind structures create the individual regions in this distribution? (B--v--\Kp{} circle plot)\\
What is their individual contribution to the \Kp{} ranges (e.g. high \Kp{}: CMEs 70\% and CIRs 30\%)?\\

How can the impact field strength of CMEs be forecasted (v->B correlation for CMEs)?\\
Internal solar wind correlations: B--v correlation\\
ACE MAGSWE 64~s data -> yearly overlay plot\\

Using \vBz{} implies that both quantities are sufficiently independent. $B$ and $v$ are dependent, as can be seen from fig~XX, however, as $v$ and $\theta_c$ are, so are $v$ and $B_\text{z}$, see fig~XX.\\

rt data errors/gaps... vs science data (see paper Kp as V replacement: \citet{Machol2013})\\
DSCOVR as replacement was launched on 11~Februar 2015. It is NOAA's SWPC real-time solar wind prime source since 27 July 2016.\footnote{\urlfoot{http://www.swpc.noaa.gov/products/real-time-solar wind}}\\


