\begin{titlepage}
	\begin{center}
		%\renewcommand{\baselinestretch}{1.5}
		{\LARGE\sffamily
			\textbf{Solar wind -- Predicted magnetospheric influence\\and near-Sun environment}\\
		}
		\Large
		\textit{Malte~S.~Venzmer}\\
		title?\\
		middle name?\\
		english vs deutsch?\\
		make cover page image...; combine figs 4.9+4.10; combine fig 5.5\\
		University of Göttingen logo?\\
		Institute for Astrophysics logo?\\
		January 2012 -- Feb? 2018\\
		german title and abstract?\\
	\end{center}

\vfill

\begin{footnotesize}
\noindent version log:\\
1	2013-11-06\\
2	2013-11-07	inserted lem excerpt\\
3	2014-09-01	outlined introduction structure and appendix key words\\
4	2014-09-11	bibtex included, lem excerpt citet\\
5	2014-09-12	included italic titles in bibliography\\
6	2015-01-12	worked on introduction structure and included magnetic butterfly diagram\\
7	2015-05-04	first printout, small additions\\
8	2015-07-03	coupling functions, content structure\\
9	2015-09-09	Alfv\'en wave velocity formula\\
10	2015-09-10	Appendix physics: Alfv\'en and compressional MHD waves, plasma beta; citation sources Cranmer2005, Kivelson1995\\
% 11	2015-09-15	CGAUSS work overview; DQCS model figure\\
% 12	2015-09-16	magnetic energy density is the same as magnetic pressure, abbreviations\\
% 13	2015-09-30	empirical solar-wind model plots in analyses, figure label [Figure 4.2  Text]\\
% 14	2015-10-02	table column decimal alignment; set caption width\\
% 15	2015-10-05	table units; radial parameter (N, T) profiles in literature\\
% 16	2015-10-06	literature radial electron density models; log-normal distribution\\
% 17	2015-10-07	log-normal formulae\\
% 18	2015-10-08	log-normal fit\\
% 19	2015-10-09	shape fit parameter table; radial variance fitting\\
20	2015-10-14	variance fit values; log-normal citation Bronstein\\
% 21	2015-10-19	radial distribution width fitting; clarifying of solar-wind model formulae structure and naming variables\\
% 22	2015-10-20	log-normal distribution width factor changes the shape!\\
% 23	2015-10-23	log-normal distributions mu and sigma plot\\
% 24	2015-10-30	new sigmafit2 figures implemented; formulae adaption\\
% 25	2015-11-02	reason for use of log-normal function\\
% 26	2015-11-20	single log-normal shape fit; not chisquare - SSR; $R^2$ in appendix\\
% 27	2015-11-26	reduced SSR into tables; table column aligning\\
% 28	2015-12-02	smoothing SSR integration\\
% 29	2015-12-07	integrating structure remarks from printout\\
30	2015-12-09	splitting analyses chapter into two; browsing folders for usable things for thesis\\
% 31	2015-12-10	integrating citations and references\\
% 32	2015-12-16	integrating citations and references\\
% 33	2015-12-18	solar-wind structures list of Richardson\\
% 34	2016-02-02	sws distributions\\
% 35	2016-02-11	basics structure\\
% 36	2016-02-12	analyses questions\\
% 37	2016-02-15	analyses structure; in situ, near-Sun\\
% 38	2016-02-16	Kp index internal resolution\\
% 39	2016-02-17	Bartels1962; non-breaking hyphen shorthand implemented\\
40	2016-02-18	analysesI structure\\
% 41	2016-02-25	abstract\\
% 42	2016-02-26	english comma\\
% 43	2016-03-01	Bartels Kp origin paper, bibtex reference\\
% 44	2016-03-02	abstract\\
% 45	2016-03-03	Planck paper\\
% 46	2016-03-07	formation of stars\\
% 47	2016-03-08	formation of stars\\
% 48	2016-03-16	solar interior structure\\
% 49	2016-03-17	solar interior structure\\
50	2016-03-18	solar surface\\
% 51	2016-03-22	Astronomical Almanac citation\\
% 52	2016-03-23	solar core definition\\
% 53	2016-03-26	sunspots\\
% 54	2016-03-29	Sun interior figure\\
% 55	2016-03-30	solar composition\\
% 56	2016-03-31	Sun interior figure finished; chromosphere\\
% 57	2016-04-01	Sun atmosphere figure finished\\
% 58	2016-04-03	chromosphere, corona, CV\\
% 59	2016-04-07	CV\\
% 60	2016-04-08	CV; heliosphere\\
% 61	2016-04-09	corona; heliosphere\\
% 62	2016-04-11	heliosheath magnetic bubbles; Opher2011\\
% 63	2016-04-12	heliosphere; voyagers; Gurnett2013\\
% 64	2016-04-13	solar-wind effects, solar rotation\\
65	2016-04-14	analysesI sws structure draft plots\\
% 66	2016-04-15	analysesI concept work\\
% 67	2016-04-19	incorporate annotations from processed printout\\
% 68	2016-04-21	incorporate annotations from processed printout II\\
% 69	2016-04-22	incorporate annotations from processed printout III\\
% 69	2016-04-23	updated .bst file (for displaying arXiv id); implemented optional url displaying for electronic version; Kp year variations\\
% 70	2016-04-24	Kp frequency by month statistics and plot; Sun and Earth tilt plot\\
% 71	2016-04-25	solar cycle extrema dates literature\\
% 72	2016-04-27	Kp semi annual variation, Rangarajan1997\\
% 73	2016-05-01	Kp frequency by year\\
% 74	2016-05-04	Kp + SSN plot; ACE data errors/gaps; log-normal plot; differential rotation plot\\
% 75	2016-05-05	aggregate correlation coefficient tables, CME fraction plot\\
% 76	2016-05-06	overall sws fractions\\
% 77	2016-05-17	figure sw frequencies by year; restructured solar-wind distributions section; formal cover page\\
% 78	2016-05-20	Earth-Sun distance; JPL web interface\\
% 79	2016-05-31	GSM system\\
80	2016-06-10	Offline Anmerkungen einarbeiten\\
% 81	2016-06-12	LaTeX Earth symbol package\\
% 82	2016-06-13	numbers=noenddot\\
% 83	2016-06-14	references removed comma before ampersand; offline Anmerkungen einarbeiten; thesis\_raw version\\
% 84	2016-06-17	gnuplot figures and latex; latex document style options\\
% 85	2016-06-19	header implemented\\
% 86	2016-06-20	minor things; introduction\\
% 87	2016-06-23	solar tilt axis\\
% 88	2016-06-24	solar rotation\\
% 89	2016-07-12	magnetosphere temp figure; analyses: solar-wind model\\
% 90	2016-07-13	lognormal spelling\\
% 91	2016-07-14	constant mass flow rate\\
% 92	2016-07-18	line-up periods\\
% 93	2016-07-19	line-up periods table\\
% 94	2016-07-20	redone line-up analysis\\
% 95	2016-07-21	line-up tables\\
% 96	2016-07-22	line-up fit function plots; coefficient table\\
% 97	2016-07-22	line-up sw types and period plots; overview plot\\
% 98	2016-07-24	line-up averaging and justification\\
% 99	2016-07-29	line-up rework\\
100	2016-07-30	line-up structure\\
% 101	2016-07-31	line-up table\\
% 102	2016-08-02	line-up structure\\
% 103	2016-08-03	line-up fits and table\\
% 104	2016-08-04	insert offline comments\\
% 105	2016-08-15	literature radial sw functions\\
% 106	2016-08-18	work offline comments in\\
% 107	2016-08-19	work offline comments in\\
% 108	2016-08-28	analyses II rework structure\\
% 109	2016-08-31	analyses II rework structure\\
% 110	2016-09-01	analyses II rework figures\\
% 111	2016-09-02	analyses II rework structure\\
% 112	2016-09-03	Helios latitude dependency\\
% 113	2016-09-04	extrapolation model fit parameter table; Helios latitude dependence\\
% 114	2016-09-06	work in offline comments\\
% 115	2016-09-14	implement median and mean in composite fit table; create kile project for thesis; url and raw fork\\
% 116	2016-09-17	Ulysses polar plot; solar distance plots\\
% 117	2016-09-18	for solar distance and latitude -> 4-panel figures created and implemented; appendix figures\\
% 118	2016-09-20	latitude plots updated\\
% 119	2016-09-26	single lognormal fit figure\\
120	2016-09-27	minor tweaks in analyses II\\
% 121	2016-09-29	simple fit table errors; table style\\
% 122	2016-10-05	lognormal fit errors for tables\\
% 123	2016-10-10	line-up passing remarks from AP\\
% 124	2016-10-12	siunitx package incorporated\\
% 125	2017-01-04	package fancyhdr replaced with scrlayer-scrpage; offline comments for Basics + Data incorporated\\
% 126	2017-01-06	package floatrow included for figure side captions; some cites; magnetometer\\
% 127	2017-01-08	sun figure updates and new ion energy spectrum figure incorporated\\
% 128	2017-01-13	offline comments, new chapter order\\
% 129	2017-01-18	activated hyperref package; included autoref and eqref; adjusted link colors; header links\\
% 130	2017-02-02	offline commtents\\
% 131	2017-02-03	offline commtents II, eclipse image\\
% 132	2017-02-06	basics figures with floatrow; flow structure\\
% 133	2017-02-07	ion energy spectrum figure\\
% 134	2017-02-09	Helios mission ssn plot\\
% 135	2017-02-12	offline comments\\
% 136	2017-02-13	section number link to toc; offline comments\\
% 137	2017-02-14	eclipse and magbfly figure captions\\
% 138	2017-02-16	offline comments\\
% 139	2017-02-19	coronal heating and sw acceleration\\
140	2017-04-22	set up git repository\\
% 141	2017-04-23	incorporate offline comments\\
% 142	2017-04-26	bipolar region draft\\
% 143	2017-04-27	magnetic layer Ossendrijver2003\\
% 144	2017-06-29	magnetosphere figure drafts implemented\\
% 145	2017-07-16	streamlined content - including paper and chapter2\\
% 146	2017-07-22	streamlined duplicate content\\
% 147	2017-10-05	Kp index data\\
% 148	2017-10-28	geomagnetic indices; Kp index; Kp to ap table\\
% 149	2017-10-29	Kp index\\
% 150	2017-10-30	OMNI s/c coverage figure\\
% 151	2017-10-31	OMNI s/c coverage figure upgrade; data set timeline figure\\
% 152	2017-11-01	SSN prediction; omega-effect\\
% 153	2017-11-02	alpha-omega dynamo; magnetogram figure upgrade\\
% 154	2017-11-03	worked on figure permissions; alpha-omega-figure\\
% 155	2017-11-04	paperMVVB inserted; worked on compatibility\\
% 156	2017-11-12	paperMVVB: figure size, tablefoot, adjusted table width; included chapter2; ordered file implementing\\
% 157	2017-11-14	figure permission Babaszkiewicz1998\\
% 158	2017-11-15	poking around in image credits; solar activity cycle\\
% 159	2017-11-16	solar activity cycle; Zhao image permission\\
160	2017-11-17	Zhao copyright symbol included; solar-wind; commata and sentence wording\\
% 161	2017-11-22	notes from github paper folder\\
% 162	2017-11-23	Ulysses figure permission and caption; pdfpages package; paper included as is\\
% 163	2017-11-24	GUM reference; Helios data frequency figure\\
% 164	2017-11-25	intro to paper\\
% 165	2017-11-26	intro to paper\\
% 167	2017-11-27	Cranmer2005 fig1; Banaszkiewicz fig\\
% 168	2017-11-28	HCS and Parker spiral\\
% 169	2017-11-29	heliosheath; Cranmer2005 permission\\
% 170	2017-11-30	chapter2 updated\\
% 171	2017-12-06	upgraded: title, abstract, introduction, basics outline, notes to chapter2; worked in offline comments until HMF\\
% 172	2017-12-08	MBPs; Solar wind section structure\\
% 173	2017-12-10	solar wind basics, draft figs for CIR and CME\\
% 174	2017-12-11	draft figs for COR2 CME and ACE MC; solar wind history\\
% 175	2017-12-13	B-field improvement\\
% 176	2017-12-14	B-field improvement \#2\\
% 177	2017-12-15	B-field improvement \#3\\
% 178	2017-12-16	B-field angles plot\\
% 179	2017-12-17	B-field improvement \#4\\
180	2017-12-21	B-field: four plots upgraded\\
% 181	2017-12-31	B-field: worked offline comments in; included chapter2 update\\
% 182	2018-01-01	autoref names; solar wind properties; literature\\
% 183	2018-01-02	sw helium share; plasma beta; source surface; mass flux considerations\\
% 184	2018-01-04	sw considerations; fast/slow wind\\
% 185	2018-01-05	fig Hundhausen; fast slow sw\\
% 186	2018-01-08	fast slow sw\\
% 187	2018-01-09	slow sw sources\\
% 188	2018-01-10	slow sw; slow/fast cycle pattern\\
% 189	2018-01-11	Wang1990\\
% 190	2018-01-14	HCS and SIR\\
% 191	2018-01-15	HCS + CIR\\
% 192	2018-01-16	in-situ event plots\\
% 193	2018-01-17	in-situ CIR/HSS plot\\
% 194	2018-01-19	CME/MC plot; VB comments\\
% 195	2018-01-20	VB comments...\\
% 196	2018-01-21	VB comments; Alfvén waves\\
% 197	2018-01-22	VB comments; removed line-up study; merged appendix; removed unused tex and image files\\
% 198	2018-01-24	HMF offline comments\\
% 199	2018-01-26	B improvement offline comments; APL PSP pictures; adjust pdfinfo\\
200	2018-01-28	chapter2 implemented into thesis; figure folders rearranged; figures floatrow implemented\\
% 201	2018-01-29	abstract updated; contents structure streamlined\\
% 202	2018-01-29b	instrumentation and data sources\\
% 203	2018-01-30	OMNI and Helios data\\
% 204	2018-01-31	chapter2 R-M effect literature\\
% 205	2018-01-31b	R-M effect literature; chapter2 coupling into basics\\
% 206	2018-02-01	formal titlepage\\
% 207	2018-02-03	front matter; pdf toc bookmarks\\
% 208	2018-02-05	sw in-situ fig\\
% 209	2018-02-06	SIRs\\
210	2018-02-07	SIRs II\\
211	2018-02-09	CMEs\\
212	2018-02-11	CMEs II\\
213	2018-02-11b	CMEs IIb\\
214	2018-02-12	CMEs III\\
215	2018-02-13	CMEs IV\\
216	2018-02-14	CMEs V\\
217	2018-02-15	CMEs VI\\
218	2018-02-16	CME images...\\
\vspace{\baselineskip}
\ISOToday{} \thistime{} -- last save
\end{footnotesize}

\end{titlepage}

\clearpage


\vspace*{\fill}

%Zitat in Diplomarbeit:
%\textit{"[...], und während er sich solchermaßen in der unteren Region des Reiches der Wissenschaft bewegte, dort, wo dieses unmerklich in das Reich der Psychiater übergeht, eignete er sich schließlich doch eine ganze Menge durchaus nützlicher Kenntnisse an, [...]"}
%denn er wusste immer erstaunlich gut Bescheid,
%\noindent Zitat: Stanislaw Lem, "Die Stimme des Herrn", S.56.

\noindent \textit{``Despite the `Dr.' before his name, he had completed no course of study and received no degree. When people tried to pin him down about this, he would say that the letters were merely an abbreviation of his first name - Drummond - which he did not use. But it was as `Dr.' Sam Laserowitz that he appeared in a number of science-fiction magazines; he was also known, in the circles of the fans of that genre, as a lecturer, and spoke on `cosmic' themes at their many conferences and convention. Laserowitz's speciality was earthshaking discoveries, wich he happened upon two or three times a year. [...] We really have no idea what a multitude of con men and crackpots inhabit the domain that lies halfway between contemporary science and the insane asylum.''}
\vspace{\baselineskip}

\noindent Excerpt from Stanis\l aw Lem 1968, \textit{His Master's Voice} \citep[p.~38]{Lem1984}.
%quote vs quotation vs excerpt

\vspace{\stretch{4}}

\newpage


\begin{abstract}
	\section*{Abstract}	%or abstract=true
	\noindent This thesis analyzes and derives predictions of how strong solar wind and coronal mass ejections impact the terrestrial magnetosphere and how the solar-wind environment scales in the near-Sun region.
	%magnetosphere impact
	Near-Earth in-situ solar-wind measurements, consisting of 35~years of minutely OMNI data, are analyzed together with the time series of the planetary geomagnetic disturbance indicator \Kp{}. Correlations and functional dependencies are compiled with regard to nowcast the magnitude of the geomagnetic disturbances from in-situ solar-wind measurements and to forecast them from remotely observed solar-wind streams and CMEs.
	%solar activity
	Further, 53~years of hourly OMNI data and sunspot number data are correlated for the purpose of deriving functional dependencies with the state of the solar cycle for the key solar-wind parameters magnetic field strength, proton velocity, density, and temperature.
	%solar distance
	Solar-wind in-situ data from the Helios~1 and~2 missions, operational in the 1970s and orbiting the Sun within the distance range \SIrange{0.3}{1.0}{\au}, are analyzed and empirical solar-wind distance dependencies are derived. Additionally, in view of the planned near-Sun spacecraft mission Parker~Solar~Probe with a planned launch in mid~2018, the solar-wind environment is estimated down to its closest perihelion at \num{9.86}~solar radii.
\end{abstract}
%key words: solar wind; sun: corona; sun: heliosphere; Parker Solar Probe (PSP); magnetosphere; Kp index; space weather\\

\newpage

%\cleardoublepage
%\tableofcontents
\pdfbookmark[0]{Contents}{toc_pdf_bookmark}
{\let\cleardoublepage\clearpage\tableofcontents}	%open contents on left page
\label{sec:toc}
\cleardoublepage
%\listoffigures
%\listoftables

