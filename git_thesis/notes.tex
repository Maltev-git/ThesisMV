\section*{notes}

% thesis style:
%VB sensibilisieren in Bezug auf genutzten Style:
%	american english
%	1~au statt 1~AU
%	in situ statt in-situ
%	lognormal statt log-normal
%	R_S oder R_\odot??
%	Astronomy \& Astrophysics typography style:
%	http://cds.aanda.org/index.php?option=com_content&task=view&id=172&Itemid=216
% 

Fragen:\\
	- references vol \# fett drucken?\\
	- references journal kursiv drucken?\\
		- maybe style references different. like in space science reviews?\\
	- page \# oben lassen?\\
	- A5 format?\\

nice phrases:\\
	...this leads to the question wh... . The answer to this question is developed in the course of the next section.\\
	
write basics without third persons view ``we''\\

define if using astronomical symbols...\\
...and use style consistently\\

Letzte Änderungen:\\
	Genehmigungen für die Abb. besorgen!!\\
	check for topic sentences!\\
	Am Anfang der Arbeit den folgenden logischen Aufbau der Kapitel erlaeutern.\\
	Am Anfang jedes Kapitels den folgenden logischen Aufbau der Abschnitte erlaeutern.\\
	Beides auf Aenderungen ueberpruefen.\\
	print figures to check look of colors\\
	adjust figure width to textwidth...\\
	complete pdfinfo text...\\

	Abkuerzungen CME, usw. konsequent nutzen und beim ersten Auftauchen ausschreiben.\\

	englische Kommasetzung beachten!\\
	schauen, ob Kommata oder Punkte durch Semikolon ersetzt werden koennen\\
	Pruefen, ob Ueberschriften den Inhalt des Kapitels gut beschreiben/gut zum Kapitel passen.\\
	check if AE spelling consistently, search and replace: analyse, etc.\\
	check for thin spaces in numbers with 5 digits and more\\
	
	Header und footer fuer notes, seitenzahl, Thema und Kapitel einfuegen\\
	check gnuplot plot text sizes and finish figures\\
	
	check for changed links and update access date\\
	
	sinnlosen Satz irgendwo einfügen\\
	76~\%, ACME irgendwo einfügen (ACME is good...)\\
	comic strip aus thesis extrahieren\\
	put movie in header or footer\\

	create hyperref version including bibtex links\\
		adjust link colors and remove boxes... (hyperref package options)\\
		remove google-books links!\\
		maybe adjust References like in Hathaway2010\\


LaTeX commands:\\
block commenting: strg+d\\
srtg+shift+d\\

\begin{align}
	N_mean_0 = ee\,e\:e\;e\ e\quad e\qquad e\enskip e\\	%\enskip ist Zahlenbreite
	N_median_0 = ||\,|\:|\;|\ |\quad|\qquad|\enskip|
\end{align}

Provides generic commands \degree, \celsius, \perthousand, \micro{} and \ohm{} which work both in text and maths mode.\\

%quotes '" `` ''
\citet{Lem1984}, \citep{Lem1984}, ``citep[see][page 45]{Lem1984}'' => \citep[see][page 45]{Lem1984}

included persons via citations:
Volker Bothmer, Rainer Schwenn, Eckhard Bosman, Russel Howard, Manuela Temmer, Angelos Vourlidas, Craig DeForest, Tim Howard, Rodney Viereck\\

Style rules:\\
the Sun's/solar rotation period\\
Earth's/terrestrial\\
the Sun, sunspots; near-Sun measurements, near-relativistic electrons, in situ measurements without hyphen\\
64~s data; but 64"~second data\\
Therefore, a comma is set after certain adverbs at the beginning of a sentence.\\
lognormal or log-normal, not log normal\\

A\&A style:\\
Italics should never be used for units\\
italics should be avoided for the following: mathematical signs such as ``d'' (total differential), ``e'' (base of natural logarithm), ``i'' (imaginary unit), ``pi''\\
Physical constants such as the speed of light, the Boltzmann constant, the Hubble constant and the solar mass are also set in regular italics.\\
\begin{align}
	\text{e}^x &= 10~\text{km\,s}^{-1}\\
	\text{e}^x &= 10~\text{km\textperiodcentered s}^{-1}\\
	\text{e}^x &= 10~\text{km/s}\\
	k_\text{B}
\end{align}
Sample input:
20\,000 km, 1\,000\,000 s, HD 174\,638 1950--1985, p.~11--21, this -- written on a computer -- is now printed, this---written on a computer---is now printed, signal-to-noise ratio, early-type, metal-poor, non-relativistic $-30$~K, $-5\ ^{\circ}$C Dr.~h.c.~Rockefeller-Smith and Prof.~Dr.~Mallory

------------\\

-----------------------------------------------------------\\
N\"utzliche Beispiele f\"ur LaTeX-Kommandos:\\
Autob\-ahn\\	%soft hyphen (trennt das Wort bei Bedarf am Zeilenumbruch nur an der Stelle: Autob-   ahn)
X-Ray\\		%hyphen
X"~Ray\\	%non-breaking hyphen (works only for ngerman babel; use \mbox{X-Ray} instead
% \useshorthands{"}	%activate shorthand for non-breaking hyphen "~; needs activation of ngerman in babel package
% \addto\extrasenglish{\languageshorthands{ngerman}}
Dr.~Huber\\	%non-breaking space
100\,000\,V\\	%thin non-breaking space
100\,000~V\\	%both possible.?
The food---which was delicious---reminded me of home.\\
Red, white, and blue---these are the colors of the flag.\\

$U = 20$\,$000$\,V	%Masseinheiten (m, kg, ...), mathematische Konstanten (e, i, ...) und definierte Funktionen (sin, cos, log) Aufrecht schreiben. Den Rest Kursiv.

solar radius: $R_\odot = 696\,000$~km\\

%\earth, $\odot$, \astrosun\\	%wasysym package



\noindent 10\,\textdegree{}C\\
10\,$^\circ$C\\			%nicht verwenden.
10\,\textcelsius\\
Ein Winkel von 10\textdegree.\\

\begin{align}
	\frac{T_\text{D}}{[\text{K}]} &= 21,8\,\cdot\,\left( \frac{\Delta v_\text{FWHM}^\text{LMC}}{[\text{km\,s}^{-1}]} \right)^2\\
	\frac{T_\text{D}}{[\text{K}]} &= 21,8\,\cdot\,\left( \frac{\Delta v_{{}_\text{FWHM}}^{{}^\text{LMC}}}{[\text{km\,s}^{-1}]} \right)^2
\end{align}

\begin{align}
	k = 3.56 \text{\,e--6}\\
	k = 3.56 \cdot 10^{-6}\\
	k = 3.56 \times 10^{-6}
\end{align}

\begin{enumerate}
	\item CM LMC Longitude
	\item [drei] CM LMC Latitude
\end{enumerate}

\begin{description}
  	\item Volume in (\textdegree)\,$^2$m\,s$^{-1}$. Summe aller Voxel des Objektes.
  	\item [Volume] in (\textdegree)\,$^2$m\,s$^{-1}$. Summe aller Voxel des Objektes.
\end{description}

\begin{itemize}
 	\item Volume
	\item [sdgffs] CM GSR Velocity
\end{itemize}

\begin{list}{asdf}{\makelabel{label}}
 	\item Volume
	\item [drei] CM LMC Latitude
\end{list}

\footnote{All listed websites were existent on 2008-09-15.} oder \footnotemark \footnotetext{All listed websites were existent on 2008-09-15.}

\begin{equation}
	a^2 + b^2 = c^2
	\label{eq:pythagoras} 
\end{equation}
Formel zitieren: (Siehe Formel \ref{eq:pythagoras}, Formel \eqref{eq:pythagoras}, Seite \pageref{eq:pythagoras})\\

package siunitx:\\
\SI{20.5e-3}{\kilo\gram}\\
\SI{20.5e-3}{\kg}\\
\SI{20.5e-3}{kg}\\

%\begin{figure}[htb]
%	\centering
%	\includegraphics[width=0.5\textwidth]{pictures/AVG_full_MW_GAL_with_LA_field.jpg}
%	\caption{Dargestellt ist die H\i-S\"aulendichte in galaktischen Koordinaten. Grundlage ist der vollst\"andige GASS-Datensatz.}
%	\label{fig:AVG_full_MW_GAL_with_LA_field}
%\end{figure}
%Bild zitieren: (Siehe Abbildung \ref{fig:AVG_full_MW_GAL_with_LA_field})
	%sec:, fig:, tab:, eq:, chap:, lst:


width defines the width of the resulting box as seen from the outside (This means it can be \raisebox{0pt}[0pt][0pt]{\framebox[1.2\width]{smaller}} than the material inside the box. You can even set the width to 0pt so that the text inside the box will be typeset without influ%
\raisebox{0pt}[0pt][0pt]{%
\raisebox{-0.1ex}{en}%
\raisebox{-0.3ex}{c}%
\raisebox{-0.6ex}{i}%
\raisebox{-1.0ex}{n}%
\raisebox{-1.5ex}{g}%
\raisebox{-2.1ex}{ }%
\raisebox{-2.8ex}{t}%
\raisebox{-3.6ex}{h}%
\raisebox{-4.5ex}{e}%
\raisebox{-2ex}{ }%
\raisebox{2.8ex}{s}%
\raisebox{2.2ex}{u}%
\raisebox{1.8ex}{r}%
\raisebox{1.2ex}{r}%
\raisebox{0.8ex}{o}%
\raisebox{0.3ex}{u}}%
nding boxes). Besides the length expressions, you can also u


semi-log plots:\\
log-lin:	logarithmic scale on the y-axis, and a linear scale on the x-axis\\
lin-log:	logarithmic scale on the x-axis, and a linear scale on the y-axis\\
the naming is output-input (y-x), the opposite order from (x, y)


