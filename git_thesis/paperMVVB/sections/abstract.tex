
\titley{Solar wind predictions for the Parker Solar Probe orbit}
\subtitley{Near-Sun extrapolations derived from an empirical solar wind model based on Helios and OMNI observations}

% \author{M.~S.~Venzmer
% \and V.~Bothmer}
% 
% \institute{University of Goettingen, Institute for Astrophysics, Friedrich-Hund-Platz~1, 37077~Göttingen, Germany}
% 
% \date{Received 25 August 2017; accepted date}
% %First draft 9 August 2016; first changes 21 June 2017; second changes 17 July 2017; third changes 17 August 2017; fourth changes 24 August 2017; submitted 25 August 2017

\abstracty
{The Parker Solar Probe (PSP) (formerly Solar Probe Plus) mission will be humanity’s first in situ exploration of the solar corona with closest perihelia at \num{9.86}~solar radii (\si{\Rs}) distance to the Sun. It will help answer hitherto unresolved questions on the heating of the solar corona and the source and acceleration of the solar wind and solar energetic particles. The scope of this study is to model the solar wind environment for PSP’s unprecedented distances \textbf{in} its prime mission phase during the years \numrange{2018}{2025}. The study is performed within the project Coronagraphic German And US \textbf{SolarProbePlus} Survey (CGAUSS) which is the German contribution to the PSP mission as part of the Wide-field Imager for Solar PRobe (WISPR).}	%context
{We present an empirical solar wind model for the inner heliosphere which is derived from OMNI and Helios data. The German-US space probes Helios~1 and Helios~2 flew in the 1970s and observed solar wind in the ecliptic within heliocentric distances of \SIrange{0.29}{0.98}{\au}. The OMNI database consists of multi-spacecraft intercalibrated in situ data obtained near \SI{1}{\au} over more than five solar cycles. The international sunspot number (SSN) and its predictions are used to derive dependencies of the major solar wind parameters on solar activity and to forecast their properties for the PSP mission.}	%aims
{The frequency distributions for the solar wind key parameters magnetic field strength, proton velocity, density and temperature are represented by lognormal functions. In addition, we consider the velocity distribution’s bi-componental shape, consisting of a slower and a faster part. Functional relations to solar activity are compiled with use of the OMNI data by correlating and fitting the frequency distributions with the SSN. Further, based on the combined data set from both Helios probes, the parameters’ frequency distributions are fitted with respect to solar distance to obtain power law dependencies. Thus an empirical solar wind model for the inner heliosphere confined to the ecliptic region is derived, accounting for solar activity and for solar distance through adequate shifts of the lognormal distributions. Finally, the inclusion of SSN predictions and the extrapolation \textbf{down} to PSP’s perihelion \textbf{region} enables us to estimate the solar wind environment for PSP’s planned trajectory during its mission duration.}	%methods
{The CGAUSS empirical solar wind model for PSP yields dependencies \textbf{on solar activity and solar distance for the solar wind parameters' frequency distributions}. The estimated solar wind median values for PSP’s first perihelion in 2018 at a solar distance of \SI{0.16}{\au} are \SI{87}{\nT}, \SI{340}{\km\per\s}, \SI{214}{\per\cm\cubed} and \SI{503000}{\K}. The estimates for PSP’s \textbf{first closest perihelion, occuring} in 2024 at \SI{0.046}{\au} (\SI{9.86}{\Rs}), are \SI{943}{\nT}, \SI{290}{\km\per\s}, \SI{2951}{\per\cm\cubed} and \SI{1930000}{\K}. \textbf{Since the modeled velocity and temperature values below about \SI{20}{\Rs} appear overestimated in comparison with existing observations, this suggests that PSP will directly measure solar wind acceleration and heating processes below \SI{20}{\Rs} as planned. } }	%results
{}	%conclusions

% \keywords{solar wind -- sun: heliosphere -- sun: corona}
% 
% \maketitle

%\titlerunning{Solar wind extrapolation to PSP orbit}
%\authorrunning{Venzmer \& Bothmer}


%title and abstract for submitting:

%Solar wind predictions for the Parker Solar Probe orbit
%Near-Sun extrapolations derived from an empirical solar wind model based on Helios and OMNI observations

%The Parker Solar Probe (PSP) (formerly Solar Probe Plus) mission will be humanity’s first in situ exploration of the solar corona with closest perihelia at 9.86 solar radii (Rs) distance to the Sun. It will help answer hitherto unresolved questions on the heating of the solar corona and the source and acceleration of the solar wind and solar energetic particles. The scope of this study is to model the solar wind environment for PSP’s unprecedented distances during its prime mission phase during the years 2018--2025. The study is performed within the project Coronagraphic German And US Solar Probe Survey (CGAUSS) which is the German contribution to the PSP mission as part of the Wide field Imager for Solar PRobe (WISPR).
%We present an empirical solar wind model for the inner heliosphere which is derived from OMNI and Helios data. The German-US space probes Helios 1 and Helios 2 flew in the 1970s and observed solar wind in the ecliptic within heliocentric distances of 0.29--0.98 au. The OMNI database consists of multi-spacecraft intercalibrated in situ data obtained near 1 au over more than five solar cycles. The international sunspot number (SSN) and its predictions are used to derive dependencies of the major solar wind parameters on solar activity and to forecast their properties for the PSP mission.
%The frequency distributions for the solar wind key parameters magnetic field strength, proton velocity, density and temperature are represented by lognormal functions. In addition, we consider the velocity distribution’s bi-componental shape, consisting of a slower and a faster part. Functional relations to solar activity are compiled with use of the OMNI data by correlating and fitting the frequency distributions with the SSN. Further, based on the combined data set from both Helios probes, the parameters’ frequency distributions are fitted with respect to solar distance to obtain power law dependencies. Thus an empirical solar wind model for the inner heliosphere confined to the ecliptic region is derived, accounting for solar activity and for solar distance through adequate shifts of the lognormal distributions. Finally, the inclusion of SSN predictions and the extrapolation to PSP’s perihelion enables us to estimate the solar wind environment for PSP’s planned trajectory during its mission duration.
%The CGAUSS empirical solar wind model for PSP yields dependencies of the solar wind parameters on solar activity and radial distance. The estimated solar wind median values for PSP’s first perihelion in 2018 at a solar distance of 0.16 au are 87 nT, 340 km s-1, 4015 cm-3 and 503000 K. The estimates for PSP’s closest perihelia, beginning in 2024 at 0.046 au (9.86 Rs), are 943 nT, 290 km s-1, 9733 cm-3 and 1930000 K. Though, the modeled velocity and temperature values below about 20 Rs appear overestimated in comparison with existing observations. Thus, PSP is expected to directly measure solar wind acceleration and heating processes below 20 Rs as planned.
